
Assume that we want to maximize the network throughput, based
on the considerations of the previous section.  Consider the
network example in Figure~\ref{D31-f3}; one source sends at rate $x_0$Mb/s, one at rate $x_1$Mb/s and nine sources at rate $x_2$Mb/s each.  We assume that we implement some form
of congestion control and that there are negligible losses.
Thus, the flow, in Mb/s, on the first link $i$ is $x_{0}+x_{1}$, and on the second link it is $x_{0}+9x_{2}$.   For a
given value of $x_{0}$, maximizing the throughput
requires that $x_{1}= 10 - x_{0}$  and $9x_{2}= 10 - x_{0}$.
The total throughput, measured at the network output, is then
$x_0+x_1+9x_2=20-x_0$; it is maximum for $x_{0} = 0$~!

\begin{figure}[htbp]
        \insfig{D31F3a}{0.7}
        \mycaption{A simple network used to illustrate fairness and efficiency. }
        \protect\label{D31-f3}
\end{figure}

The example shows that maximizing network throughput as a
primary objective may lead to gross unfairness; in the worst
case, some sources may get a zero throughput, which is
probably considered unfair by these sources.

In general, the concept of efficiency is captured by the notion of \emph{Pareto Efficiency}. Consider an allocation problem; define the vector $\vec{x}$
whose $i$th coordinate is the allocation for user $i$. %Let $\calX$ be the set of all feasible allocations.
\begin{definition}[Pareto Efficiency]
       A feasible allocation of rates
$\vec{x}$ is ``Pareto-Efficient'' (also called ``Pareto-Optimal'') if and only if an increase of
any rate within the domain of feasible allocations must be at
the cost of a decrease of some other rate. Formally,
for any other feasible allocation $\vec{y}$, if $y_{s} >
x_{s}$ then there must exist some $s'$ such that $y_{s'} < x_{s'}$. \label{def-paretoeff}
\end{definition}

In general, there exist many Pareto-efficient allocations. For the example in \fref{D31-f3}, any allocation that saturates every link (i.e., such that $x_0+x_{1}=$  and $x_0+9x_{2}= 10$) is Pareto-efficient. The allocation that maximizes total throughput (and has $x_0=0$) is one of them; another Pareto efficient allocation is $x_0=0.1, x_1=9.9, x_2=1.1$; yet another one is $x_0=1, x_1=9, x_2=1$. Among all of these allocations, which one should a fair and efficient congestion control scheme choose~? This requires a formal definition of fairness, given in the next section.
