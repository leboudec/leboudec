

In an egalitarian vision, fairness would simply mean allocating the same
share to all, while maximizing this share. In the simple case of Figure~\ref{D31-f3}, this would
mean allocating $x_0=x_1=x_2=1$. However, such an egalitarian allocation is not Pareto-efficient (because it is possible to unilaterally increase $x_1$).

A better allocation could be based on the following observation. Starting from the egalitarian allocation, we can increase the rate $x_1$ without impacting other rates, up to the value $x_1=9$. The resulting allocation $x_0=1, x_1=9, x_2=1$ \emph{is} Pareto-efficient, and still appears to be as fair as possible. It turns out that this allocation is exactly the Max-Min fair allocation, which is defined next.

%Consider again the example of Figure~\ref{D31-f3}, now with
%general values for $n_{i}$. If we follow the previous line of
%reasoning, we would allocate the fraction
%$\frac{c}{n_{0}+n_{i}}$ to each of the $n_{0}+n_{i}$ sources
%using link $i$.  This yields $x_{i}=\frac{c}{n_{0}+n_{i}}$ for
%$i \geq 1$; for $i=0$, the reasoning of the previous section
%indicates that we should allocate $x_{0}= \min_{1 \leq i \leq
%I} \frac{c}{n_{0}+n_{i}}$. For example, with $I=2$,
%$n_{0}=n_{1}=1$ and $n_{2}=9$, we would allocate $x_{0}=0.1
%c$, $x_{1}=0.5 c$ and $x_{2}=0.1 c$. This allocation however
%would not fully utilize link $1$; we could decide to increase
%the share of sources of type 1 since this can be done without
%decreasing the shares of other sources. Thus, a final
%allocation could be $x_{0}=0.1 c$, $x_{1}=0.9 c$ and
%$x_{2}=0.1 c$. We have illustrated that allocating resources
%in an equal proportion is not a good solution since some
%sources can get more that others without decreasing others'
%shares. Formally, this leads to our first definition of
%fairness called max-min fairness.

%Consider an allocation problem; define the vector $\vec{x}$
%whose $i$th coordinate is the allocation for user $i$. Let
%$\calX$ be the set of all feasible allocations.
%Recall that the $i$th coordinate of vector $\vec{x}$
%is the rate allocated to user $i$ and $\calX$ is the set of all feasible allocations.

\begin{definition}[Max-min Fairness \cite{BG}] A feasible allocation of rates
$\vec{x}$ is ``max-min fair'' if and only if an increase of
any rate within the domain of feasible allocations must be at
the cost of a decrease of some already smaller rate. Formally,
for any other feasible allocation $\vec{y}$, if $y_{s} >
x_{s}$ then there must exist some $s'$ such that $x_{s'} \leq
x_{s}$ and $y_{s'} < x_{s'}$. \label{def-mmf}
\end{definition}
The name
``max-min'' comes from the idea that it is forbidden to
decrease the share of sources that have small values, thus, in
some sense, we give priority to flows with small values.

\begin{theorem}A max-min fair allocation is Pareto-efficient.
\end{theorem}
\pr If we increase one rate in a max-min fair allocation, we must decrease some other rate, which expresses that the allocation is Pareto-efficient.
\qed

The converse is not true -- a Pareto-efficient allocation is not, in general, max-min fair. In fact, as we will see next, the max-min fair allocation is unique, whereas there are usually many Pareto-efficient allocations -- see the in \fref{D31-f3}.

\begin{theorem}A max-min fair allocation, if it exists, is unique.
\end{theorem}
\pr Assume now that
$\vec{x}$ and $\vec{y}$ are two max-min fair allocations for
the same problem, with $\vec{x} \neq \vec{y}$. Without loss of
generality, we can assume that there exists some $i$ such that
$x_{i} < y_{i}$. Consider the smallest value of $x_{i}$ that
satisfies  $x_{i} < y_{i}$, and call $i_{0}$ the corresponding
index. Thus, $x_{i_{0}} < y_{i_{0}}$ and
\begin{equation}
        \mif  x_{i} < y_{i} \mthen x_{i_{0}} \leq x_{i}
        \label{eq-preuve-uni-mmf}
\end{equation}
Now since $\vec{x}$ is max-min fair, from
Definition~\ref{def-mmf}, there exists some $j$ with
\begin{equation}
        y_{j}<x_{j} \leq x_{i_{0}}
        \label{eq-preuve-uni-mmf-2}
\end{equation}
Now $\vec{y}$ is also max-min fair, thus by the same token
there exists some $k$ such that
\begin{equation}
        x_{k}<y_{k} \leq y_{j}
        \label{eq-preuve-uni-mmf-3}
\end{equation}
Combining (\ref{eq-preuve-uni-mmf-2}) and
(\ref{eq-preuve-uni-mmf-3}), we obtain
$$
x_{k}<y_{k} \leq y_{j}<x_{j} \leq x_{i_{0}}
$$
which contradicts (\ref{eq-preuve-uni-mmf}). \qed
\qed

For some allocation problems, a max-min fair allocation might
not exist. However, for the practical examples of interest in the context of congestion control, there is always one (and only one) max-min fair allocation, as we show next. For more general cases,
see \cite{radunovic2007ufm}, where it is shown in particular that a max-min fair allocation always exists when the feasible set is convex and compact.

\paragraph{Network Model}

We use the following simplified network model in the rest of
this section. We consider a set of sources $s=1, \ldots, S$
and links $1,\ldots,L$. Let $A_{l,s}$ be the fraction of
traffic of source $s$ which flows on link $l$, and let $c_{l}$
be the capacity of link $l$. We define a network as the couple
$(\vec{x}, A)$.

A \emph{feasible allocation} of rates $x_{s}\geq 0$ is defined
by: $\sum_{s=1}^{S}A_{l,s}x_{s} \leq c_{l}$ for all $l$.

Our network model supports both multicast and load sharing.
For a given source $s$, the set of links $l$ such that
$A_{l,s} > 0$ is the path followed by the data flow with
source $s$. In the simplest case (no load sharing), $A_{l,s}
\in \{0, 1\}$; if a flow from source $s$ is equally split
between two links $l_{1}$ and $l_{2}$, then $A_{l_{1},s} =
A_{l_{2},s} =0.5$. In principle, $A_{l,s} \leq 1$, but this is
not mandatory (in some encapsulation scenarios, a flow may be
duplicated on the same link).



%It can be seen (and this is left as an exercise) that the
%allocation in the previous example is max-min fair.

We now show that there exists a
max-min fair allocation to our network model, and how to
obtain it. This will result from the key concept of
``bottleneck link''.

\begin{definition}[Bottleneck Link]
        With our network model above, we say that link $l$ is a bottleneck
        for source $s$ if and only if
        \begin{enumerate}
                \item  link $l$ is saturated: $c_{l}= \sum_{i}A_{l,i}x_{i}$

                \item  source $s$ on link $l$ has the maximum rate among all
                sources using link $l$:
                $x_{s}\geq x_{s'}$ for all $s'$ such that $A_{l,s'} > 0$.
        \end{enumerate}
\end{definition}

Intuitively, a bottleneck link for source $s$ is a link which
is limiting, for a given allocation.  In the previous
example, the first link is a bottleneck for blue flow with rate $x_1$ and for the red flow with rate $x_0$; the second link is a bottleneck for each of the nine green flows with rates $x_2$ and for the red flow with rate $x_0$.

\begin{theorem}
        A feasible allocation of rates $\vec{x}$ is max-min fair if and
        only if every source has a bottleneck link.
        \label{theo-bn}
\end{theorem}
\pr \textbf{Part 1.}  Assume that every source has a bottleneck
link. Consider a source $s$ for which we can increase the rate
$x_{s}$ while keeping the allocation feasible. Let $l$ be a
bottleneck link for $s$. Since $l$ is saturated, it is necessary
to decrease $x_{s'}$ for some $s'$ such that $A_{l,s'} >0$. We
assumed that we can increase the rate of $s$: thus there must
exist some $s' \neq s$ that shares the bottleneck link $l$. But
for all such $s'$, we have $x_{s}\geq x_{s'}$, thus we are forced
to decrease $x_{s'}$ for some $s'$ such that $x_{s}\geq x_{s'}$:
this shows that the allocation is max-min fair.

\textbf{Part 2.} Conversely, assume that the allocation is max-min
fair.  For any source $s$, we need to find a bottleneck link.  We
proceed by contradiction.  Assume there exists a source $s$ with
no bottleneck link.  Call $L_{1}$ the set of saturated links used
by source $s$, namely, $L_{1} = \{l \mst c_{l}=
\sum_{i}A_{l,i}x_{i} \mand A_{l,s} > 0 \}$.  Similarly, call
$L_{2}$ the set of non-saturated links used by source $s$. Thus a
link is either in $L_{1}$ or $L_{2}$, or is not used by $s$.
Assume first that $L_1$ is non-empty.

\begin{figure}[h]
        \insfig{D31F6}{0.5}
        \mycaption{A network example showing one multicast source}
        \protect\label{fig-sigma}
\end{figure}

By our assumption, for all $l \in L_{1}$ , there exists some
$s'$ such that $A_{l,s'} >0$ and $x_{s'} > x_{s}$. Thus we can
build a mapping $\sigma$ from $L_{1}$ into the set of sources
$\{1,\ldots,S \}$ such that $A_{l,\sigma(l)} >0$ and
$x_{\sigma(l)} > x_{s}$ (see Figure~\ref{fig-sigma} for an
illustration). Now we will show that we can increase the rate
$x_{s}$ in a way that contradicts the max-min fairness
assumption. We want to increase $x_{s}$ by some value
$\delta$, at the expense of decreasing $x_{s'}$ by some other
values $\delta_{s'}$, for all $s'$ that are equal to some
$\sigma(l')$. We want the modified allocation to be feasible;
to that end, it is sufficient to have:

\begin{eqnarray}
         A_{l,s} \delta \leq A_{l, \sigma(l)} \delta_{\sigma(l)} \; \mfa l \in
        L_{1}
        \label{eq-mmf-1}  \\
        A_{l,s} \delta  \leq c_{l} - \sum_{i}A_{l,i}x_{i} \; \mfa l \in L_{2}
        \label{eq-mmf-2}  \\
        \delta_{\sigma(l)} \leq x_{\sigma(l)} \; \mfa l \in L_{1}
        \label{eq-mmf-3}
\end{eqnarray}
Equation~(\ref{eq-mmf-1}) expresses that the increase of flow
due to source $s$ on a saturated link $l$ is at least
compensated by the decrease of flow due to source $\sigma(l)$.
Equation~(\ref{eq-mmf-2}) expresses that the increase of flow
due to source $s$ on a non-saturated link $l$ does not exceed
the available capacity. Finally, equation~(\ref{eq-mmf-3})
states that rates must be non-negative.

This leads to the following choice.
\begin{equation}
        \delta = \min_{l \in L_{1}}
        \{  \frac{x_{\sigma(l)} A_{l, \sigma(l)}}{A_{l,s}}\}
        \wedge
        \min_{l \in L_{2}}
        \{  \frac{c_{l}-\sum_{i}A_{l,i}x_{i}}{A_{l,s}}\}
        \label{eq-mmf-4}
\end{equation}
which ensures that Equation~(\ref{eq-mmf-2}) is satisfied and
that $\delta >0$.

In order to satisfy Equations~(\ref{eq-mmf-1}) and
(\ref{eq-mmf-3}) we need to compute the values of
$\delta_{\sigma(l)}$ for all $l$ in $L_{1}$. Here we need to
be careful with the fact that the same source $s'$ may be
equal to $\sigma(l)$ for more than one $l$. We define
$\delta(s')$ by

\begin{eqnarray}
         \delta(s') & = 0  \textrm{ if there is no } l \mst s'=\sigma(l)
        \label{eq-mmf-50}  \\
        \delta(s') & =
                \max_{\{l \mst \sigma(l) = s'\}}
                \{ \frac{\delta A_{l,s}}{A_{l, \sigma(l)}} \}
                \textrm{ otherwise }
        \label{eq-mmf-5}
\end{eqnarray}

This definition ensures that Equation~(\ref{eq-mmf-1}) is
satisfied. We now examine Equation~(\ref{eq-mmf-3}). Consider
some $s'$ for which there exists an $l$ with $\sigma(l)=s$,
and call $l_{0}$ the value which achieves the maximum in
(\ref{eq-mmf-5}), namely:
\begin{equation}
        \delta(s') = \frac{\delta A_{l_{0},s}}{A_{l_{0}, s'}}
        \label{eq-mmf-6}
\end{equation}
From the definition of $\delta$ in (\ref{eq-mmf-4}), we have
$$
\delta \leq \frac{x_{\sigma(l_{0})}A_{l_{0}, \sigma(l_{0})} }{
A_{l_{0},s}}
=
\frac{x_{s'}A_{l_{0}, s'} }{ A_{l_{0},s}}
$$
Combined with (\ref{eq-mmf-6}), this shows that
Equation~(\ref{eq-mmf-3}) holds. In summary, we have shown
that we can increase $x_{s}$ at the expense of decreasing the
rates for only those sources $s'$ such that $s'=\sigma(l)$ for
some $l$. Such sources have a rate higher than $x_{s}$, which
shows that the allocation $\vec{x}$ is not max-min fair and
contradicts our hypothesis.

It remains to examine the case where $L_1$ is empty. The
reasoning is the same, we can increase $x_s$ without
decreasing any other source, and we also have a contradiction.
\qed

\paragraph{The Water-Filling Algorithm}

The previous theorem is particularly useful in deriving a
practical method for obtaining a max-min fair allocation,
called ``progressive filling'' or ``water filling''.  The idea is as follows.  You
start with all rates equal to 0 and grow all rates together at
the same pace, until one or several link capacity limits are
hit.  The rates for the sources that use these links are not
increased any more, and you continue increasing the rates for
other sources.  All the sources that are stopped have a
bottleneck link.  This is because they use a saturated link,
and all other sources using the saturated link are stopped at
the same time, or were stopped before, thus have a smaller or
equal rate.  The algorithm continues until it is not possible
to increase.  The algorithm terminates because $L$ and $S$ are
finite.  Lastly, when the algorithm terminates, all sources
have been stopped at some time and thus have a bottleneck
link.  By application of Theorem~\ref{theo-bn}, the allocation
is max-min fair.

\paragraph{Example}
Let us apply the water-filling algorithm to \fref{D31-f3}.

We first let $x_{0}=x_1=x_2=t$ and increase $t$ until we hit a limit, i.e. we maximize $t$ subject to the constraints imposed by the link capacities. The constraints are
$$
x_0+x_1\leq 10\mand x_1 +9x_2\leq 10$$
which gives
$$
2t\leq 10 \mand 10t\leq 10$$
The maximum is for $t=1$Mb/s and at this value the second constraint is hit, i.e., the second link is a bottleneck link. The sources using this link are the sources of type 0 and 2, therefore we let
$$
x_0=1\mand x_2=1
$$
and these values are final.

In a second round we increase the rates of the other sources, namely we increase $x_1$ until a capacity limit is hit. There is only one constraint left, that of the first link, which is now expressed as
$$1 + x_1\leq 10
$$
The maximal value of $x_1$ is $9$. With this, all sources are stopped and the algorithm terminates. The max-min fair allocation is thus $x=1, x_1=9, x_2=1$.

We see that all sources of type $0$ and $2$ obtain
the same rate. In some sense, max-min fairness ignores the
fact that sources of type $0$ use more network resources than
those of type $2$.

\begin{theorem}For the network defined above,
 with fixed routing parameters $A_{l,s}$, there exists a unique
        max-min fair allocation.  It can be obtained by the water-filling algorithm.
\end{theorem}
\pr We have already proven uniqueness.  Existence follows from the water-filling algorithm. \qed


% Pb avec fairness (exemple de Jim Roberts) avec shortest packet first

The notion of max-min fairness can be generalized by
using weights in the definition \cite{BG, MR98}. 
