% modified LEB Nov 2002, based on input by BOZ
\chapter{Congestion Control for Best Effort: Theory}
\label{d31}
In this chapter you learn
\begin{itemize}
        \item  why it is necessary to perform congestion control

        \item  the additive increase, multiplicative decrease rule

        \item  the three forms of congestion control schemes

        \item  max-min fairness, proportional fairness

\end{itemize}

This chapter gives the theoretical basis required for Congestion Control in the Internet.

\section{The objective of congestion control}

\subsection{Congestion Collapse}
\label{sec-d31-why}

Consider a network where sources may send at a rate limited
only by the source capabilities. Such a network may suffer of
congestion collapse, which we explain now on an example.

We assume that the only resource to allocate is link bit
rates. We also assume that if the offered traffic on some link
$l$ exceeds the capacity $c_{l}$ of the link, then all sources
see their traffic reduced in proportion of their offered
traffic. This assumption is approximately true if queuing is
first in first out in the network nodes, neglecting possible
effects due to traffic burstiness.

Consider first the network illustrated on Figure~\ref{D31-f1}.
Sources 1 and 2 send traffic to destination nodes D1 and D2
respectively, and are limited only by their access rates.
There are five links labeled 1 through 5 with capacities shown
on the figure. Assume sources are limited only by their first
link, without feedback from the network. Call $\lambda_i$ the
sending rate of source $i$, and $\lambda'i$ the outgoing rate.

For example, with the values given on the figures we find
$\lambda_1=100$kb/s and $\lambda_2=1000$kb/s, but only
$\lambda'_1=\lambda'_2=10$kb/s, and the total throughput is
$20$kb/s~! Source 1 can send only at 10 kb/s because it is
competing with source 2 on link 3, which sends at a high rate
on that link; however, source 2 is limited to 10 kb/s because
of link 5. If source 2 would be aware of the global situation,
and if it would cooperate, then it would send at 10 kb/s only
already on link 2, which would allow source 1 to send at 100
kb/s, without any penalty for source 2. The total throughput
of the network would then become $\theta = 110 kb/s$.

\begin{figure}[htbp]
        \insfig{D31F1}{0.7}
        \mycaption{A simple network exhibiting some inefficiency if sources
        are not limited by some feedback from the network}
        \protect\label{D31-f1}
\end{figure}

The first example has shown some inefficiency. In complex
network scenarios, this may lead to a form of instability
known as congestion collapse. To illustrate this, we use the
network illustrated on Figure~\ref{D31-f1n}. The topology is a
ring; it is commonly used in many networks, because it is a
simple way to provide some redundancy. There are $I$ nodes and
links, numbered $0, 1, ..., I-1$. Source $i$ enters node $i$,
uses links $[(i+1) \bmod I] $ and $[(i+2) \bmod I]$, and
leaves the network at node $(i+2) \bmod I$. Assume that source
$i$ sends as much as $\lambda_{i}$, without feedback from the
network. Call $\lambda'_{i}$ the rate achieved by source $i$
on link $[(i+1) \bmod I]$ and $\lambda''_{i}$ the rate
achieved on link $[(i+2) \bmod I]$. This corresponds to every
source choosing the shortest path to the destination. In the
rest of this example, we omit ``$\bmod I$" when the context is
clear. We have then:
\begin{equation}
 \left \{
  \begin{array}{l}
  \lambda'_i = \min
     \left(\lambda_i, \frac{c_i}{\lambda_i +
     \lambda'_{i-1}}\lambda_i
     \right) \\
  \lambda''_i = \min
     \left(\lambda'_i, \frac{c_{i+1}}{\lambda'_i +
     \lambda_{i+1}}\lambda'_i
     \right) \\
  \end{array}
  \right.
    \mylabel{D31-eqkkkLAUW}
\end{equation}

\begin{figure}[htbp]
        \insfig{D31F1n}{0.35}
        \mycaption{A network exhibiting congestion collapse if sources
        are not limited by some feedback from the network}
        \mylabel{D31-f1n}
\end{figure}


Applying Equation~\ref{D31-eqkkkLAUW} enables us to compute
the total throughput $\theta$. In order to obtain a closed
form solution, we further study the symmetric case, namely, we
assume that $c_i=c$ and $\lambda_i= \lambda$ for all $i$. Then
we have obviously $\lambda'_i= \lambda'$ and $\lambda''_i=
\lambda''$ for some values of $\lambda'$ and $\lambda''$ which
we compute now.

If $\lambda \leq \frac{c}{2}$ then there is no loss and
$\lambda''=\lambda'=\lambda$ and the throughput is $\theta = I
\lambda$. Else, we have, from Equation~(\ref{D31-eqkkkLAUW})
$$
 \lambda' = \frac{c \lambda}{\lambda + \lambda'}
$$
We can solve for $\lambda'$ (a polynomial equation of degree
2) and obtain
$$
\lambda'=\frac{\lambda}{2} \left( -1 + \sqrt{1 + 4
\frac{c}{\lambda}}\right)
$$
We have also from Equation~(\ref{D31-eqkkkLAUW})
$$
 \lambda'' = \frac{c \lambda'}{\lambda + \lambda'}
$$
Combining the last two equations gives
$$ \lambda'' = c -\frac{\lambda}{2}\left( \sqrt{1 + 4 \frac{c}{\lambda}} -1 \right)
$$
Using the limited development, valid for $u \rightarrow 0$
 $$\sqrt{1+u} = 1 + \frac{1}{2}u - \frac{1}{8}u^2 + o(u^2)$$we have
 $$  \lambda'' = \frac{c^2}{\lambda} + o(\frac{1}{\lambda})$$
Thus, the limit of the achieved throughput, when the offered
load goes to $+\infty$, is $0$. This is what we call
\emph{congestion collapse}.

Figure~\ref{D31-f2n} plots the throughput per source
$\lambda''$ as a function of the offered load per source
$\lambda$. It confirms that after some point, the throughput
decreases with the offered load, going to $0$ as the offered
load goes to $+\infty$.
\begin{figure}[htbp]
        \insfig{D31F2n}{0.6}
        \mycaption{Throughput per source as a function of the offered load per
        source, in Mb/s, for the network of Figure~\ref{D31-f1n}. Numbers are in Mb/s. The
        link rate is $c=20$Mb/s for all links.}
        \protect\label{D31-f2n}
\end{figure}

The previous discussion has illustrated the following fact:

\begin{fact}[Efficiency Criterion]
        In a packet network, sources should limit their sending rate by
        taking into consideration the state of the network.  Ignoring this
        may put the network into congestion collapse. One objective of
        congestion control is to avoid such inefficiencies.
\end{fact}

Congestion collapse occurs when some resources are consumed by
traffic that will be later discarded. This phenomenon did
happen in the Internet in the middle of the eighties. At that
time, there was no end-to-end congestion control in TCP/IP. As
we will see in the next section, a secondary objective is
fairness.

\subsection{Efficiency versus Fairness}

Assume that we want to maximize the network throughput, based
on the considerations of the previous section.  Consider the
network example in Figure~\ref{D31-f3}; one source sends at rate $x_0$Mb/s, one at rate $x_1$Mb/s and nine sources at rate $x_2$Mb/s each.  We assume that we implement some form
of congestion control and that there are negligible losses.
Thus, the flow, in Mb/s, on the first link $i$ is $x_{0}+x_{1}$, and on the second link it is $x_{0}+9x_{2}$.   For a
given value of $x_{0}$, maximizing the throughput
requires that $x_{1}= 10 - x_{0}$  and $9x_{2}= 10 - x_{0}$.
The total throughput, measured at the network output, is then
$x_0+x_1+9x_2=20-x_0$; it is maximum for $x_{0} = 0$~!

\begin{figure}[htbp]
        \insfig{D31F3a}{0.7}
        \mycaption{A simple network used to illustrate fairness and efficiency. }
        \protect\label{D31-f3}
\end{figure}

The example shows that maximizing network throughput as a
primary objective may lead to gross unfairness; in the worst
case, some sources may get a zero throughput, which is
probably considered unfair by these sources.

In general, the concept of efficiency is captured by the notion of \emph{Pareto Efficiency}. Consider an allocation problem; define the vector $\vec{x}$
whose $i$th coordinate is the allocation for user $i$. %Let $\calX$ be the set of all feasible allocations.
\begin{definition}[Pareto Efficiency]
       A feasible allocation of rates
$\vec{x}$ is ``Pareto-Efficient'' (also called ``Pareto-Optimal'') if and only if an increase of
any rate within the domain of feasible allocations must be at
the cost of a decrease of some other rate. Formally,
for any other feasible allocation $\vec{y}$, if $y_{s} >
x_{s}$ then there must exist some $s'$ such that $y_{s'} < x_{s'}$. \label{def-paretoeff}
\end{definition}

In general, there exist many Pareto-efficient allocations. For the example in \fref{D31-f3}, any allocation that saturates every link (i.e., such that $x_0+x_{1}=$  and $x_0+9x_{2}= 10$) is Pareto-efficient. The allocation that maximizes total throughput (and has $x_0=0$) is one of them; another Pareto efficient allocation is $x_0=0.1, x_1=9.9, x_2=1.1$; yet another one is $x_0=1, x_1=9, x_2=1$. Among all of these allocations, which one should a fair and efficient congestion control scheme choose~? This requires a formal definition of fairness, given in the next section.


%In summary, the objective of congestion control is to provide both
%efficiency and some form of fairness. We will study fairness in
%more detail now.

\section{Fairness}

\subsection{Max-Min Fairness}


In an egalitarian vision, fairness would simply mean allocating the same
share to all, while maximizing this share. In the simple case of Figure~\ref{D31-f3}, this would
mean allocating $x_0=x_1=x_2=1$. However, such an egalitarian allocation is not Pareto-efficient (because it is possible to unilaterally increase $x_1$).

A better allocation could be based on the following observation. Starting from the egalitarian allocation, we can increase the rate $x_1$ without impacting other rates, up to the value $x_1=9$. The resulting allocation $x_0=1, x_1=9, x_2=1$ \emph{is} Pareto-efficient, and still appears to be as fair as possible. It turns out that this allocation is exactly the Max-Min fair allocation, which is defined next.

%Consider again the example of Figure~\ref{D31-f3}, now with
%general values for $n_{i}$. If we follow the previous line of
%reasoning, we would allocate the fraction
%$\frac{c}{n_{0}+n_{i}}$ to each of the $n_{0}+n_{i}$ sources
%using link $i$.  This yields $x_{i}=\frac{c}{n_{0}+n_{i}}$ for
%$i \geq 1$; for $i=0$, the reasoning of the previous section
%indicates that we should allocate $x_{0}= \min_{1 \leq i \leq
%I} \frac{c}{n_{0}+n_{i}}$. For example, with $I=2$,
%$n_{0}=n_{1}=1$ and $n_{2}=9$, we would allocate $x_{0}=0.1
%c$, $x_{1}=0.5 c$ and $x_{2}=0.1 c$. This allocation however
%would not fully utilize link $1$; we could decide to increase
%the share of sources of type 1 since this can be done without
%decreasing the shares of other sources. Thus, a final
%allocation could be $x_{0}=0.1 c$, $x_{1}=0.9 c$ and
%$x_{2}=0.1 c$. We have illustrated that allocating resources
%in an equal proportion is not a good solution since some
%sources can get more that others without decreasing others'
%shares. Formally, this leads to our first definition of
%fairness called max-min fairness.

%Consider an allocation problem; define the vector $\vec{x}$
%whose $i$th coordinate is the allocation for user $i$. Let
%$\calX$ be the set of all feasible allocations.
%Recall that the $i$th coordinate of vector $\vec{x}$
%is the rate allocated to user $i$ and $\calX$ is the set of all feasible allocations.

\begin{definition}[Max-min Fairness \cite{BG}] A feasible allocation of rates
$\vec{x}$ is ``max-min fair'' if and only if an increase of
any rate within the domain of feasible allocations must be at
the cost of a decrease of some already smaller rate. Formally,
for any other feasible allocation $\vec{y}$, if $y_{s} >
x_{s}$ then there must exist some $s'$ such that $x_{s'} \leq
x_{s}$ and $y_{s'} < x_{s'}$. \label{def-mmf}
\end{definition}
The name
``max-min'' comes from the idea that it is forbidden to
decrease the share of sources that have small values, thus, in
some sense, we give priority to flows with small values.

\begin{theorem}A max-min fair allocation is Pareto-efficient.
\end{theorem}
\pr If we increase one rate in a max-min fair allocation, we must decrease some other rate, which expresses that the allocation is Pareto-efficient.
\qed

The converse is not true -- a Pareto-efficient allocation is not, in general, max-min fair. In fact, as we will see next, the max-min fair allocation is unique, whereas there are usually many Pareto-efficient allocations -- see the in \fref{D31-f3}.

\begin{theorem}A max-min fair allocation, if it exists, is unique.
\end{theorem}
\pr Assume now that
$\vec{x}$ and $\vec{y}$ are two max-min fair allocations for
the same problem, with $\vec{x} \neq \vec{y}$. Without loss of
generality, we can assume that there exists some $i$ such that
$x_{i} < y_{i}$. Consider the smallest value of $x_{i}$ that
satisfies  $x_{i} < y_{i}$, and call $i_{0}$ the corresponding
index. Thus, $x_{i_{0}} < y_{i_{0}}$ and
\begin{equation}
        \mif  x_{i} < y_{i} \mthen x_{i_{0}} \leq x_{i}
        \label{eq-preuve-uni-mmf}
\end{equation}
Now since $\vec{x}$ is max-min fair, from
Definition~\ref{def-mmf}, there exists some $j$ with
\begin{equation}
        y_{j}<x_{j} \leq x_{i_{0}}
        \label{eq-preuve-uni-mmf-2}
\end{equation}
Now $\vec{y}$ is also max-min fair, thus by the same token
there exists some $k$ such that
\begin{equation}
        x_{k}<y_{k} \leq y_{j}
        \label{eq-preuve-uni-mmf-3}
\end{equation}
Combining (\ref{eq-preuve-uni-mmf-2}) and
(\ref{eq-preuve-uni-mmf-3}), we obtain
$$
x_{k}<y_{k} \leq y_{j}<x_{j} \leq x_{i_{0}}
$$
which contradicts (\ref{eq-preuve-uni-mmf}). \qed
\qed

For some allocation problems, a max-min fair allocation might
not exist. However, for the practical examples of interest in the context of congestion control, there is always one (and only one) max-min fair allocation, as we show next. For more general cases,
see \cite{radunovic2007ufm}, where it is shown in particular that a max-min fair allocation always exists when the feasible set is convex and compact.

\paragraph{Network Model}

We use the following simplified network model in the rest of
this section. We consider a set of sources $s=1, \ldots, S$
and links $1,\ldots,L$. Let $A_{l,s}$ be the fraction of
traffic of source $s$ which flows on link $l$, and let $c_{l}$
be the capacity of link $l$. We define a network as the couple
$(\vec{x}, A)$.

A \emph{feasible allocation} of rates $x_{s}\geq 0$ is defined
by: $\sum_{s=1}^{S}A_{l,s}x_{s} \leq c_{l}$ for all $l$.

Our network model supports both multicast and load sharing.
For a given source $s$, the set of links $l$ such that
$A_{l,s} > 0$ is the path followed by the data flow with
source $s$. In the simplest case (no load sharing), $A_{l,s}
\in \{0, 1\}$; if a flow from source $s$ is equally split
between two links $l_{1}$ and $l_{2}$, then $A_{l_{1},s} =
A_{l_{2},s} =0.5$. In principle, $A_{l,s} \leq 1$, but this is
not mandatory (in some encapsulation scenarios, a flow may be
duplicated on the same link).



%It can be seen (and this is left as an exercise) that the
%allocation in the previous example is max-min fair.

We now show that there exists a
max-min fair allocation to our network model, and how to
obtain it. This will result from the key concept of
``bottleneck link''.

\begin{definition}[Bottleneck Link]
        With our network model above, we say that link $l$ is a bottleneck
        for source $s$ if and only if
        \begin{enumerate}
                \item  link $l$ is saturated: $c_{l}= \sum_{i}A_{l,i}x_{i}$

                \item  source $s$ on link $l$ has the maximum rate among all
                sources using link $l$:
                $x_{s}\geq x_{s'}$ for all $s'$ such that $A_{l,s'} > 0$.
        \end{enumerate}
\end{definition}

Intuitively, a bottleneck link for source $s$ is a link which
is limiting, for a given allocation.  In the previous
example, the first link is a bottleneck for blue flow with rate $x_1$ and for the red flow with rate $x_0$; the second link is a bottleneck for each of the nine green flows with rates $x_2$ and for the red flow with rate $x_0$.

\begin{theorem}
        A feasible allocation of rates $\vec{x}$ is max-min fair if and
        only if every source has a bottleneck link.
        \label{theo-bn}
\end{theorem}
\pr \textbf{Part 1.}  Assume that every source has a bottleneck
link. Consider a source $s$ for which we can increase the rate
$x_{s}$ while keeping the allocation feasible. Let $l$ be a
bottleneck link for $s$. Since $l$ is saturated, it is necessary
to decrease $x_{s'}$ for some $s'$ such that $A_{l,s'} >0$. We
assumed that we can increase the rate of $s$: thus there must
exist some $s' \neq s$ that shares the bottleneck link $l$. But
for all such $s'$, we have $x_{s}\geq x_{s'}$, thus we are forced
to decrease $x_{s'}$ for some $s'$ such that $x_{s}\geq x_{s'}$:
this shows that the allocation is max-min fair.

\textbf{Part 2.} Conversely, assume that the allocation is max-min
fair.  For any source $s$, we need to find a bottleneck link.  We
proceed by contradiction.  Assume there exists a source $s$ with
no bottleneck link.  Call $L_{1}$ the set of saturated links used
by source $s$, namely, $L_{1} = \{l \mst c_{l}=
\sum_{i}A_{l,i}x_{i} \mand A_{l,s} > 0 \}$.  Similarly, call
$L_{2}$ the set of non-saturated links used by source $s$. Thus a
link is either in $L_{1}$ or $L_{2}$, or is not used by $s$.
Assume first that $L_1$ is non-empty.

\begin{figure}[h]
        \insfig{D31F6}{0.5}
        \mycaption{A network example showing one multicast source}
        \protect\label{fig-sigma}
\end{figure}

By our assumption, for all $l \in L_{1}$ , there exists some
$s'$ such that $A_{l,s'} >0$ and $x_{s'} > x_{s}$. Thus we can
build a mapping $\sigma$ from $L_{1}$ into the set of sources
$\{1,\ldots,S \}$ such that $A_{l,\sigma(l)} >0$ and
$x_{\sigma(l)} > x_{s}$ (see Figure~\ref{fig-sigma} for an
illustration). Now we will show that we can increase the rate
$x_{s}$ in a way that contradicts the max-min fairness
assumption. We want to increase $x_{s}$ by some value
$\delta$, at the expense of decreasing $x_{s'}$ by some other
values $\delta_{s'}$, for all $s'$ that are equal to some
$\sigma(l')$. We want the modified allocation to be feasible;
to that end, it is sufficient to have:

\begin{eqnarray}
         A_{l,s} \delta \leq A_{l, \sigma(l)} \delta_{\sigma(l)} \; \mfa l \in
        L_{1}
        \label{eq-mmf-1}  \\
        A_{l,s} \delta  \leq c_{l} - \sum_{i}A_{l,i}x_{i} \; \mfa l \in L_{2}
        \label{eq-mmf-2}  \\
        \delta_{\sigma(l)} \leq x_{\sigma(l)} \; \mfa l \in L_{1}
        \label{eq-mmf-3}
\end{eqnarray}
Equation~(\ref{eq-mmf-1}) expresses that the increase of flow
due to source $s$ on a saturated link $l$ is at least
compensated by the decrease of flow due to source $\sigma(l)$.
Equation~(\ref{eq-mmf-2}) expresses that the increase of flow
due to source $s$ on a non-saturated link $l$ does not exceed
the available capacity. Finally, equation~(\ref{eq-mmf-3})
states that rates must be non-negative.

This leads to the following choice.
\begin{equation}
        \delta = \min_{l \in L_{1}}
        \{  \frac{x_{\sigma(l)} A_{l, \sigma(l)}}{A_{l,s}}\}
        \wedge
        \min_{l \in L_{2}}
        \{  \frac{c_{l}-\sum_{i}A_{l,i}x_{i}}{A_{l,s}}\}
        \label{eq-mmf-4}
\end{equation}
which ensures that Equation~(\ref{eq-mmf-2}) is satisfied and
that $\delta >0$.

In order to satisfy Equations~(\ref{eq-mmf-1}) and
(\ref{eq-mmf-3}) we need to compute the values of
$\delta_{\sigma(l)}$ for all $l$ in $L_{1}$. Here we need to
be careful with the fact that the same source $s'$ may be
equal to $\sigma(l)$ for more than one $l$. We define
$\delta(s')$ by

\begin{eqnarray}
         \delta(s') & = 0  \textrm{ if there is no } l \mst s'=\sigma(l)
        \label{eq-mmf-50}  \\
        \delta(s') & =
                \max_{\{l \mst \sigma(l) = s'\}}
                \{ \frac{\delta A_{l,s}}{A_{l, \sigma(l)}} \}
                \textrm{ otherwise }
        \label{eq-mmf-5}
\end{eqnarray}

This definition ensures that Equation~(\ref{eq-mmf-1}) is
satisfied. We now examine Equation~(\ref{eq-mmf-3}). Consider
some $s'$ for which there exists an $l$ with $\sigma(l)=s$,
and call $l_{0}$ the value which achieves the maximum in
(\ref{eq-mmf-5}), namely:
\begin{equation}
        \delta(s') = \frac{\delta A_{l_{0},s}}{A_{l_{0}, s'}}
        \label{eq-mmf-6}
\end{equation}
From the definition of $\delta$ in (\ref{eq-mmf-4}), we have
$$
\delta \leq \frac{x_{\sigma(l_{0})}A_{l_{0}, \sigma(l_{0})} }{
A_{l_{0},s}}
=
\frac{x_{s'}A_{l_{0}, s'} }{ A_{l_{0},s}}
$$
Combined with (\ref{eq-mmf-6}), this shows that
Equation~(\ref{eq-mmf-3}) holds. In summary, we have shown
that we can increase $x_{s}$ at the expense of decreasing the
rates for only those sources $s'$ such that $s'=\sigma(l)$ for
some $l$. Such sources have a rate higher than $x_{s}$, which
shows that the allocation $\vec{x}$ is not max-min fair and
contradicts our hypothesis.

It remains to examine the case where $L_1$ is empty. The
reasoning is the same, we can increase $x_s$ without
decreasing any other source, and we also have a contradiction.
\qed

\paragraph{The Water-Filling Algorithm}

The previous theorem is particularly useful in deriving a
practical method for obtaining a max-min fair allocation,
called ``progressive filling'' or ``water filling''.  The idea is as follows.  You
start with all rates equal to 0 and grow all rates together at
the same pace, until one or several link capacity limits are
hit.  The rates for the sources that use these links are not
increased any more, and you continue increasing the rates for
other sources.  All the sources that are stopped have a
bottleneck link.  This is because they use a saturated link,
and all other sources using the saturated link are stopped at
the same time, or were stopped before, thus have a smaller or
equal rate.  The algorithm continues until it is not possible
to increase.  The algorithm terminates because $L$ and $S$ are
finite.  Lastly, when the algorithm terminates, all sources
have been stopped at some time and thus have a bottleneck
link.  By application of Theorem~\ref{theo-bn}, the allocation
is max-min fair.

\paragraph{Example}
Let us apply the water-filling algorithm to \fref{D31-f3}.

We first let $x_{0}=x_1=x_2=t$ and increase $t$ until we hit a limit, i.e. we maximize $t$ subject to the constraints imposed by the link capacities. The constraints are
$$
x_0+x_1\leq 10\mand x_1 +9x_2\leq 10$$
which gives
$$
2t\leq 10 \mand 10t\leq 10$$
The maximum is for $t=1$Mb/s and at this value the second constraint is hit, i.e., the second link is a bottleneck link. The sources using this link are the sources of type 0 and 2, therefore we let
$$
x_0=1\mand x_2=1
$$
and these values are final.

In a second round we increase the rates of the other sources, namely we increase $x_1$ until a capacity limit is hit. There is only one constraint left, that of the first link, which is now expressed as
$$1 + x_1\leq 10
$$
The maximal value of $x_1$ is $9$. With this, all sources are stopped and the algorithm terminates. The max-min fair allocation is thus $x=1, x_1=9, x_2=1$.

We see that all sources of type $0$ and $2$ obtain
the same rate. In some sense, max-min fairness ignores the
fact that sources of type $0$ use more network resources than
those of type $2$.

\begin{theorem}For the network defined above,
 with fixed routing parameters $A_{l,s}$, there exists a unique
        max-min fair allocation.  It can be obtained by the water-filling algorithm.
\end{theorem}
\pr We have already proven uniqueness.  Existence follows from the water-filling algorithm. \qed


% Pb avec fairness (exemple de Jim Roberts) avec shortest packet first

The notion of max-min fairness can be generalized by
using weights in the definition \cite{BG, MR98}. 

\subsection{Proportional Fairness}

The previous definition of fairness puts emphasis on
maintaining high values for the smallest rates. As shown in
the previous example, this may be at the expense of some
network inefficiency. An alternative definition of fairness
has been proposed in the context of game theory \cite{MMD91}.

\begin{definition}[Proportional Fairness]
An allocation of rates $\vec{x}$ is ``proportionally fair'' if and only if all rates are positive and, for any other feasible allocation $\vec{y}$, we
have:
$$
\sum_{s=1}^S \frac{y_{s}-x_{s}}{x_{s}} \; \leq 0
$$
        \label{def-pf}
\end{definition}

In other words, any change in the allocation must have a
negative average change.



Let us consider for example the
parking lot scenario in \fref{D31-f3p}. Is the
max-min fair allocation proportionally fair~?


\begin{figure}[htbp]
        \insfig{D31F3p}{0.7}
        \mycaption{A simple network used to illustrate proportional fairness (the``parking
        lot'' scenario)}
        \protect\label{D31-f3p}
\end{figure}



To get the answer, observe that the max-min fair
allocation is obtained easily with water-filling and is $x_{s}=c/2$ for $s=0,1..4$. Consider a new allocation
resulting from a decrease of $x_{0}$ equal to $\delta$:
\begin{eqnarray*}
         y_{0}  &= \frac{c}{2} - \delta \\
        y_{s} &= \frac{c}{2} + \delta \; s=1,\ldots,4
\end{eqnarray*}
For $\delta<\frac{c}{2}$, the new allocation $\vec{y}$ is
feasible. The average rate of change is
$$
\left( \sum_{s=1}^4 \frac{2 \delta }{ c} \right)
 -  \frac{2 \delta }{ c} =
\frac{6\delta }{c}
$$
which is positive.  Thus the max-min fair
allocation for this example is not proportionally fair.  In this example, we see that a decrease in rate for
sources of type $0$ is less important than the corresponding
increase which is made possible for the other sources, because
the increase is multiplied by the number of sources.
Informally, we say that proportional fairness takes into
consideration the usage of network resources.

\begin{theorem}A proportionally fair allocation is Pareto-efficient.
\end{theorem}
\pr Let $\vec{x}$ be a proportionally fair allocation and let $\vy$ be some feasible allocation such that $y_i>x_i$ for some $i$. The rate of change is
$$
\frac{y_i-x_i}{x_i}+\sum_{s=1...S, s\neq i}\frac{y_s-x_s}{x_s}\leq 0
$$ because $\vec{x}$ is proportionally fair. The first term is positive, therefore at least one of the other terms, say $\frac{y_s-x_s}{x_s}$ must be negative, and thus $y_s<x_s$. This shows that $\vec{x}$ is Pareto-efficient.
\qed


Now we derive a practical result which can be used to compute
a proportionally fair allocation.  To that end, we interpret
the average rate of change as $\nabla J_{\vec{x}}\cdot
(\vec{y}-\vec{x})$, with
$$
J(\vec{x})= \sum_{s} \ln(x_{s})
$$
Thus, intuitively, a proportionally fair allocation should
maximize $J$.
\begin{theorem}
There exists one unique proportionally fair allocation. It is
obtained by maximizing $J(\vec{x})= \sum_{s} \ln(x_{s})$ over
the set of feasible allocations.
        \label{theo-pf}
\end{theorem}
\pr 1. We first prove that the maximization problem has a unique
solution.  Function $J$ is concave, as a sum of concave
functions. The feasible set is convex, as intersection of
convex sets, thus any local maximum of $J$ is an absolute
maximum.  Now $J$ is strictly concave, which means that
$$\mif 0< \alpha< 1 \mthen J(\alpha \vec{x}
+ (1-\alpha) \vec{y}) > \alpha J(\vec{x}) + (1-\alpha)
J(\vec{y})
$$
This can be proven by studying the second derivative
of the restriction of $J$ to any linear segment.  Now
a strictly concave function has at most one maximum on
a convex set. %(Chapter~\ref{D4-fluid11}).

Now $J$ is continuous if we allow $\log(0)=-\infty$ and the
set of feasible allocations is compact (because it is a
closed, bounded subset of $\Reals^S$). Thus $J$ has at least
one maximum over the set of feasible allocations.
Combining all the arguments together proves that $J$ has
exactly one maximum over the set of feasible allocations, and
that any local maximum is also exactly the global maximum.

2. For any $\vec{\delta}$ such that $\vec{x}+ \vec{\delta}$
is feasible,

$$
J(\vec{x}+ \vec{\delta})-J(\vec{x})=\nabla J_{\vec{x}}\cdot
\vec{\delta} + \frac{1}{2}   \vec{\delta}^T \nabla^2
J_{\vec{x}} \vec{\delta} +  o(||\vec{\delta}||^2)
$$
Now by the strict concavity, $\nabla^2 J_{\vec{x}}$ is
definite negative thus, for
$||\vec{\delta}||$ small enough:
$$\frac{1}{2}  \vec{\delta}^T \nabla^2
J_{\vec{x}} \vec{\delta} +  o(||\vec{\delta}||^2) \; <0 $$
and therefore
$$
J(\vec{x}+ \vec{\delta})-J(\vec{x})\leq \nabla J_{\vec{x}}\cdot
\vec{\delta}
$$


Now assume that $\vec{x}$ is a proportionally fair allocation.
This means that

$$\nabla(J)_{\vec{x}}\cdot \vec{\delta} \leq 0$$
and thus $J$ has a local maximum at $\vec{x}$, thus also a
global maximum. This also shows the uniqueness of a
proportionally fair allocation.


3. Conversely, assume that $J$ has a global maximum at $\vec{x}$.
Observe that our network model assumes that there exists some positive and feasible allocation, therefore, the maximum of $J$ is not $-\infty$, therefore the maximiser $\vec{x}$ satisfies $x_s>0$ for all $s$. Let $\vec{y}$ be some other feasible allocation and call $D$ the
average rate of change. We also have:
$$ D = \nabla(J)_{\vec{x}}\cdot (\vec{y}-\vec{x}) $$
Since the feasible set is convex, the segment $[\vec{x},
\vec{y}]$ is entirely feasible, and thus $J(\vec{x} + t (\vec{y}-\vec{x})) \leq  J(\vec{x} )$ for $t\in [0;1]$. Observe that
$$D = \lim_{t \rightarrow 0^+}
\frac{J(\vec{x} + t (\vec{y}-\vec{x})) - J(\vec{x} ) }{ t}
$$
thus $D \leq 0$. \qed
\paragraph{Example}
Let us apply Theorem~\ref{theo-pf} to the parking lot
scenario in \fref{D31-f3p}. For any choice of $x_{0}$, we should set $x_{i}$
such that
$$
x_{0} + x_{i} = c,  i=1,\ldots,4
$$
otherwise we could increase $x_{i}$ without affecting other
values, and thus increase function $J$. The value of $x_{0}$
is found by maximizing $f(x_{0})$, defined by
$$
f(x_{0}) =  \ln(x_{0}) + \sum_{i=1}^4  \ln(c
-x_{0})
$$
over the set $0 \leq x_{0} \leq c$. The
derivative of $f$ is
$$
f'(x_{0}) =\frac{1}{x_{0}} - \frac{4}{ c- x_{0}}
$$
After some algebra, we find that the maximum is for
$$
x_{0}= \frac{c}{5}
$$
and
$$
x_{i}= \frac{4c}{5}\mfor i=1...4
$$
%For example, if $n_{i}=1$ for all $i=0,\ldots,4$, we obtain:
%\begin{eqnarray*}
%         x_{0}&= \frac{c }{4 +1} \\
%        x_{i} &= \frac{cI}{4+1}
%\end{eqnarray*}
Compare with max-min fairness, where, in that case, the
allocation is $\frac{c}{2}$ for all rates.  We see that
sources of type $0$ get a smaller rate, since they use more
network resources.

The concept of proportional fairness can easily extended to
\emph{weighted} proportional fairness, where the allocation
maximizes a weighted sum of logarithms \cite{KMT97}.


\subsection{Utility Approach to Fairness}
Proportional fairness is an example of a more general fairness
concept, called the ``utility'' approach, which is defined as
follows. Every source $s$ has a utility function $u_{s}$ where
$u_{s}(x_{s})$ indicates the value to source $s$ of having
rate $x_{s}$. Every link $l$ (or network resource in general)
has a cost function $g_{l}$,  where $g_{l}(f)$ indicates the
cost to the network of supporting an amount of flow $f$ on
link $l$. Then, a ``utility fair'' allocation of rates is an
allocation which maximizes $H(\vec{x})$, defined by
$$H(\vec{x}) = \sum_{s=1}^S u_{s}(x_{s}) - \sum_{l=1}^L g_{l}(f_{l})
$$
with $f_{l}=\sum_{s=1}^S A_{l,s}x_{s}$, over the set of
feasible allocations.

Proportional fairness corresponds to $u_{s}= \ln$ for all $s$,
and $g_{l}(f)=0$ for $f< c_{l}$, $g_{l}(f)=+ \infty$ for
$f\geq c_{l}$. Rate proportional fairness corresponds to
$u_{s}(x_{s})= w_{s}\ln(x_{s})$ and the same choice of
$g_{l}$.

Computing utility fairness requires solving constrained
optimization problems; a reference is \cite{WHI}.

\subsection{Max Min Fairness as a limiting case of Utility Fairness}
We show in this section that max-min fairness is the limiting case
of a utility fairness. Indeed, we can define a set of concave,
increasing utility functions $f_m$, indexed by $m \in \Reals^+$
such that the allocation $\vec{x^m}$ which maximizes
$\sum_{i=1}^{I} f_m(x_i)$ over the set of feasible allocations
converges towards the max-min fair allocation.

The proof is a correct version of the ideas expressed in
\cite{mo-98-a} and later versions of it.

Let $f_m$ be a family of increasing, differentiable and concave
functions defined on $\Reals^+$. Assume that, for any fixed
numbers $x$ and $\delta
>0$,
\begin{equation}\mylabel{eq-deffm}
  \lim_{m \rightarrow + \infty} \frac{f'_m(x+\delta)}{f'_m(x)}=0
\end{equation}
% modified Nov 2002, based on input by BOZ
%\begin{equation}\mylabel{eq-deffm}
%  \lim_{m \rightarrow + \infty} \frac{f_m(x)}{f_m(x+\delta)}=0
%\end{equation}

The assumption on $f_m$ is satisfied if $f_m$ is defined by
 $$f_m(x) = c - g(x)^m
 $$
where $c$ is a constant and $g$ is a differentiable, decreasing,
convex and positive function. For example, consider
 $f_m(x) = 1 - \frac{1}{x^m}$.

Define $\vec{x}^m$ the unique allocation which maximizes
$\sum_{i=1}^{I} f_m(x_i)$ over the set of feasible allocations.
Thus  $\vec{x}^m$ is the fair allocation of rates, in the sense of
utility fairness defined by $f_m$. The existence of this unique
allocation follows from the same reasoning as for
Theorem~\ref{theo-pf}. Our result if the following theorem.

\begin{theorem}
\mylabel{theo-approx} The set of utility-fair allocation
$\vec{x}^m$ converges towards the max-min fair allocation as $m$
tends to $+\infty$.
\end{theorem}
The rest of this section is devoted to the proof of the theorem.
We start with a definition and a lemma.

\begin{definition}
We say that a vector $\vec{z}$ is an accumulation point for a set
of vectors $\vec{x}_m$ indexed by $m\in \Reals^+$ if there exists
a sequence $m_n, n\in \Nats$, with $\lim_{n \rightarrow + \infty}
m_n = + \infty$ and $\lim_{n \rightarrow + \infty} \vec{x}^{m_n} =
\vec{z}$.
\end{definition}

\begin{lemma}
If $\vec{x}^*$ is an accumulation point for the set of vectors
$\vec{x}^{m}$, then $\vec{x}^*$ is the max-min fair allocation.
\end{lemma}
\paragraph{Proof of Lemma: }
We give the proof for the case where $A_{l,i} \in \{0, 1\}$; in
the general case the proof is similar We proceed by contradiction.
Assume that $\vec{x^*}$ is not max-min fair. Then, from
Theorem~\ref{theo-bn}, there is some source $i$ which has no
bottleneck. Call $L_1$ the set (possibly empty) of saturated links
used by $i$, and $L_2$ the set of other links used by $i$. For
every link $l \in L_1$, we can find some source $\sigma(l)$ which
uses $l$ and such that $x^*_{\sigma(l)} > x^*_i$. Define $\delta$
by
 $$ \delta = \frac{1}{5} \min \left\{
 \min_{l \in L1} ( x^*_{\sigma(l)} - x^*_i), \;
 \min_{l \in L2} ( c_l -(A\vec{x}^*)_{l})
 \right\}
 $$
The term $c_l - (A\vec{x}^*)_{l}$ in the last part of the
expression is the unused capacity on link $l$. We also use the
convention that the minimum of an empty set is $+ \infty$.

%Now note that, for $n$ large enough, (say $n \geq n_0$), the
%allocation $\vec{x}^{m_n}$ is such that the set of saturated links
%used by source $i$ is included in $L_1$. Indeed, if this would not
%be true, there would be some link $l$ not in $L_1$ and an infinite
%number of values of $n$ for which link $l$ is saturated for
%allocation $\vec{x}^{m_n}$. Thus, by continuity of the mapping
%$\vec{x}^m \rightarrow (A\vec{x}^*)_l$, link $l$ is also saturated
%for allocation $\vec{x}^*$, which is a contradiction.

From the convergence of $\vec{x}^{m_n}$ to $\vec{x}^*$, we can
find some $n_0$ such that, for all $n \geq n_0$ and for all $j$ we
have
\begin{equation}\mylabel{eq-defdesx}
   x^*_j - \frac{\delta}{I} \leq x^{m_n}_j \leq  x^*_j + \frac{\delta}{I}
\end{equation}
where $I$ is the number of sources. Now we construct, for all $n
\geq n_0$, an allocation $\vec{y}^n$ which will lead to a
contradiction. Define
 $$
  \left\{
  \begin{array}{rcl}
   y^n_i &=& x^{m_n}_i + \delta \\
   y^n_{\sigma(l)} &=&x^{m_n}_{\sigma(l)} - \delta \mbox{ for } l \in L_1\\
   y^n_j&=&x^{m_n}_j  \; \; \mbox{ otherwise}
  \end{array}
  \right.
 $$

We prove first that the allocation $\vec{y}^n$ is feasible.
Firstly, we show that the rates are non-negative. From
Equation~(\ref{eq-defdesx}), we have, for all $l \in L_1$
 $$
 x^{m_n}_{\sigma(l)} \geq x^*_{\sigma(l)} - \frac{\delta}{I}
 \geq x^*_{\sigma(l)} - \delta
 $$
 thus, from the definition of $\delta$:
\begin{equation}\mylabel{eq-3del}
 y^n_{\sigma(l)} \geq x^*_{\sigma(l)} - 2 \delta \geq  x^*_i + 3 \delta
\end{equation}
This shows that $y^n_j \geq 0$ for all $j$.

Secondly, we show that the total flow on every link is bounded by
the capacity of the link. We need to consider only the case of
links $l \in (L_1 \cup L_2)$. If $l \in L_1$ then
$$(A\vec{y}^n)_{l} \leq (A\vec{x}^{m_n})_{l}+ \delta - \delta = (A\vec{x}^{m_n})_{l}$$
thus the condition is satisfied. Assume now that $l \in L_2$. We
have then
 $$
(A\vec{y}^n)_{l} = (A\vec{x}^{m_n})_{l} + \delta $$
 Now, from Equation~(\ref{eq-defdesx})
 $$
 (A\vec{x}^{m_n})_{l} \leq (A\vec{x}^{*})_{l} + I \frac{\delta}{I}
 =(A\vec{x}^{*})_{l} + \delta
 $$
 Thus, from the definition of $\delta$:
 $$
 (A\vec{y}^n)_{l} \leq (A\vec{x}^{*})_{l} + 2 \delta \leq c
 $$
which ends the proof that $\vec{y}^n$ is a feasible allocation.

Now we show that, for $n$ large enough, we have a contradiction
with the optimality of $\vec{x}^{m_n}$. Consider the expression
$A$ defined by
$$A =
 \sum_j \left(f_{m_n}(y^{n}_j) - f_{m_n}(x^{m_n}_j)\right)
 $$
 From the optimality of $\vec{x}^{m_n}$,  we have: $A \leq 0$.

Now
$$A = f_{m_n}(x^{m_n}_i + \delta) - f_{m_n}(x^{m_n}_i ) +
\sum_{l \in L_1} \left(f_{m_n}(x^{m_n}_{\sigma(l)}-\delta) -
f_{m_n}(x^{m_n}_{\sigma(l)})\right)
$$

From the theorem of intermediate values, there exist numbers
$c^{n}_i$ such that
$$
\bracket{
 x^{m_n}_i \leq c^{n}_i \leq x^{m_n}_i + \delta \\
 f_{m_n}(x^{m_n}_i + \delta) - f_{m_n}(x^{m_n}_i ) =
 f'_{m_n}(c^{n}_i)\delta
 }
$$
where $f'_{m_n}$ is the right-derivative of $f_{m_n}$. Combining
with Equation~(\ref{eq-defdesx}) we find
\begin{equation}\mylabel{eq-ci}
c^{n}_i \leq x^*_i +  \frac{\delta}{I} + \delta \leq x^*_i + 2
\delta
\end{equation}

Similarly, there exist some numbers $c^{n}_{\sigma(l)}$ such that
$$
\bracket{
 x^{m_n}_{\sigma(l)}  -  \delta \leq c^{n}_{\sigma(l)}+ \delta \leq x^{m_n}_{\sigma(l)}+
 \delta \\
 f_{m_n}(x^{m_n}_{\sigma(l)} - \delta) - f_{m_n}(x^{m_n}_{\sigma(l)} ) =
 - f'_{m_n}(c^{n}_{\sigma(l)})\delta
 }
$$
and combining with Equation~(\ref{eq-defdesx}) we find also
\begin{equation}\mylabel{eq-cj}
c^{n}_{\sigma(l)} \geq x^*_i +  3 \delta
\end{equation}
Thus
$$
A = \delta \left(
 f'_{m_n}(c^{n}_i)
 -
 \sum_{l \in L_1} f'_{m_n}(c^{n}_{\sigma(l)})
  \right)
$$
Now $f'_{m_n}$ is wide-sense decreasing ($f_{m_n}$ is concave)
thus, combining with Equations~(\ref{eq-ci}) and~(\ref{eq-cj}):
$$A
\geq \delta \left(f'_{m_n}(x^*_i + 2 \delta) - M  f'_{m_n}(x^*_i + 3 \delta)\right) = \delta
f'_{m_n}(x^*_i + 2 \delta) \left( 1- M \frac{f'_{m_n}(x^*_i + 3 \delta)}{f'_{m_n}(x^*_i + 2
\delta)} \right)
$$
where $M$ is the cardinal of set $L_1$. Now from
Equation~(\ref{eq-deffm}), the last term in the above equation
tends to $1$ as $n$ tends to infinity. Now $f'_{m_n} >0$ from our
assumptions thus, for $n$ large enough, we have $A >0$, which is
the required contradiction.

\paragraph{Proof of Theorem: }

The set of vectors $\vec{x}^m$ is in a compact (= closed +
bounded) subset of $\Reals^I$; thus, it has at least one
accumulation point. From the uniqueness of the max-min fair
vector, it follows that the set of vectors $\vec{x}^m$ has a
unique accumulation point, which is equivalent to saying that
$$
\lim_{m \rightarrow + \infty} \vec{x}^m = \vec{x}^*
$$
\qed

\section{Different forms of congestion control}

We can consider that there are three families of solutions for
congestion control.
\begin{description}
        \item[Rate Based: ]  Sources know an explicit rate at which they can
        send. The rate may be given to the source during a negotiation
        phase;
        this is the case with ATM or RSVP. In such cases, we have a network
        with reservation. Alternatively, the rate may be imposed
        dynamically to the source by the network; this is the case
        for the ABR class of ATM. In such cases we have a best effort
        service (since the source cannot be sure of how long a given rate
        will remain valid), with explicit rate. In the example of the
        previous section, source 1 would obtain a rate not exceeding 10 kb/s.

        \item[Hop by Hop: ] A source needs some feedback from the next hop in
        order to send any amount of data. The next hop also must obtain some
        feedback from the following hop and so on. The feedback may be positive
        (credits) or negative (backpressure). In the simplest form, the
        protocol is stop and go. In the example of the
        previous section, node X would be prevented by node Y from sending
        source 2 traffic at a rate higher than 10kb/s; source 2 would then
        be throttled by node X. Hop by hop control is used
        with full duplex Ethernets using 802.3x frames called ``Pause''
        frames.

        \item[End-to-end:] A source continuously obtains feedback from all
        downstream nodes it uses.  The feedback is piggybacked in packets
        returning towards the source, or it may simply be the detection of
        a missing packet.  Sources react to negative feedback by reducing
        their rate, and to positive feedback by increasing it.  The
        difference with hop-by-hop control is that the intermediate nodes
        take no action on the feedback; all reactions to feedback are left
        to the sources.  In the example of the previous section, node Y
        would mark some negative information in the flow of source 2 which would be
        echoed to the source by destination D2; the source would then react
        by reducing the rate, until it reaches 10 kb/s, after which there
        would be no negative feedback. Alternatively, source 2 could detect
        that a large fraction of packets is lost, reduce its rate, until
        there is little loss. In broad terms, this is the method invented for
        Decnet, which is now used after some modification in the Internet.

\end{description}

        In the following section we focus on end-to-end control.

\section{Max-min fairness with fair queuing}
Consider a network implementing fair queuing per flow. This is
equivalent to generalized processor sharing (GPS) with equal
weights for all flows. With fair queuing, all flows that have data
in the node receive an equal amount of service per time slot.

Assume all sources adjust their sending rates such that there is
no loss in the network. This can be implemented by using a sliding
window, with a window size which is large enough, namely, the
window size should be as large as the smallest rate that can be
allocated by the source, multiplied by the round trip time.
Initially, a source starts sending with a large rate; but in order
to sustain the rate, it has to receive acknowledgements. Thus,
finally, the rate of the source is limited to the smallest rate
allocated by the network nodes. At the node that allocates this
minimum rate, the source has a rate which is as high as the rate
of any other sources using this node. Following this line of
thoughts, the alert reader can convince herself that this node is
a bottleneck link for the source, thus the allocation is max-min
fair. The detailed proof is complex and is given in
\cite{Hahne91}.

\begin{proposition}
A large sliding window at the sources plus fair queuing in the
nodes implements max-min fairness.
\end{proposition}


Fixed window plus fair queuing is a possible solution to
congestion control. It is implemented in some (old) proprietary
networks such as IBM SNA.

Assume now that we relax the assumption that sources are using a
window. Assume thus that sources send at their maximum rate,
without feedback from the network, but that the network implements
fair queuing per flow. Can congestion collapse occur as in
Section~\ref{sec-d31-why}?

Let us start first with a simple scenario with only one network
link of capacity $C$. Assume there are $N$ sources, and the only
constraint on source $i$ is a rate limit $r_i$. Thus, sources for
which $r_i \leq \frac{C}{N}$ have a throughput equal to $r_i$ and
thus experience no loss. If some sources have a rate $r_i <
\frac{C}{N}$, then there is some extra capacity which can will be
used to serve the backlog of other sources. This distribution will
follow the algorithm of progressive filling, thus the capacity
will be distributed according to max-min fairness, \emph{in the
case of a single node}.

In the multiple node case, things are not as nice, as we show now on
the following example. The network is as on Figure~\ref{D31-f1}.
There is 1 source at S1, which sends traffic to D1. There are 10
sources at S2, which all send traffic to D2. Nodes X and Y implement
fair queuing per flow. Capacities are :
\begin{itemize}
  \item 11 Mb/s for link X-Y
  \item 10 Mb/s for link Y-D1
  \item 1 Mb/s for link Y-D1
\end{itemize}

Every source receives 1 Mb/s at node X. The S1- source keeps its
share at Y. Every S2 source experiences 90\% loss rate at Y and
has a final rate of 0.1 Mb/s.

Thus finally, the useful rate for every source is
  \begin{itemize}
  \item 1 Mb/s for a source at S1
 \item 0.1 Mb/s for a source at S2
  \end{itemize}

The max-min fair share is
 \begin{itemize}
  \item 10 Mb/s for a source at  S1
 \item 0.1 Mb/s for a source at S2
\end{itemize}

Thus fair queuing alone is not sufficient to obtain max-min
fairness. However, we can say that if all nodes in a network
implement fair queuing per flow, the throughput for any source $s$
is at least $\min_{l \mst l \in s }\frac{C_l}{N_l}$, where $C_l$ is
the capacity of link $l$, and $N_l$ is the number of active sources
at node $l$. This implies that congestion collapse as described
earlier is avoided.


\section{Additive increase,
Multiplicative decrease and Slow-Start}

End-to-end congestion control in packet networks is based on binary
feedback and the adaptation mechanisms of additive increase,
multiplicative decrease and slow start. We describe here a
motivation for this approach. It comes from the following modeling,
from \cite{CJ89}. Unlike the fixed window mentioned earlier, binary
feedback does not require fair queuing, but can be implemented with
FIFO queues. This is why it is the preferred solution today.

\subsection{Additive Increase, Multiplicative Decrease}
Assume $I$ sources, labeled $i=1,\ldots, I$ send data at
a time dependent rate $x_{i}(t)$, into a network constituted of one
buffered link, of rate $c$.  We assume that time is discrete ($t \in
\Nats$), and that the feedback cycle lasts exactly one time unit.
During one time cycle of duration 1 unit of time, the source rates
are constant, and the network generates a binary feedback signal
$y(t) \in \{0, 1 \}$, sent to all sources. Sources react to the
feedback by increasing the rate if $y(t)=0$, and decreasing if
$y(t)=1$. The exact method by which this is done is called the
adaptation algorithm. We further assume that the feedback is defined
by
$$
y(t) = [ \mif (\sum_{i=1}^I x_{i}(t) \leq c ) \mthen  0 \melse 1 ]
$$
The value $c$ is the target rate which we wish the system not to
exceed. At the same time we wish that the total traffic be as
close to $c$ as possible.

We are looking for a linear adaptation algorithm, namely, there
must exist constants $u_{0}, u_{1}$ and $v_{0}, v_{1}$ such that

\begin{equation}
        x_{i}(t+1) = u_{y(t)} x_{i}(t) + v_{y(t)}
        \label{eq-theo-addmul1}
\end{equation}

We want the adaptation algorithm to converge towards a fair
allocation. In this simple case, there is one single bottleneck
and all fairness criteria are equivalent. At equilibrium, we
should have $x_{i}= \frac{c}{I}$. However, a simple adaptation
algorithm as described above cannot converge, but in contrast,
oscillates around the ideal equilibrium.

We now derive a number of necessary conditions. First, we would
like the rates to increase when the feedback is $0$, and to
decrease otherwise. Call $f(t)=\sum_{i=1}^I x_{i}(t)$. We have

\begin{equation}
        f(t+1) = u_{y(t)} f(t) + v_{y(t)}
        \label{eq-jain-48}
\end{equation}
Now our condition implies that, for all $f \geq 0$:
$$
u_{0}f + v_{0} > f
$$
and
$$
u_{1}f + v_{1} < f
$$
This gives the following necessary conditions
 \begin{equation}
 \left \{
        \begin{array}{cc}
        u_{1} < 1 & \mand v_{1} \leq 0  \\
        \mor \\
        u_{1} = 1 & \mand v_{1} < 0
        \end{array}
 \right.
        \label{eq-jain-1}
 \end{equation}
and
\begin{equation}
 \left \{
        \begin{array}{cc}
        u_{0} > 1 & \mand v_{0} \geq 0  \\
        \mor \\
        u_{0} = 1 & \mand v_{0} > 0
        \end{array}
 \right.
        \label{eq-jain-2}
 \end{equation}

The conditions above imply that the total rate $f(t)$ remains
below $c$ until it exceeds it once, then returns below $c$. Thus
the total rate $f(t)$ oscillates around the desired equilibrium.

Now we also wish to ensure fairness.  A first step is to measure
how much a given rate allocation deviates from fairness.  We
follow the spirit of \cite{CJ89} and use as a measure of
unfairness the distance between the rate allocation $\vec{x}$ and
its nearest fair allocation $\Pi(\vec{x})$, where $\Pi$ is the
orthogonal projection on the set of fair allocations, normalized
by the length of the fair allocation (Figure~\ref{fig-jain-1}).
\begin{figure}[h]
        \insfig{D31F4}{0.6}
        \mycaption{The measure of unfairness is $\tan(\alpha)$. The figure
        shows the effect on fairness of an increase or a decrease. The
        vector $\vec{1}$ is defined by $\vec{1}_{i}=1$ for all $i$.}
        \protect\label{fig-jain-1}
\end{figure}
In other words, the measure of unfairness is
$$d(\vec{x}) = \frac{||\vec{x} - \Pi(\vec{x})||}{||\Pi(\vec{x})||}$$
with $\Pi(\vec{x})_{i}= \frac{\sum_{j=1}^I x_{j}  }{ I}$ and the
norm is the standard euclidian norm defined by $||\vec{y}|| =
\sqrt{\sum_{j=1}^I y_{j}^2}$ for all $\vec{y}$.

Now we can study the effect of the linear control on fairness.
Figure~\ref{fig-jain-1}) illustrates that: (1) when we apply a
multiplicative increase or decrease, the unfairness is unchanged;
(2) in contrast, an additive increase decreases the unfairness,
whereas an additive decrease increases the unfairness.

We wish to design an algorithm such that at every step, unfairness
decreases or remains the same, and such that in the long term it
decreases. Thus,  we must have $v_{1}=0$, in other words, the
decrease must be multiplicative, and the increase must have a
non-zero additive component. Moreover, if we want to converge as
quickly as possible towards, fairness, then we must have
$u_{0}=0$. In other words, the increase should be purely additive.
In summary we have shown that:
\begin{fact}
Consider a linear adaptation algorithm of the form in
Equation~\ref{eq-theo-addmul1}. In order to satisfy efficiency and
convergence to fairness, we must have a multiplicative decrease
(namely, $u_{1}<1$ and $v_{1}=0$) and a non-zero additive
component in the increase (namely, $u_{0}\geq 1$ and $v_{0}>0$).
If we want to favour a rapid convergence towards fairness, then
the increase should be additive only (namely, $u_{0}= 1$ and
$v_{0}>0$). \label{fact-addmul}
\end{fact}

The resulting algorithm is called Additive Increase Multiplicative
Decrease (\aref{alg:aimd}).

\begin{algorithm}
\caption{Additive Increase Multiplicative Decrease (\nt{AIMD}) with
increase term $v_0>0$ and decrease factor
$0<u_1<1$.}\label{alg:aimd}
\begin{algorithmic}[0]% 0 means no line numbering; 1=number all lines
 \If{received feedback is negative}
 \State multiply rate by $u_1$
 \Else
 \State add $v_0$ to rate
 \EndIf
\end{algorithmic}
\end{algorithm}



 Let us now consider the dynamics of the
control system, in the case of additive increase, multiplicative
decrease. From Formula~\ref{eq-jain-48} we see that the \emph{total}
amount of traffic oscillates around the optimal value $c$.  In
contrast, we see on the numerical examples below that the measure of
unfairness converges to $0$ (This is always true if the conclusions
of Fact~(\ref{fact-addmul}) are followed; the proof is left as an
exercise to the reader). Figure~\ref{fig-addmulNum} shows some
numerical simulations.

The figure also shows a change in the number of active sources. It
should be noted that the value of $u_{1}$ (the multiplicative
decrease factor) plays an important role. A value close to $1$
ensures smaller oscillations around the target value; in contrast,
a small value of $u_{1}$ can react faster to decreases in the
available capacity.
\begin{figure}[h]
        \insfig{D31F7}{1.0}
        \mycaption{Numerical simulations of additive increase, multiplicative
        decrease. There are three sources, with initial rates of 3 Mb/s, 15
        Mb/s and 0. The total link rate is 10 Mb/s. The third source is
        inactive until time unit 100. Decrease factor = $0.5$; increment for
        additive increase: 1 Mb/s. The figure shows the rates for the three
        sources, as well as the aggregate rate and the measure of
        unfairness. The measure of unfairness is counted for two sources
        until time 100, then for three sources.}
        \protect\label{fig-addmulNum}
\end{figure}

Lastly, we must be aware that the analysis made here ignores the
impact of variable and different round trip times.


\subsection{Slow Start}
Slow start is a mechanism that can be combined with Additive
Increase Multiplicative Decrease; it applies primarily to the
initial state of a flow. It is based on the observation that if we
know that a flow receives much less than its fair share, then we can
deviate from Additive Increase Multiplicative Decrease and give a
larger increase to this flow.

We assume that we use Additive Increase Multiplicative Decrease,
with the same notation as in the previous subsection, i.e. with
additive increase term $u_0$ and decrease factor $v_1$.



\begin{figure}
\subfigure[Without Slow
Start]{\Ifignc{3sourcesnoSS}{0.9}{0.3}\label{fig-ss-noss}}
\subfigure[With Slow
Start]{\Ifignc{3sourcesSS}{0.9}{0.3}\label{fig-ss-ss}}
\mycaption{(a) Three sources applying AIMD, the third starting at
time 0 while the other two have reached a high rate. Left: rate of
source 1 on $x$-axis versus rate of source 3. Right: rates of all
sources versus time (number of iterations) on $x$-axis. AIMD
parameters: $v_0=0.01$; $u_1=0.5$. It takes many iterations for the
third source to reach the same rate. (b) Same but with the third
source applying slow start. It quickly converges to a similar rate.
Slow start parameters: $w_0 = 2$, $r_{\max}=10$}
        \protect\label{fig-ss}
\end{figure}

In \fref{fig-addmulNum} we showed flows that started with
arbitrary rates. In practice, it may not be safe to accept such
a behaviour; in contrast, we would like a flow starts with a
very small rate. We set this small rate to $v_0$. Now, a
starting flow in the situation, mentioned earlier, of a flow
competing with established flows, most probably receives less
than all others (\fref{fig-ss-noss}). Therefore, we can, for
this flow, try to increase its rate more quickly, for example
multiplicatively, until it receives a negative feedback. This
is what is implemented in the algorithm called ``slow start"
(\aref{alg:ss}).

\begin{algorithm}
\caption{Slow Start with the following parameters: AIMD constants
$v_0>0$ ,  $0<u_1<1$; multiplicative increase factor $w_0>1$;
maximum rate $r_{\max}>0$. }\label{alg:ss}
\begin{algorithmic}[1]
 \State $\mbox{rate}\gets v_0$\label{alg:ss:ini}
 \State $\mbox{targetRate}\gets r_{\max}$
 \State \textbf{do forever}
 \State receive feedback
 \If{feedback is positive}
 \State $\mbox{rate}\gets w_0 \cdot \mbox{rate}$\label{alg:ss:pos}
 \If{$\mbox{rate}\geq \mbox{targetRate} $ }
    \State $\mbox{rate}\gets \mbox{targetRate}$
    \State \textbf{exit do loop} \label{alg:ss:exit}
 \EndIf
 \Else
 \State $\mbox{targetRate} \gets \max(u_1 \cdot \mbox{rate}, v_0)$\label{alg:ss:rdw}
 \State $\mbox{rate}\gets v_0 $\label{alg:ss:neg}
 \EndIf
 \State \textbf{end do}
\end{algorithmic}
\end{algorithm}

The algorithm maintains both a rate (called $x_i$ in the
previous section) and a target rate. Initially, the rate is set
to the minimum additive increase $v_0$ and the target rate to
some large, predefined value $r_{\max}$ (lines~\ref{alg:ss:ini} and next).
The rate increases multiplicatively as long as positive
feedback is received (line~\ref{alg:ss:pos}). In contrast, if a
negative feedback is received, the target rate is decreased
multiplicatively (this is applied to the rate achieved so far,
line~\ref{alg:ss:rdw}) as with AIMD, and the rate is returned
to the initial value (lines~\ref{alg:ss:neg} and next).

The algorithm terminates when and if the rate reached the target
rate (line~\ref{alg:ss:exit}). From there on, the source applies
AIMD, starting from the current value of the rate. See \fref{fig-ss}
for a simulation (\fref{fig-ss-1}).

Note that what is slow in \emph{slow} start is the starting point $v_0$,
not the increase.


\begin{figure}\Ifig{ss}{0.5}{0.4}
\caption{Zoom on source 3 of \fref{fig-ss-ss} from times 1 to 20,
i.e. during slow start. Dashed line: target rate; plain line: rate.
Slow start ends at time 16.}\label{fig-ss-1}\end{figure}

\section{The fairness of additive increase, multiplicative
decrease with FIFO queues}
 \mylabel{fpaimd}

A complete modeling is very complex because it contains both a
random feedback (under the form of packet loss) and a random delay
(the round trip time, including time for destinations to give
feedback).  In this section we consider that all round trip times
are constant (but may differ from one source to another). The
fundamental tool is to produce an ordinary differential equation
capturing the dynamics of the network,

\subsection{A simplified model}
Call$x_i(t)$ the sending rate for source $i$. Call $t_{n,i}$ the
$n$th rate update instant for source $i$, and let $E_{n,i}$ be the
binary feedback: $E_{n,i}=1$ is a congestion indication; otherwise
$E_{n,i}=0$. Call $1-\eta_i$ the multiplicative decrease factor
and $r_i$ the additive increase component. The source reacts as
follows.
$$
x(t_{n+1,i}) = E_{n,i} (1-\eta_i) x(t_{n,i}) + (1 - E_{n,i})
(x(t_{n,i}) + r_i)
$$
which we can rewrite as \be x(t_{n+1,i}) - x(t_{n,i})
= r - E_{n,i} \left( \eta_i x(t_{n,i}) + r_i\right)
 \label{eq-sys-ode}
\ee If some cases, we can approximate this dynamic
behaviour by an ordinary differential equation (ODE).
The idea, which was developed by Ljung \cite{lju77}
and Kushner and Clark \cite{KC78}, is that the above
set of equations is a discrete time, stochastic
approximation of a differential equation. The ODE is
obtained by writing \bear
    \frac{dx_i}{dt} & = & \mbox{expected rate of change given the state of all
        sources at time $t$} \\\nonumber
    & = &\mbox{expected change divided by the expected update
    interval} \label{eq-ode}
  \eear
The result of the method is that the stochastic system
in Equation~\ref{eq-sys-ode} converges, in some sense,
towards the global attractor of the ODE in
Equation~\ref{eq-ode}, under the condition that the
ODE indeed has a global attractor \cite{BMP90}. A
global attractor of the ODE is a vector
$\vec{x*}=(x^*_i)$ towards which the solution
converges, under any initial conditions. The
convergence is for $\eta_i$ and $r_i$ tending to $0$.
Thus the results in this section are only
asymptotically true, and must be verified by
simulation.

Back to our model, call $\mu_i(t, \vec{x})$ the expectation of
$E_{n,i}$, given a set of rates $\vec{x}$; also call $u_i(t)$ the
expected update interval for source $i$. The ODE is:
\begin{equation}\mylabel{eq-odedebase}
   \frac{dx_i}{dt}
        =
   \frac{r_i - \mu_i(t, \vec{x}(t) )\left( \eta_i x_i(t) + r_i \right)}{u_i(t)}
\end{equation}

We first consider the original additive increase, multiplicative
decrease algorithm, then we will study the special case of TCP.

\subsection{Additive Increase, Multiplicative Decrease with one Update per RTT}
We assume that the update interval is a constant $\tau_i$ equal to
the round trip time for source $i$. Thus
$$
u_i(t) = \tau_i
$$
The expected feedback $\mu_i$ is given by
\begin{equation}
        \mu_i(t,\vec{x}(t)) = \tau x_{i}(t) \sum_{l=1}^{L} g_{l}(f_{l}(t))
        A_{l,i}
        \label{eq-feedback-2}
\end{equation}

with $f_{l}(t)= \sum_{j=1}^I A_{l,j}x_{j}(t)$.  In the formula,
$f_{l}(t)$ represents the total amount of traffic flow on link
$l$, while $A_{l,i}$ is the fraction of traffic from source $i$
which uses link $l$.  We interprete Equation~(\ref{eq-feedback-2})
by assuming that $g_{l}(f)$ is the probability that a packet is
marked with a feedback equal to 1 (namely, a negative feedback) by
link $l$, given that the traffic load on link $l$ is expressed by
the real number $f$.
%; in the regime of rare negative feedback, we
%assume that we can neglect the occurence of one packet marked with
%a negative feedback on several links within one time cycle.
Then Equation~(\ref{eq-feedback-2}) simply gives the expectation
of the number of marked packets received during one time cycle by
source $i$. This models accurately the case where the feedback
sent by a link is quasi-stationary, as can be achieved by using
active queue management such as RED (see later). This also assumes
that the propagation time is large compared to the transmission
time.

Putting all this together we obtain the ODE:

 \begin{equation}
        \frac{dx_{i}}{dt}
        =
        \frac{r_i}{\tau_i} -  x_{i} (r_{i}+\eta_i x_{i}) \sum_{l=1}^{L}
        g_{l}(f_{l})A_{l,i}
        \label{eq-ode-cas-A}
  \end{equation}

  with
  \begin{equation}
        f_{l} = \sum_{j=1}^{I} A_{l,j} x_{j}
        \label{eq-flux-l}
  \end{equation}

  In order to study the attractors of this ODE, we identify a Lyapunov
  for it \cite{refSurLyapunov}. To that end, we follow \cite{KMT97}
  and \cite{GB98} and note that
  $$
  \sum_{l=1}^{L}
        g_{l}(f_{l})A_{l,i} = \frac{\partial}{\partial x_{i}}\sum_{l=1}^{L}
        G_{l}(f_{l}) = \frac{\partial G (\vec{x})}{\partial x_{i}}
  $$
  where $G_{l}$ is a primitive of $g_{l}$ defined for example by
  $$
  G_{l}(f) = \int_{0}^f g_{l}(u) du
  $$
  and
  $$G(\vec{x}) = \sum_{l=1}^{L} G_{l}(f_{l})
  $$
  We can then rewrite Equation~(\ref{eq-ode-cas-A}) as
  \begin{equation}
                \frac{dx_{i}}{dt}
        =
        x_{i}(r_{i}+ \eta_i x_{i})
        \left \{
            \frac{r_{i}}{\tau_i x_{i}(r_{i}+ \eta_i x_{i})}  -
            \frac{\partial G (\vec{x})}{\partial x_{i}}
        \right \}
        \label{eq-cas-A-modif}
  \end{equation}

  Consider now the function $J_{A}$ defined by
  \begin{equation}
        J_{A}(\vec{x}) = \sum_{i=1}^I \phi(x_{i}) -  G(\vec{x})
        \label{eq-def-de-J}
  \end{equation}

  with
  $$\phi(x_{i}) = \int_{0}^{x_{i}} \frac{r_{i}du}{\tau_i u(r_{i}+ \eta_i u)}=
  \frac{1}{\tau_i}\log \frac{x_{i}}{ r_{i}+ \eta_i x_{i}}
  $$

  then we can rewrite Equation~(\ref{eq-cas-A-modif}) as

  \begin{equation}
                \frac{dx_{i}}{dt}
        =
        x_{i}(r_{i}+ \eta_i x_{i})
        \frac{\partial J_{A} (\vec{x})}{\partial x_{i}}
        \label{eq-cas-A-modif-2}
  \end{equation}

  Now it is easy to see that $J_{A}$ is strictly concave and therefore has a unique
  maximum over any bounded region. It follows from this and from
  Equation~(\ref{eq-cas-A-modif-2}) that $J_{A}$ is a Lyapunov for the ODE
  in~(\ref{eq-ode-cas-A}), and thus, the ODE
  in~(\ref{eq-ode-cas-A}) has a unique attractor, which is the point
  where the maximum of $J_{A}$ is reached. An intuitive explanation of the Lyapunov is as
  follows. Along any
  solution $\vec{x}(t)$ of the ODE, we have
  $$
\frac{d}{dt}J_A(\vec{x}(t)) = \sum_{i=1}^{I} \frac{\partial
J_A}{\partial x_{i}}\frac{dx_i}{dt}
=
\sum_{i=1}^{I} x_i(r_{i}+ \eta_i x_{i})
 \left( \frac{\partial J_A}{\partial x_{i}} \right)^2
  $$
  Thus $J_A$ increases along any solution, thus solutions tend to
  converge towards the unique maximum of $J_A$.

  This shows that
  the rates $x_{i}(t)$ converge at equilibrium towards a
  set of value that maximizes $J_{A}(\vec{x})$, with $J_{A}$ defined by
  $$
  J_{A}(\vec{x}) =
  \sum_{i=1}^I
   \frac{1}{\tau_i}\log \frac{x_{i}}{ r_{i}+ \eta_i x_{i}}
   \; -  \; G(\vec{x})
  $$
\paragraph{Interpretation}
In order to interpret the previous results, we follow \cite{KMT97}
and assume that, calling $c_{l}$ the capacity of link $l$, the
function $g_{l}$ can be assumed to be arbitrarily close to
$\delta_{c_{l}}$, in some sense, where
$$
\delta_{c}(f) = 0 \mif f < c \mand \delta_{c}(f) = + \infty \mif f
\geq c
$$

Thus, at the limit, the method in \cite{KMT97} finds that the
rates are distributed so as to maximize
$$F_{A}(\vec{x})=\sum_{i=1}^I \frac{1}{\tau_i}\log \frac{x_{i}}{ r_{i}+ \eta_i x_{i}}$$
subject to the constraints
$$
\sum_{j=1}^I A_{l,j}x_{j} \; \leq c_{l} \; \mfa l
$$

We can learn two things from this. Firstly, the weight given to
$x_{i}$ tends to $-\log \eta_i$ as $x_{i}$ tends to $+\infty$.
Thus, the distribution of rates will tend to favor small rates,
and should thus be closer to max-min fairness than to proportional
fairness. Secondly, the weight is inversely proportional to the
round trip time, thus flows with large round trip times suffer
from a negative bias, independent of the number of hops they use.
%
%We see that, under the simplifying assumptions above, the additive
%increase, multiplicative decrease principle distributes rates
%according to a utility fairness, which is between max-min fairness
%and proportional fairness.

\subsection{Additive Increase, Multiplicative Decrease with one
Update per Packet}
 Assume now that a source reacts to every feedback received.
 Assume that losses are detected immediately, either because the
 timeout is optimal (equal to the roundtrip time), or because of
 some other clever heuristic. Then we can use the same analysis as
 in the previous section, with the following adaptations.

 The ODE is still given by Equation (\ref{eq-odedebase}). The
 expected feedback is now simply equal to the probability of a
 packet being marked, and is equal to
 $$
 \mu_i(t, \vec{x}) = \sum_{l=1}^{L}
        g_{l}(f_{l})A_{l,i}
 $$

 The average update interval is now equal to $\frac{1}{x_i}$. Thus
 the ODE is
\begin{equation}\mylabel{eq-odeevpack}
\frac{dx_{i}}{dt}
        =
        r_i x_i -  x_{i} (r_{i}+\eta_i x_{i}) \sum_{l=1}^{L}
        g_{l}(f_{l})A_{l,i}
\end{equation}

which can also be written as
\begin{equation}\mylabel{eq-odeevpack-2}
\frac{dx_{i}}{dt}
        =
 x_{i}(r_{i}+ \eta_i x_{i})
        \left \{
            \frac{r_{i}}{r_{i}+ \eta_i x_{i}}  -
            \frac{\partial G (\vec{x})}{\partial x_{i}}
        \right \}
\end{equation}

Thus a Lyapunov for the ODE is
$$
J_B(\vec{x})= \sum_{i=1}^I
   \frac{r_i}{\eta_i}\log ( r_{i}+ \eta_i x_{i})
   \; -  \; G(\vec{x})
$$

Thus, in the limiting case where the feedback expectation is close
to a Dirac function, the rates are distributed so as to maximize
$$F_{B}(\vec{x})=\sum_{i=1}^I \frac{r_i}{\eta_i}\log ( r_{i}+ \eta_i x_{i})$$
subject to the constraints
$$
\sum_{j=1}^I A_{l,j}x_{j} \; \leq c_{l} \; \mfa l
$$
The weight given to $x_{i}$ is close to $\log(\eta_i x_{i})$ ($=
\log x_i + \mbox{a constant} $) for large $x_i$, and is larger
than $\log(\eta_i x_{i})$ in all cases. Thus, the distribution of
rates is derived from proportional fairness, with a positive bias
given to small rates. Contrary to the previous case, there is no
bias against long round trip times.
%

In Section~\ref{sec-tcpalgodecon} on
page~\pageref{sec-tcpalgodecon} we study the case of TCP.
\section{Summary}\begin{enumerate}
    \item In a packet network, sources should limit their sending rate in order
    to account for the state of the network. Failing
    to do so might result in congestion collapse.
    \item Congestion collapse is defined as a severe decrease in total
    network throughput when the offered load increases.
    \item Maximizing network throughput as a primary objective might lead to
large unfairness and is not a viable objective.
  \item The
objective of congestion control is to provide Pareto efficiency and
some form of fairness.


\item Fairness can be defined in various ways: max-min,
proportional and variants of proportional.

\item End-to-end congestion control in packet networks is based on the adaptation mechanism of additive increase,
multiplicative decrease.

\item Slow start is a mechanism for a source to quickly increase its sending rate (instead of starting upfront at a high rate).
\end{enumerate}
%
% calcul de fair share a la golestani sur un exemple (Kuhn Tucker)
%
% objectif perte 0.05
%
% 4.3.2 de Jin Roberts: forme explicite du throughput pour rate prop. fair
% sharing
%
% reprendre mon exmple et calculer le temps moyen en utilisant JimM
% Robert section 4.3.2
%
% exo: calculer le max min fair share pour l'exemple de JEC papier EFOC
%      comparer au throughput total
%      calculer rate proportional fairness pour cet exemple
%
% peut etre: forme produit et methode de Metropolis
%
% peut etre: fixed window de JR
%
% \section{Appendix: Optimization}
%
% Linear programming
%
% Lagrangian Methods
%
% Kuhn Tucker: theoreme de Leunberger page 401 et Gustavo de veciana
%    Lagrangien
%    conditions de KT: page 401 de Luenberger
