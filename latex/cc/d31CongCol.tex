
Consider a network where sources may send at a rate limited
only by the source capabilities. Such a network may suffer of
congestion collapse, which we explain now on an example.

We assume that the only resource to allocate is link bit
rates. We also assume that if the offered traffic on some link
$l$ exceeds the capacity $c_{l}$ of the link, then all sources
see their traffic reduced in proportion of their offered
traffic. This assumption is approximately true if queuing is
first in first out in the network nodes, neglecting possible
effects due to traffic burstiness.

Consider first the network illustrated on Figure~\ref{D31-f1}.
Sources 1 and 2 send traffic to destination nodes D1 and D2
respectively, and are limited only by their access rates.
There are five links labeled 1 through 5 with capacities shown
on the figure. Assume sources are limited only by their first
link, without feedback from the network. Call $\lambda_i$ the
sending rate of source $i$, and $\lambda'i$ the outgoing rate.

For example, with the values given on the figures we find
$\lambda_1=100$kb/s and $\lambda_2=1000$kb/s, but only
$\lambda'_1=\lambda'_2=10$kb/s, and the total throughput is
$20$kb/s~! Source 1 can send only at 10 kb/s because it is
competing with source 2 on link 3, which sends at a high rate
on that link; however, source 2 is limited to 10 kb/s because
of link 5. If source 2 would be aware of the global situation,
and if it would cooperate, then it would send at 10 kb/s only
already on link 2, which would allow source 1 to send at 100
kb/s, without any penalty for source 2. The total throughput
of the network would then become $\theta = 110 kb/s$.

\begin{figure}[htbp]
        \insfig{D31F1}{0.7}
        \mycaption{A simple network exhibiting some inefficiency if sources
        are not limited by some feedback from the network}
        \protect\label{D31-f1}
\end{figure}

The first example has shown some inefficiency. In complex
network scenarios, this may lead to a form of instability
known as congestion collapse. To illustrate this, we use the
network illustrated on Figure~\ref{D31-f1n}. The topology is a
ring; it is commonly used in many networks, because it is a
simple way to provide some redundancy. There are $I$ nodes and
links, numbered $0, 1, ..., I-1$. Source $i$ enters node $i$,
uses links $[(i+1) \bmod I] $ and $[(i+2) \bmod I]$, and
leaves the network at node $(i+2) \bmod I$. Assume that source
$i$ sends as much as $\lambda_{i}$, without feedback from the
network. Call $\lambda'_{i}$ the rate achieved by source $i$
on link $[(i+1) \bmod I]$ and $\lambda''_{i}$ the rate
achieved on link $[(i+2) \bmod I]$. This corresponds to every
source choosing the shortest path to the destination. In the
rest of this example, we omit ``$\bmod I$" when the context is
clear. We have then:
\begin{equation}
 \left \{
  \begin{array}{l}
  \lambda'_i = \min
     \left(\lambda_i, \frac{c_i}{\lambda_i +
     \lambda'_{i-1}}\lambda_i
     \right) \\
  \lambda''_i = \min
     \left(\lambda'_i, \frac{c_{i+1}}{\lambda'_i +
     \lambda_{i+1}}\lambda'_i
     \right) \\
  \end{array}
  \right.
    \mylabel{D31-eqkkkLAUW}
\end{equation}

\begin{figure}[htbp]
        \insfig{D31F1n}{0.35}
        \mycaption{A network exhibiting congestion collapse if sources
        are not limited by some feedback from the network}
        \mylabel{D31-f1n}
\end{figure}


Applying Equation~\ref{D31-eqkkkLAUW} enables us to compute
the total throughput $\theta$. In order to obtain a closed
form solution, we further study the symmetric case, namely, we
assume that $c_i=c$ and $\lambda_i= \lambda$ for all $i$. Then
we have obviously $\lambda'_i= \lambda'$ and $\lambda''_i=
\lambda''$ for some values of $\lambda'$ and $\lambda''$ which
we compute now.

If $\lambda \leq \frac{c}{2}$ then there is no loss and
$\lambda''=\lambda'=\lambda$ and the throughput is $\theta = I
\lambda$. Else, we have, from Equation~(\ref{D31-eqkkkLAUW})
$$
 \lambda' = \frac{c \lambda}{\lambda + \lambda'}
$$
We can solve for $\lambda'$ (a polynomial equation of degree
2) and obtain
$$
\lambda'=\frac{\lambda}{2} \left( -1 + \sqrt{1 + 4
\frac{c}{\lambda}}\right)
$$
We have also from Equation~(\ref{D31-eqkkkLAUW})
$$
 \lambda'' = \frac{c \lambda'}{\lambda + \lambda'}
$$
Combining the last two equations gives
$$ \lambda'' = c -\frac{\lambda}{2}\left( \sqrt{1 + 4 \frac{c}{\lambda}} -1 \right)
$$
Using the limited development, valid for $u \rightarrow 0$
 $$\sqrt{1+u} = 1 + \frac{1}{2}u - \frac{1}{8}u^2 + o(u^2)$$we have
 $$  \lambda'' = \frac{c^2}{\lambda} + o(\frac{1}{\lambda})$$
Thus, the limit of the achieved throughput, when the offered
load goes to $+\infty$, is $0$. This is what we call
\emph{congestion collapse}.

Figure~\ref{D31-f2n} plots the throughput per source
$\lambda''$ as a function of the offered load per source
$\lambda$. It confirms that after some point, the throughput
decreases with the offered load, going to $0$ as the offered
load goes to $+\infty$.
\begin{figure}[htbp]
        \insfig{D31F2n}{0.6}
        \mycaption{Throughput per source as a function of the offered load per
        source, in Mb/s, for the network of Figure~\ref{D31-f1n}. Numbers are in Mb/s. The
        link rate is $c=20$Mb/s for all links.}
        \protect\label{D31-f2n}
\end{figure}

The previous discussion has illustrated the following fact:

\begin{fact}[Efficiency Criterion]
        In a packet network, sources should limit their sending rate by
        taking into consideration the state of the network.  Ignoring this
        may put the network into congestion collapse. One objective of
        congestion control is to avoid such inefficiencies.
\end{fact}

Congestion collapse occurs when some resources are consumed by
traffic that will be later discarded. This phenomenon did
happen in the Internet in the middle of the eighties. At that
time, there was no end-to-end congestion control in TCP/IP. As
we will see in the next section, a secondary objective is
fairness.
