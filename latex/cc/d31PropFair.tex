
The previous definition of fairness puts emphasis on
maintaining high values for the smallest rates. As shown in
the previous example, this may be at the expense of some
network inefficiency. An alternative definition of fairness
has been proposed in the context of game theory \cite{MMD91}.

\begin{definition}[Proportional Fairness]
An allocation of rates $\vec{x}$ is ``proportionally fair'' if and only if all rates are positive and, for any other feasible allocation $\vec{y}$, we
have:
$$
\sum_{s=1}^S \frac{y_{s}-x_{s}}{x_{s}} \; \leq 0
$$
        \label{def-pf}
\end{definition}

In other words, any change in the allocation must have a
negative average change.



Let us consider for example the
parking lot scenario in \fref{D31-f3p}. Is the
max-min fair allocation proportionally fair~?


\begin{figure}[htbp]
        \insfig{D31F3p}{0.7}
        \mycaption{A simple network used to illustrate proportional fairness (the``parking
        lot'' scenario)}
        \protect\label{D31-f3p}
\end{figure}



To get the answer, observe that the max-min fair
allocation is obtained easily with water-filling and is $x_{s}=c/2$ for $s=0,1..4$. Consider a new allocation
resulting from a decrease of $x_{0}$ equal to $\delta$:
\begin{eqnarray*}
         y_{0}  &= \frac{c}{2} - \delta \\
        y_{s} &= \frac{c}{2} + \delta \; s=1,\ldots,4
\end{eqnarray*}
For $\delta<\frac{c}{2}$, the new allocation $\vec{y}$ is
feasible. The average rate of change is
$$
\left( \sum_{s=1}^4 \frac{2 \delta }{ c} \right)
 -  \frac{2 \delta }{ c} =
\frac{6\delta }{c}
$$
which is positive.  Thus the max-min fair
allocation for this example is not proportionally fair.  In this example, we see that a decrease in rate for
sources of type $0$ is less important than the corresponding
increase which is made possible for the other sources, because
the increase is multiplied by the number of sources.
Informally, we say that proportional fairness takes into
consideration the usage of network resources.

\begin{theorem}A proportionally fair allocation is Pareto-efficient.
\end{theorem}
\pr Let $\vec{x}$ be a proportionally fair allocation and let $\vy$ be some feasible allocation such that $y_i>x_i$ for some $i$. The rate of change is
$$
\frac{y_i-x_i}{x_i}+\sum_{s=1...S, s\neq i}\frac{y_s-x_s}{x_s}\leq 0
$$ because $\vec{x}$ is proportionally fair. The first term is positive, therefore at least one of the other terms, say $\frac{y_s-x_s}{x_s}$ must be negative, and thus $y_s<x_s$. This shows that $\vec{x}$ is Pareto-efficient.
\qed


Now we derive a practical result which can be used to compute
a proportionally fair allocation.  To that end, we interpret
the average rate of change as $\nabla J_{\vec{x}}\cdot
(\vec{y}-\vec{x})$, with
$$
J(\vec{x})= \sum_{s} \ln(x_{s})
$$
Thus, intuitively, a proportionally fair allocation should
maximize $J$.
\begin{theorem}
There exists one unique proportionally fair allocation. It is
obtained by maximizing $J(\vec{x})= \sum_{s} \ln(x_{s})$ over
the set of feasible allocations.
        \label{theo-pf}
\end{theorem}
\pr 1. We first prove that the maximization problem has a unique
solution.  Function $J$ is concave, as a sum of concave
functions. The feasible set is convex, as intersection of
convex sets, thus any local maximum of $J$ is an absolute
maximum.  Now $J$ is strictly concave, which means that
$$\mif 0< \alpha< 1 \mthen J(\alpha \vec{x}
+ (1-\alpha) \vec{y}) > \alpha J(\vec{x}) + (1-\alpha)
J(\vec{y})
$$
This can be proven by studying the second derivative
of the restriction of $J$ to any linear segment.  Now
a strictly concave function has at most one maximum on
a convex set. %(Chapter~\ref{D4-fluid11}).

Now $J$ is continuous if we allow $\log(0)=-\infty$ and the
set of feasible allocations is compact (because it is a
closed, bounded subset of $\Reals^S$). Thus $J$ has at least
one maximum over the set of feasible allocations.
Combining all the arguments together proves that $J$ has
exactly one maximum over the set of feasible allocations, and
that any local maximum is also exactly the global maximum.

2. For any $\vec{\delta}$ such that $\vec{x}+ \vec{\delta}$
is feasible,

$$
J(\vec{x}+ \vec{\delta})-J(\vec{x})=\nabla J_{\vec{x}}\cdot
\vec{\delta} + \frac{1}{2}   \vec{\delta}^T \nabla^2
J_{\vec{x}} \vec{\delta} +  o(||\vec{\delta}||^2)
$$
Now by the strict concavity, $\nabla^2 J_{\vec{x}}$ is
definite negative thus, for
$||\vec{\delta}||$ small enough:
$$\frac{1}{2}  \vec{\delta}^T \nabla^2
J_{\vec{x}} \vec{\delta} +  o(||\vec{\delta}||^2) \; <0 $$
and therefore
$$
J(\vec{x}+ \vec{\delta})-J(\vec{x})\leq \nabla J_{\vec{x}}\cdot
\vec{\delta}
$$


Now assume that $\vec{x}$ is a proportionally fair allocation.
This means that

$$\nabla(J)_{\vec{x}}\cdot \vec{\delta} \leq 0$$
and thus $J$ has a local maximum at $\vec{x}$, thus also a
global maximum. This also shows the uniqueness of a
proportionally fair allocation.


3. Conversely, assume that $J$ has a global maximum at $\vec{x}$.
Observe that our network model assumes that there exists some positive and feasible allocation, therefore, the maximum of $J$ is not $-\infty$, therefore the maximiser $\vec{x}$ satisfies $x_s>0$ for all $s$. Let $\vec{y}$ be some other feasible allocation and call $D$ the
average rate of change. We also have:
$$ D = \nabla(J)_{\vec{x}}\cdot (\vec{y}-\vec{x}) $$
Since the feasible set is convex, the segment $[\vec{x},
\vec{y}]$ is entirely feasible, and thus $J(\vec{x} + t (\vec{y}-\vec{x})) \leq  J(\vec{x} )$ for $t\in [0;1]$. Observe that
$$D = \lim_{t \rightarrow 0^+}
\frac{J(\vec{x} + t (\vec{y}-\vec{x})) - J(\vec{x} ) }{ t}
$$
thus $D \leq 0$. \qed
\paragraph{Example}
Let us apply Theorem~\ref{theo-pf} to the parking lot
scenario in \fref{D31-f3p}. For any choice of $x_{0}$, we should set $x_{i}$
such that
$$
x_{0} + x_{i} = c,  i=1,\ldots,4
$$
otherwise we could increase $x_{i}$ without affecting other
values, and thus increase function $J$. The value of $x_{0}$
is found by maximizing $f(x_{0})$, defined by
$$
f(x_{0}) =  \ln(x_{0}) + \sum_{i=1}^4  \ln(c
-x_{0})
$$
over the set $0 \leq x_{0} \leq c$. The
derivative of $f$ is
$$
f'(x_{0}) =\frac{1}{x_{0}} - \frac{4}{ c- x_{0}}
$$
After some algebra, we find that the maximum is for
$$
x_{0}= \frac{c}{5}
$$
and
$$
x_{i}= \frac{4c}{5}\mfor i=1...4
$$
%For example, if $n_{i}=1$ for all $i=0,\ldots,4$, we obtain:
%\begin{eqnarray*}
%         x_{0}&= \frac{c }{4 +1} \\
%        x_{i} &= \frac{cI}{4+1}
%\end{eqnarray*}
Compare with max-min fairness, where, in that case, the
allocation is $\frac{c}{2}$ for all rates.  We see that
sources of type $0$ get a smaller rate, since they use more
network resources.

The concept of proportional fairness can easily extended to
\emph{weighted} proportional fairness, where the allocation
maximizes a weighted sum of logarithms \cite{KMT97}.

