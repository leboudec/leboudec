\documentclass[12pt,oneside,onecolumn,a4paper]{book}
%\usepackage{epsfig,latexsym}
%%%%%%%%%%%%%%%%%%%%%%%%%%%%%%%%%%%%%%%%%%%%%%%%%
%%%%%%%% This is Leb's usual stuff  %%%%%%%%%%%%%
%%%%%%%%                            %%%%%%%%%%%%%
%%%%%%%%%%%%%%%%%%%%%%%%%%%%%%%%%%%%%%%%%%%%%%%%%

\usepackage{lebListes}  %%% modifie les bullets selon AMS et fournit les macros suivants
\noitemsep  %%% no spacing between list items
%\doitemsep %%% retablir spacing par defaut

%%%% euro symbol by Alain Delplanque
\def\euro{\mbox{\raisebox{.25ex}{{\it =}}\hspace{-.5em}{\sf C}}}



\usepackage{ifthen}


\usepackage{algorithm,algpseudocode}

%% Bremaud's stuff
\usepackage{color,pstcol,pstricks}
\newgray{backgroundgray}{0.90}
\definecolor{backgroundgray}{gray}{0.90}

%\usepackage{xspace}
%\newcounter{example}[section]
%\renewcommand{\theexample}{\arabic{section}.\arabic{example}}
%\newlength{\labelexample}
%\setlength{\labelexample}{2cm}
%\newenvironment{example}[1]
%{\refstepcounter{example}\par\noindent{\gray\rule{\labelexample}{1pt}}\par\vspace*{-.2\baselineskip}\noindent\textsc{Example~\theexample:\xspace}\ifthenelse{\equal{#1}{}}{}{{\sc\blue#1}}}
%{\nobreak\par\nobreak{\gray\hfill\rule{\labelexample}{1pt}}\par} %\vspace{-.5\baselineskip}

\newcommand{\eqframe}[1]{{\psframebox[linecolor=gray, framesep=4pt]{#1}}}

\newenvironment{preuve}
{\begin{quote}
\par\vspace{.5\baselineskip}\noindent\footnotesize\imp{Proof.}}
{\end{quote}\nobreak\hfill$\Box$\par\vspace{.5\baselineskip}}

\newenvironment{petit}
{\par\vspace{.5\baselineskip}\noindent\footnotesize}
{\nobreak\par\vspace{.5\baselineskip}}


\newenvironment{petiteNote}
{\begin{petit} \begin{quote} \imp{Note. }}
 {\end{quote}\end{petit}}



%% end Bremaud's stuff

%%%%%%%% paths, system dependent   %%%%%%%%
%\def\figuresfolder{../exoFigures/}
%\def\sourcefolder{../sourceFiles/}
\def\tempfolder{C://tmp/}
%\def\tempfolder{/Documents and Settings/leboudec/Desktop/evidbackup/leboudec/DocsDeCours/Common/BD d'exos/Temp/}
%%%%%%% end of system dependent part %%%%%

%%% set the flags to default values, unless they are already defined in
%% the main file

\providecommand{\imprimeTexteCache}{non}
\providecommand{\imprimeTexteSecret}{non}


%%% do not modify these lines
\newif\ifsol
\newif\ifnfs

\ifthenelse{\equal{\imprimeTexteCache}{oui}}{\soltrue}{\solfalse }
\ifthenelse{\equal{\imprimeTexteSecret}{oui}}{\nfstrue}{\nfsfalse}

%%% insert solutions or comments not for students with \sol, \ifsol...\fi
%%%  and \nfs



\ifsol
   \providecommand{\sol}[1]{\bs #1 \es}
\else
   \providecommand{\sol}[1]{}
\fi

\ifnfs
   \providecommand{\nfs}[1]{~\\{\footnotesize \sffamily \emph{\textbf{Note: } #1}~\\}}
\else
   \providecommand{\nfs}[1]{}
\fi

%% new method for solutions
%% use \bs \es instead

\newenvironment{solution}
{\red \it
\par\vspace{.5\baselineskip}\noindent\textbf{Solution.~}}
{\vspace{.5\baselineskip}}
\def\bs{\begin{solution}}
\def\es{\end{solution}}





\def\problemfile#1{%
\begin{filecontents}{\tempfolder#1}}

%\providecommand{\finproblemfile}{
%\end{filecontents}}

\providecommand{\m}[1]{
   \mylabel{#1}
   \input{\tempfolder#1}
   }



% \providecommand{\gradCourse}{oui}
% eventually modify to become compatible with my database of exercises
% \providecommand{\pointex}[1]{\fbox{$\rightarrow$ Exercise \ref{#1}}}


%%%% include graphics only with these commands %%%%%%%%%%%%%%%
\ifnfs
\providecommand{\insfignc}[2]{% same as insfig, no centering
 \includegraphics[width=#2\textwidth,keepaspectratio]{#1.eps}
 \footnote{\fbox{\texttt{file: #1}}}
 }

\else
\providecommand{\insfignc}[2]{% same as insfig, no centering
 \includegraphics[width=#2\textwidth,keepaspectratio]{#1.eps}
 }

\fi


\ifnfs

\providecommand{\Ifignc}[3]{% same as Ifig, no centering
 \includegraphics[width=#2\textwidth,height=#3\textwidth]{#1.eps}
 \footnote{\fbox{\texttt{file: #1}}}
 }

\else

\providecommand{\Ifignc}[3]{% same as Ifig, no centering
 \includegraphics[width=#2\textwidth,height=#3\textwidth]{#1.eps}
 }

\fi

\providecommand{\insfig}[2] {
\begin{center}
\insfignc{#1}{#2}\\
\end{center}
}

\providecommand{\Ifig}[3]{% modify width and height separately
 \begin{center}
 \Ifignc{#1}{#2}{#3}\\
 \end{center}
 }


\providecommand{\ifig}[2]{%% obsolete, for compatibility
 \insfignc{#1}{#2}
 }
%\providecommand{\ifig}[2] {\epsfig{file=#1.eps,
%width=#2\textwidth}}

% macros to be used instead of \label, \marginpar and \caption
% values of labels are printed when \imprimeTexteSecret is equal to oui
\newcommand{\mycaption}[1]{\caption{\footnotesize \sffamily #1}}
\newcommand{\mymarginpar}[1]{
    \marginpar{\sffamily \footnotesize #1}
    }
\ifnfs
\newcommand{\mylabel}[1]{
    %\fbox{\ttfamily \footnotesize #1 \label{#1}
    \label{#1}
    \footnote{\fbox{\texttt{label: #1}}}
  }
\else
\newcommand{\mylabel}[1]{
    \label{#1}
    }
\fi

\newcommand{\petitcar}[1]  {
   {\small \sffamily #1}
   }


%%%%%%%%%%%%%%%%%%% " to be done " flag %%%%%%%%%%%%%%%%%%%
%% printed only if solutions are printed
%%%%%%%%%%%%%%%%%%%%%%%%%%%%%%%%%%%%%%%%%%%%%%%%%%%%%%%%%%%
\ifsol
\newcommand{\tbd}[1]{\textsc{To be done:\\#1}}
\else
\newcommand{\tbd}[1]{}
\fi


%%%%%%%%%%% inline questions %%%%%%%%%%%%%%%
% use mq for inline questions

\newtheorem{maquestion}{\sc Question}[section]
\newcommand\metEnFin[3]{
   %\AtEndDocument{\textbf{Question \ref{#1} on page \pageref{#1}: } #2~\\~\textbf{Answer. }#3~\\~\\}%
   \AtEndDocument{{\red{\sc\textbf{\ref{#1} (p.~\pageref{#1}). }} #2}~\\~{\sc \textbf{Answer. }}#3~\\~\\}%
   }
%\newcounter{monNumeroDeQuestion}
%\newcommand{\str}[2]{\addtocounter{strophe}{1}\thestrophe
%\newcommand\mq[3]{%%%label, question, reponse
%  \begin{maquestion}
%  \label{#1} #2 ~\\ \textbf{Solution.} #3
%  \end{maquestion}%
%  }

\ifsol
\newcommand\mq[3]{%%%label, question, reponse
  {
  \begin{maquestion}
  \label{#1} #2
  \end{maquestion}
  {\red \sc{ A. }
  }
   {\footnotesize \sffamily #3~\\}
   }
  }
\else
%\newcommand\mq[3]{%%%label, question, reponse
% {
%  \begin{maquestion}
%  \label{#1} #2
%  \end{maquestion}%
%  }
%  \metEnFin{#1}{#2}{#3}
%  }
  \newcommand\mq[3]{%%%label, question, reponse
  \begin{maquestion}
  \label{#1} #2 \footnote{#3}
  \end{maquestion}%
  }
\fi

%%%%%%% new term %%%%%%%%%%%%%%%%%%%%%%%%%%%%%
\newcommand{\nt}[1]{{\blue \it \textsf{#1}\index{#1}}}
\newcommand{\imp}[1]{{\blue \bf #1}}

%%%%%%% various shortcuts  %%%%%%%%%%%%%%%%%%%


%\newcommand{\mas}[1]{\section[#1]{\sc #1}}
%\newcommand{\mass}[1]{\subsection[#1]{\sc #1}}
%\newcommand{\masss}[1]{\subsubsection[#1]{\sc #1}}
%\newcommand{\monp}[1]{\paragraph{\sc #1}}

%%% for including pseudo code in text
\newcommand{\bp}{ % begin program
 \vspace{-3mm}\small \ttfamily
 \begin{tabbing}
 aaa\=aaa\=aaa\=aaa\=aaaaaaaa \= aaaaaaaaaa\= \kill
 }
\newcommand{\ep}{\end{tabbing}\normalfont\normalsize\vspace{-3mm}}% end program

%\newcommand{\pro}[1]{\ttfamily\small#1\normalfont\normalsize}
\newcommand{\pro}[1]{{\ttfamily\small#1}}

\newcommand{\aref}[1]{Algorithm~\ref{#1}}
\newcommand{\fref}[1]{Figure~\ref{#1}}
\newcommand{\sref}[1]{Section~\ref{#1}}
\newcommand{\cref}[1]{Chapter~\ref{#1}}
\newcommand{\coref}[1]{Corollary~\ref{#1}}
\newcommand{\eref}[1]{Equation~(\ref{#1})}
\newcommand{\tref}[1]{Table~\ref{#1}}
\newcommand{\dref}[1]{Definition~\ref{#1}}
\newcommand{\lref}[1]{Lemma~\ref{#1}}
\newcommand{\thref}[1]{Theorem~\ref{#1}}
\newcommand{\pref}[1]{Proposition~\ref{#1}}
\newcommand{\pgref}[1]{Page~\pageref{#1}}
\newcommand{\qref}[1]{Question~\ref{#1}}

\newcommand{\paref}[1]{Algorithm~\ref{#1} on page~\pageref{#1}}
\newcommand{\pfref}[1]{Figure~\ref{#1} on page~\pageref{#1}}
\newcommand{\psref}[1]{Section~\ref{#1} on page~\pageref{#1}}
\newcommand{\pcref}[1]{Chapter~\ref{#1} on page~\pageref{#1}}
\newcommand{\pcoref}[1]{Corollary~\ref{#1} on page~\pageref{#1}}
\newcommand{\peref}[1]{Equation~(\ref{#1}) on page~\pageref{#1}}
\newcommand{\ptref}[1]{Table~\ref{#1} on page~\pageref{#1}}
\newcommand{\pdref}[1]{Definition~\ref{#1} on page~\pageref{#1}}
\newcommand{\plref}[1]{Lemma~\ref{#1} on page~\pageref{#1}}
\newcommand{\pthref}[1]{Theorem~\ref{#1} on page~\pageref{#1}}
\newcommand{\ppref}[1]{Proposition~\ref{#1} on page~\pageref{#1}}
\newcommand{\pqref}[1]{Question~\ref{#1} on page~\pageref{#1}}
\newcommand{\exref}[1]{Example~\ref{#1} on page~\pageref{#1}}

\newcommand{\maketline}{\noindent \rule{16.5cm}{0.02in} \newline }
\newcommand{\makeline}{\noindent \rule{16.5cm}{0.04in}}

%\newtheorem{st}{Student Tutorial}[section]
%\newtheorem{exo}{Exercise}[section]
%\newtheorem{theorem}{Theorem}[section]
%\newtheorem{lemma}{Lemma}[section]
%\newtheorem{proposition}{Proposition}[section]
%\newtheorem{corollary}{Corollary}[section]
%\newtheorem{definition}{Definition}[section]
%\newtheorem{conj}{Conjecture}[section]
%\newtheorem{claim}{Claim}[section]
%\newtheorem{remark}{Remark}[section]
%\newtheorem{fact}{Fact}[section]
%\newtheorem{conjecture}{Conjecture}[section]
%\newtheorem{ex}{Example}[section]


\newcommand{\dis}{\displaystyle}
\newcommand{\ie}{{\em i.e., }}
\newcommand{\eg}{{\em e.g., }}
\newcommand{\cqfd}{\hfill $\diamond$}
%\newcommand{\qed}{\hfill{$\diamond$}}
\newcommand{\la}{\lambda}
\newcommand{\calA}{\mathcal A}
\newcommand{\calB}{\mathcal B}
\newcommand{\calC}{\mathcal C}
\newcommand{\calD}{\mathcal D}
\newcommand{\calE}{\mathcal E}
\newcommand{\calF}{\mathcal F}
\newcommand{\calG}{\mathcal G}
\newcommand{\calI}{\mathcal I}
\newcommand{\calL}{\mathcal L}
\newcommand{\calN}{\mathcal N}
\newcommand{\calP}{\mathcal P}
\newcommand{\calR}{\mathcal R}
\newcommand{\calS}{\mathcal S}
\newcommand{\calX}{\mathcal X}
\newcommand{\call}{\mathcal l}


\newcommand{\nin}{\in\!\!\!\!\!/} %% negation of \in
% comb(n,p) = number of combinations of p among n
\newcommand{\comb}[2]{
 \left(
 \begin{array}{c}
    #1 \\
    #2
  \end{array}
  \right)
  }

%%% indicator functions
\newcommand{\ind}[1]{1_{\{#1\}}}


 %%% norm
 \newcommand{\norm}[1]{\left\|#1\right\|}

\def\be{\begin{equation}}
\def\ee{\end{equation}}
\def\ben{\[}
\def\een{\]}
\def\bearn{\begin{eqnarray*}}
\def\eearn{\end{eqnarray*}}
\def\bear{\begin{eqnarray}}
\def\eear{\end{eqnarray}}
\def\barr{\begin{array}}
\def\earr{\end{array}}
\def\bel{\be \barr{l}}% equation array adjusted left, numbered
\def\eel{\earr\ee}
\def\beln{\ben \barr{l}}% equation array adjusted left, no number
\def\eeln{\earr\een}

% matrix with column formats in argument
% for example \bmat{cccc}
\def\bmat{\left(\begin{array}}
\def\emat{\end{array}\right)}

\newcommand{\erfc}{\mathrm{erfc}}
\newcommand{\cov}{\mathrm{cov}}
\newcommand{\var}{\mathrm{var}}
\newcommand{\sinc}{\mathrm{sinc}}
\newcommand{\diag}{\mathrm{diag}}
\newcommand{\limit}[2]{\lim_{#1 \rightarrow #2}}

\newcommand{\pd}{day$^{-1}$}% per day
\newcommand{\dert}[1]{\frac{d#1}{dt}} % derivative wr t
\newcommand{\pdert}[1]{\frac{\partial#1}{\partial t}} % partial derivative wr t
\newcommand{\rear}[1]{\stackrel{#1}{\rightarrow}} % reaction arrow
\newcommand{\drear}[2]{\begin{array}{c}#1\\ \rightleftarrows\\#2\end{array}} % double reaction arrow
\newcommand{\df}[1]{\textbf{#1}} % deterministic function
\newcommand{\rf}[1]{\textsf{\textbf{#1}}} % random quantity
\newcommand{\pbd}[1]{\parbox[t]{14cm}{#1}} % parbox approx 10 cm wide

\newcommand{\vx}{\vec{x}}
\newcommand{\vy}{\vec{y}}
\newcommand{\vz}{\vec{z}}
\newcommand{\vv}{\vec{v}}
\newcommand{\vX}{\vec{X}}
\newcommand{\vY}{\vec{Y}}
\newcommand{\vZ}{\vec{Z}}
\newcommand{\vV}{\vec{V}}

\newcommand{\mif}{\mathrm{\; if \; }}
\newcommand{\mthen} {\mathrm{\;  then \;  }}
\newcommand{\melse} {\mathrm{\;  else \;  }}
\newcommand{\mand} {\mathrm{\;  and \; }}
\newcommand{\mor} {\mathrm{\;  or \; }}
\newcommand{\mst} {\mathrm{\;  such \;  that \; }}
\newcommand{\mfa} {\mathrm{\;  for \;  all \; }}
\newcommand{\mfor} {\mathrm{\;  for \;  }}
\newcommand{\mf} {\mathrm{\;  for \; }}
\newcommand{\moth} {\mathrm{\;  otherwise \; }}

\def\I{I\!d}
\def\Reals{\mathbb{R}}
\def\Rats{\mathbb{Q}}
\def\Ints{\mathbb{Z}}
\def\Nats{\mathbb{N}}
\def\E{\mathbb{E}}
\def\P{\mathbb{P}}
\def\mpc{\otimes}
\def\mpd{\oslash}
\def\Mpc{\overline{\otimes}}
\def\Mpd{\overline{\oslash}}

% ajoute LEB 2001.01.11 to compile netcal book
\def\mod{\bmod}
\def\hdots{\cdots}


%% for compatibility, don't use any more the \pr command;
%% use \begin{preuve} instead
\newcommand{\pr}{\paragraph{Proof: }}

%% copied from VOJ
%\def\cqfd{\mbox{\rule[0pt]{1.5ex}{1.5ex}}}
%\newenvironment{preuve}{\emph{Proof: }}{\hspace*{\fill}~\QED\par\endtrivlist\unskip}
% replaced by Bremaud's proof env.
%\newenvironment{preuve}{\emph{Proof: }}{\hspace*{\fill}~\cqfd\par\endtrivlist\unskip}

\newcommand{\bracket}[1]{ \left\{\begin{array}{l} #1 \end{array} \right.}
\newcommand{\gap}{\,\,\,}



%%% don't use these commands any more- left for compatibility

\providecommand{\idb}[1]{
   \input{#1}
   }





%%% from Andrea Ridolfi's style for Bremaud's book
%% modified LEB to fit multicolumn
\newsavebox{\coloredbox}
\newenvironment{shadethm}
{\par\noindent\begin{lrbox}{\coloredbox}\begin{minipage}{\linewidth}\begin{theorem}}
{\end{theorem}\end{minipage}\end{lrbox}\noindent\psframebox[fillstyle=solid,
fillcolor=backgroundgray, linestyle=none,
framesep=10pt]{\usebox{\coloredbox}}}


%%% shaded box
\newenvironment{sh}
{\begin{lrbox}{\coloredbox}\begin{minipage}{\linewidth}}
{\end{minipage}\end{lrbox}\noindent\psframebox[fillstyle=solid,
fillcolor=backgroundgray, linestyle=none,
framesep=10pt]{\usebox{\coloredbox}}}

%%% boxed theorem
\newsavebox{\traitbox}
\newenvironment{framethm}
{\begin{lrbox}{\traitbox}\begin{minipage}[h]{0.97\linewidth}\begin{theorem}}
{\end{theorem}\end{minipage}\end{lrbox}
\framebox[\linewidth][c]{\usebox{\traitbox}}}

%%% framedbox

\newenvironment{fr}
{\begin{lrbox}{\traitbox}\begin{minipage}[h]{0.97\linewidth}}
{\end{minipage}\end{lrbox}
\framebox[\linewidth][c]{\usebox{\traitbox}}}



%%%%%%%%%%%%%%%%%%%%%%%%%%%%%%%%%%%%%%%%%%%%%%%%%
%%%%%%%%  End of Leb's usual stuff  %%%%%%%%%%%%%
%%%%%%%%                            %%%%%%%%%%%%%
%%%%%%%%%%%%%%%%%%%%%%%%%%%%%%%%%%%%%%%%%%%%%%%%%

\usepackage{amsmath, ifthen, epsfig}
\input epsf
%\usepackage{times,mathptm}
%\documentstyle{article}

%put ``oui'' in the last part of text in the line
%below if you want hidden text, including solutions, to be printed
\newcommand{\imprimeTexteCache}{non}
\newcommand{\transparents}{non}
\newcommand{\gradCourse}{oui}


% SLides Only
\newcommand{\slio}[1]{\ifthenelse{\equal{\transparents}{oui}}{#1}
}
% TextBooK Only
\newcommand{\tbko}[1]{\ifthenelse{\equal{\transparents}{non}}{#1}
}


\ifthenelse{\equal{\transparents}{oui}}{
% processing for slides
   \newcommand{\sli}{\begin{slide}}
   \newcommand{\esli}{\end{slide}}
   \newcommand{\nosli}[1]{
      \begin{note}
       #1
      \end{note}
    }
}
{
% processing for textbook
\newcommand{\sli}{}
   \newcommand{\esli}{}
   \newcommand{\nosli}[1]{
    }
}

\ifthenelse{\equal{\imprimeTexteCache}{oui}}{
   \newcommand{\sol}[1]{\emph{#1}}
  }
  {
    \newcommand{\sol}[1]{}
  }


\oddsidemargin   0.0cm
\evensidemargin  0.0cm
%\textwidth      16.5cm
\textwidth       16.5cm
\topmargin       0.00 cm
\headheight      1.00cm
\headsep         1.0cm
\textheight      22.0cm
\parskip         0.06in
\parindent       0.00cm


\newtheorem{st}{Student Tutorial}[chapter]
\newtheorem{exo}{Exercise}[section]
\newtheorem{theorem}{Theorem}[section]
\newtheorem{lemma}{Lemma}[section]
\newtheorem{proposition}{Proposition}[section]
\newtheorem{corollary}{Corollary}[section]
\newtheorem{definition}{Definition}[section]
\newtheorem{claim}{Claim}[section]
\newtheorem{remark}{Remark}[section]
\newtheorem{fact}{Fact}[section]

\newcommand{\ee}{\end{equation}}
\newcommand{\beqa}{\begin{eqnarray*}}
\newcommand{\eeqa}{\end{eqnarray*}}
\newcommand{\maketline}{\noindent \rule{16.5cm}{0.02in} \newline }
\newcommand{\makeline}{\noindent \rule{16.5cm}{0.04in}}

%\renewcommand{\thesection}{\Alph{section}}
%\renewcommand{\thepage}{\thesection-\arabic{page}}
\pagenumbering{\thesection-\arabic{page}}

\newcommand{\dis}{\displaystyle}
\newcommand{\ie}{{\em i.e., }}
\newcommand{\eg}{{\em e.g., }}
\newcommand{\cqfd}{\hfill $\Box$}
\newcommand{\qed}{\hfill{$\Box$}}
\newcommand{\la}{\lambda}
\newcommand{\calr}{\mathcal R}
\newcommand{\calS}{\mathcal S}
\newcommand{\calC}{\mathcal C}

\newcommand{\mif}{\textrm{ if }}
\newcommand{\mthen} {\textrm{ then }}
\newcommand{\melse} {\textrm{ else }}
\newcommand{\mand} {\textrm{ and }}
\newcommand{\mor} {\textrm{ or }}
\newcommand{\mst} {\textrm{ such that }}
\newcommand{\mfa} {\textrm{ for all }}

\def\Reals{\Bbb{R}}
\def\Ints{\Bbb{Z}}
\def\Nats{\Bbb{N}}
\def\E{\Bbb{E}}
\def\P{\Bbb{P}}


\newcommand{\pr}{\paragraph{Proof: }}

% write pointer to exercise \ref{coco} as
% \pointex{coco}
\newcommand{\pointex}[1]{\fbox{$\rightarrow$ Exercise \ref{#1}}}

% insert figure  ::D4:Figures:coco.eps by \insfig{coco}{0.8}
\newcommand{\insfig}[2]
{
\begin{center}
\mbox{\epsfig{file=#1.eps, width=#2\textwidth}}
\end{center}
}

\begin{document}

\pagenumbering{arabic}

% \title{
% %\maketline
% \noindent{\Large {\sc Traffic Control: }} \\
% {\Large {\sc Reserved Services}}
% %\makeline
% }
% \author{ {\sc Jean-Yves Le Boudec}  \\
% {\em ICA} \\
% {\em Ecole Polytechnique Federale de Lausanne} \\ ~\\
% %\makeline
% \date{\mbox{ }}
% }\maketitle
%
% \newpage
%
% \tableofcontents
%
% \frontmatter
%
%
%  \input{intro}


% put this line if you create a QCM
\input{QuizDB}

\mainmatter
\setcounter{chapter}{30}
% modified LEB Nov 2002, based on input by BOZ
\chapter{Congestion Control for Best Effort: Theory}
\label{d31}
In this chapter you learn
\begin{itemize}
        \item  why it is necessary to perform congestion control

        \item  the additive increase, multiplicative decrease rule

        \item  the three forms of congestion control schemes

        \item  max-min fairness, proportional fairness

\end{itemize}

This chapter gives the theoretical basis required for Congestion Control in the Internet.

\section{The objective of congestion control}

\subsection{Congestion Collapse}
\label{sec-d31-why}

Consider a network where sources may send at a rate limited
only by the source capabilities. Such a network may suffer of
congestion collapse, which we explain now on an example.

We assume that the only resource to allocate is link bit
rates. We also assume that if the offered traffic on some link
$l$ exceeds the capacity $c_{l}$ of the link, then all sources
see their traffic reduced in proportion of their offered
traffic. This assumption is approximately true if queuing is
first in first out in the network nodes, neglecting possible
effects due to traffic burstiness.

Consider first the network illustrated on Figure~\ref{D31-f1}.
Sources 1 and 2 send traffic to destination nodes D1 and D2
respectively, and are limited only by their access rates.
There are five links labeled 1 through 5 with capacities shown
on the figure. Assume sources are limited only by their first
link, without feedback from the network. Call $\lambda_i$ the
sending rate of source $i$, and $\lambda'i$ the outgoing rate.

For example, with the values given on the figures we find
$\lambda_1=100$kb/s and $\lambda_2=1000$kb/s, but only
$\lambda'_1=\lambda'_2=10$kb/s, and the total throughput is
$20$kb/s~! Source 1 can send only at 10 kb/s because it is
competing with source 2 on link 3, which sends at a high rate
on that link; however, source 2 is limited to 10 kb/s because
of link 5. If source 2 would be aware of the global situation,
and if it would cooperate, then it would send at 10 kb/s only
already on link 2, which would allow source 1 to send at 100
kb/s, without any penalty for source 2. The total throughput
of the network would then become $\theta = 110 kb/s$.

\begin{figure}[htbp]
        \insfig{D31F1}{0.7}
        \mycaption{A simple network exhibiting some inefficiency if sources
        are not limited by some feedback from the network}
        \protect\label{D31-f1}
\end{figure}

The first example has shown some inefficiency. In complex
network scenarios, this may lead to a form of instability
known as congestion collapse. To illustrate this, we use the
network illustrated on Figure~\ref{D31-f1n}. The topology is a
ring; it is commonly used in many networks, because it is a
simple way to provide some redundancy. There are $I$ nodes and
links, numbered $0, 1, ..., I-1$. Source $i$ enters node $i$,
uses links $[(i+1) \bmod I] $ and $[(i+2) \bmod I]$, and
leaves the network at node $(i+2) \bmod I$. Assume that source
$i$ sends as much as $\lambda_{i}$, without feedback from the
network. Call $\lambda'_{i}$ the rate achieved by source $i$
on link $[(i+1) \bmod I]$ and $\lambda''_{i}$ the rate
achieved on link $[(i+2) \bmod I]$. This corresponds to every
source choosing the shortest path to the destination. In the
rest of this example, we omit ``$\bmod I$" when the context is
clear. We have then:
\begin{equation}
 \left \{
  \begin{array}{l}
  \lambda'_i = \min
     \left(\lambda_i, \frac{c_i}{\lambda_i +
     \lambda'_{i-1}}\lambda_i
     \right) \\
  \lambda''_i = \min
     \left(\lambda'_i, \frac{c_{i+1}}{\lambda'_i +
     \lambda_{i+1}}\lambda'_i
     \right) \\
  \end{array}
  \right.
    \mylabel{D31-eqkkkLAUW}
\end{equation}

\begin{figure}[htbp]
        \insfig{D31F1n}{0.35}
        \mycaption{A network exhibiting congestion collapse if sources
        are not limited by some feedback from the network}
        \mylabel{D31-f1n}
\end{figure}


Applying Equation~\ref{D31-eqkkkLAUW} enables us to compute
the total throughput $\theta$. In order to obtain a closed
form solution, we further study the symmetric case, namely, we
assume that $c_i=c$ and $\lambda_i= \lambda$ for all $i$. Then
we have obviously $\lambda'_i= \lambda'$ and $\lambda''_i=
\lambda''$ for some values of $\lambda'$ and $\lambda''$ which
we compute now.

If $\lambda \leq \frac{c}{2}$ then there is no loss and
$\lambda''=\lambda'=\lambda$ and the throughput is $\theta = I
\lambda$. Else, we have, from Equation~(\ref{D31-eqkkkLAUW})
$$
 \lambda' = \frac{c \lambda}{\lambda + \lambda'}
$$
We can solve for $\lambda'$ (a polynomial equation of degree
2) and obtain
$$
\lambda'=\frac{\lambda}{2} \left( -1 + \sqrt{1 + 4
\frac{c}{\lambda}}\right)
$$
We have also from Equation~(\ref{D31-eqkkkLAUW})
$$
 \lambda'' = \frac{c \lambda'}{\lambda + \lambda'}
$$
Combining the last two equations gives
$$ \lambda'' = c -\frac{\lambda}{2}\left( \sqrt{1 + 4 \frac{c}{\lambda}} -1 \right)
$$
Using the limited development, valid for $u \rightarrow 0$
 $$\sqrt{1+u} = 1 + \frac{1}{2}u - \frac{1}{8}u^2 + o(u^2)$$we have
 $$  \lambda'' = \frac{c^2}{\lambda} + o(\frac{1}{\lambda})$$
Thus, the limit of the achieved throughput, when the offered
load goes to $+\infty$, is $0$. This is what we call
\emph{congestion collapse}.

Figure~\ref{D31-f2n} plots the throughput per source
$\lambda''$ as a function of the offered load per source
$\lambda$. It confirms that after some point, the throughput
decreases with the offered load, going to $0$ as the offered
load goes to $+\infty$.
\begin{figure}[htbp]
        \insfig{D31F2n}{0.6}
        \mycaption{Throughput per source as a function of the offered load per
        source, in Mb/s, for the network of Figure~\ref{D31-f1n}. Numbers are in Mb/s. The
        link rate is $c=20$Mb/s for all links.}
        \protect\label{D31-f2n}
\end{figure}

The previous discussion has illustrated the following fact:

\begin{fact}[Efficiency Criterion]
        In a packet network, sources should limit their sending rate by
        taking into consideration the state of the network.  Ignoring this
        may put the network into congestion collapse. One objective of
        congestion control is to avoid such inefficiencies.
\end{fact}

Congestion collapse occurs when some resources are consumed by
traffic that will be later discarded. This phenomenon did
happen in the Internet in the middle of the eighties. At that
time, there was no end-to-end congestion control in TCP/IP. As
we will see in the next section, a secondary objective is
fairness.

\subsection{Efficiency versus Fairness}

Assume that we want to maximize the network throughput, based
on the considerations of the previous section.  Consider the
network example in Figure~\ref{D31-f3}; one source sends at rate $x_0$Mb/s, one at rate $x_1$Mb/s and nine sources at rate $x_2$Mb/s each.  We assume that we implement some form
of congestion control and that there are negligible losses.
Thus, the flow, in Mb/s, on the first link $i$ is $x_{0}+x_{1}$, and on the second link it is $x_{0}+9x_{2}$.   For a
given value of $x_{0}$, maximizing the throughput
requires that $x_{1}= 10 - x_{0}$  and $9x_{2}= 10 - x_{0}$.
The total throughput, measured at the network output, is then
$x_0+x_1+9x_2=20-x_0$; it is maximum for $x_{0} = 0$~!

\begin{figure}[htbp]
        \insfig{D31F3a}{0.7}
        \mycaption{A simple network used to illustrate fairness and efficiency. }
        \protect\label{D31-f3}
\end{figure}

The example shows that maximizing network throughput as a
primary objective may lead to gross unfairness; in the worst
case, some sources may get a zero throughput, which is
probably considered unfair by these sources.

In general, the concept of efficiency is captured by the notion of \emph{Pareto Efficiency}. Consider an allocation problem; define the vector $\vec{x}$
whose $i$th coordinate is the allocation for user $i$. %Let $\calX$ be the set of all feasible allocations.
\begin{definition}[Pareto Efficiency]
       A feasible allocation of rates
$\vec{x}$ is ``Pareto-Efficient'' (also called ``Pareto-Optimal'') if and only if an increase of
any rate within the domain of feasible allocations must be at
the cost of a decrease of some other rate. Formally,
for any other feasible allocation $\vec{y}$, if $y_{s} >
x_{s}$ then there must exist some $s'$ such that $y_{s'} < x_{s'}$. \label{def-paretoeff}
\end{definition}

In general, there exist many Pareto-efficient allocations. For the example in \fref{D31-f3}, any allocation that saturates every link (i.e., such that $x_0+x_{1}=$  and $x_0+9x_{2}= 10$) is Pareto-efficient. The allocation that maximizes total throughput (and has $x_0=0$) is one of them; another Pareto efficient allocation is $x_0=0.1, x_1=9.9, x_2=1.1$; yet another one is $x_0=1, x_1=9, x_2=1$. Among all of these allocations, which one should a fair and efficient congestion control scheme choose~? This requires a formal definition of fairness, given in the next section.


%In summary, the objective of congestion control is to provide both
%efficiency and some form of fairness. We will study fairness in
%more detail now.

\section{Fairness}

\subsection{Max-Min Fairness}


In an egalitarian vision, fairness would simply mean allocating the same
share to all, while maximizing this share. In the simple case of Figure~\ref{D31-f3}, this would
mean allocating $x_0=x_1=x_2=1$. However, such an egalitarian allocation is not Pareto-efficient (because it is possible to unilaterally increase $x_1$).

A better allocation could be based on the following observation. Starting from the egalitarian allocation, we can increase the rate $x_1$ without impacting other rates, up to the value $x_1=9$. The resulting allocation $x_0=1, x_1=9, x_2=1$ \emph{is} Pareto-efficient, and still appears to be as fair as possible. It turns out that this allocation is exactly the Max-Min fair allocation, which is defined next.

%Consider again the example of Figure~\ref{D31-f3}, now with
%general values for $n_{i}$. If we follow the previous line of
%reasoning, we would allocate the fraction
%$\frac{c}{n_{0}+n_{i}}$ to each of the $n_{0}+n_{i}$ sources
%using link $i$.  This yields $x_{i}=\frac{c}{n_{0}+n_{i}}$ for
%$i \geq 1$; for $i=0$, the reasoning of the previous section
%indicates that we should allocate $x_{0}= \min_{1 \leq i \leq
%I} \frac{c}{n_{0}+n_{i}}$. For example, with $I=2$,
%$n_{0}=n_{1}=1$ and $n_{2}=9$, we would allocate $x_{0}=0.1
%c$, $x_{1}=0.5 c$ and $x_{2}=0.1 c$. This allocation however
%would not fully utilize link $1$; we could decide to increase
%the share of sources of type 1 since this can be done without
%decreasing the shares of other sources. Thus, a final
%allocation could be $x_{0}=0.1 c$, $x_{1}=0.9 c$ and
%$x_{2}=0.1 c$. We have illustrated that allocating resources
%in an equal proportion is not a good solution since some
%sources can get more that others without decreasing others'
%shares. Formally, this leads to our first definition of
%fairness called max-min fairness.

%Consider an allocation problem; define the vector $\vec{x}$
%whose $i$th coordinate is the allocation for user $i$. Let
%$\calX$ be the set of all feasible allocations.
%Recall that the $i$th coordinate of vector $\vec{x}$
%is the rate allocated to user $i$ and $\calX$ is the set of all feasible allocations.

\begin{definition}[Max-min Fairness \cite{BG}] A feasible allocation of rates
$\vec{x}$ is ``max-min fair'' if and only if an increase of
any rate within the domain of feasible allocations must be at
the cost of a decrease of some already smaller rate. Formally,
for any other feasible allocation $\vec{y}$, if $y_{s} >
x_{s}$ then there must exist some $s'$ such that $x_{s'} \leq
x_{s}$ and $y_{s'} < x_{s'}$. \label{def-mmf}
\end{definition}
The name
``max-min'' comes from the idea that it is forbidden to
decrease the share of sources that have small values, thus, in
some sense, we give priority to flows with small values.

\begin{theorem}A max-min fair allocation is Pareto-efficient.
\end{theorem}
\pr If we increase one rate in a max-min fair allocation, we must decrease some other rate, which expresses that the allocation is Pareto-efficient.
\qed

The converse is not true -- a Pareto-efficient allocation is not, in general, max-min fair. In fact, as we will see next, the max-min fair allocation is unique, whereas there are usually many Pareto-efficient allocations -- see the in \fref{D31-f3}.

\begin{theorem}A max-min fair allocation, if it exists, is unique.
\end{theorem}
\pr Assume now that
$\vec{x}$ and $\vec{y}$ are two max-min fair allocations for
the same problem, with $\vec{x} \neq \vec{y}$. Without loss of
generality, we can assume that there exists some $i$ such that
$x_{i} < y_{i}$. Consider the smallest value of $x_{i}$ that
satisfies  $x_{i} < y_{i}$, and call $i_{0}$ the corresponding
index. Thus, $x_{i_{0}} < y_{i_{0}}$ and
\begin{equation}
        \mif  x_{i} < y_{i} \mthen x_{i_{0}} \leq x_{i}
        \label{eq-preuve-uni-mmf}
\end{equation}
Now since $\vec{x}$ is max-min fair, from
Definition~\ref{def-mmf}, there exists some $j$ with
\begin{equation}
        y_{j}<x_{j} \leq x_{i_{0}}
        \label{eq-preuve-uni-mmf-2}
\end{equation}
Now $\vec{y}$ is also max-min fair, thus by the same token
there exists some $k$ such that
\begin{equation}
        x_{k}<y_{k} \leq y_{j}
        \label{eq-preuve-uni-mmf-3}
\end{equation}
Combining (\ref{eq-preuve-uni-mmf-2}) and
(\ref{eq-preuve-uni-mmf-3}), we obtain
$$
x_{k}<y_{k} \leq y_{j}<x_{j} \leq x_{i_{0}}
$$
which contradicts (\ref{eq-preuve-uni-mmf}). \qed
\qed

For some allocation problems, a max-min fair allocation might
not exist. However, for the practical examples of interest in the context of congestion control, there is always one (and only one) max-min fair allocation, as we show next. For more general cases,
see \cite{radunovic2007ufm}, where it is shown in particular that a max-min fair allocation always exists when the feasible set is convex and compact.

\paragraph{Network Model}

We use the following simplified network model in the rest of
this section. We consider a set of sources $s=1, \ldots, S$
and links $1,\ldots,L$. Let $A_{l,s}$ be the fraction of
traffic of source $s$ which flows on link $l$, and let $c_{l}$
be the capacity of link $l$. We define a network as the couple
$(\vec{x}, A)$.

A \emph{feasible allocation} of rates $x_{s}\geq 0$ is defined
by: $\sum_{s=1}^{S}A_{l,s}x_{s} \leq c_{l}$ for all $l$.

Our network model supports both multicast and load sharing.
For a given source $s$, the set of links $l$ such that
$A_{l,s} > 0$ is the path followed by the data flow with
source $s$. In the simplest case (no load sharing), $A_{l,s}
\in \{0, 1\}$; if a flow from source $s$ is equally split
between two links $l_{1}$ and $l_{2}$, then $A_{l_{1},s} =
A_{l_{2},s} =0.5$. In principle, $A_{l,s} \leq 1$, but this is
not mandatory (in some encapsulation scenarios, a flow may be
duplicated on the same link).



%It can be seen (and this is left as an exercise) that the
%allocation in the previous example is max-min fair.

We now show that there exists a
max-min fair allocation to our network model, and how to
obtain it. This will result from the key concept of
``bottleneck link''.

\begin{definition}[Bottleneck Link]
        With our network model above, we say that link $l$ is a bottleneck
        for source $s$ if and only if
        \begin{enumerate}
                \item  link $l$ is saturated: $c_{l}= \sum_{i}A_{l,i}x_{i}$

                \item  source $s$ on link $l$ has the maximum rate among all
                sources using link $l$:
                $x_{s}\geq x_{s'}$ for all $s'$ such that $A_{l,s'} > 0$.
        \end{enumerate}
\end{definition}

Intuitively, a bottleneck link for source $s$ is a link which
is limiting, for a given allocation.  In the previous
example, the first link is a bottleneck for blue flow with rate $x_1$ and for the red flow with rate $x_0$; the second link is a bottleneck for each of the nine green flows with rates $x_2$ and for the red flow with rate $x_0$.

\begin{theorem}
        A feasible allocation of rates $\vec{x}$ is max-min fair if and
        only if every source has a bottleneck link.
        \label{theo-bn}
\end{theorem}
\pr \textbf{Part 1.}  Assume that every source has a bottleneck
link. Consider a source $s$ for which we can increase the rate
$x_{s}$ while keeping the allocation feasible. Let $l$ be a
bottleneck link for $s$. Since $l$ is saturated, it is necessary
to decrease $x_{s'}$ for some $s'$ such that $A_{l,s'} >0$. We
assumed that we can increase the rate of $s$: thus there must
exist some $s' \neq s$ that shares the bottleneck link $l$. But
for all such $s'$, we have $x_{s}\geq x_{s'}$, thus we are forced
to decrease $x_{s'}$ for some $s'$ such that $x_{s}\geq x_{s'}$:
this shows that the allocation is max-min fair.

\textbf{Part 2.} Conversely, assume that the allocation is max-min
fair.  For any source $s$, we need to find a bottleneck link.  We
proceed by contradiction.  Assume there exists a source $s$ with
no bottleneck link.  Call $L_{1}$ the set of saturated links used
by source $s$, namely, $L_{1} = \{l \mst c_{l}=
\sum_{i}A_{l,i}x_{i} \mand A_{l,s} > 0 \}$.  Similarly, call
$L_{2}$ the set of non-saturated links used by source $s$. Thus a
link is either in $L_{1}$ or $L_{2}$, or is not used by $s$.
Assume first that $L_1$ is non-empty.

\begin{figure}[h]
        \insfig{D31F6}{0.5}
        \mycaption{A network example showing one multicast source}
        \protect\label{fig-sigma}
\end{figure}

By our assumption, for all $l \in L_{1}$ , there exists some
$s'$ such that $A_{l,s'} >0$ and $x_{s'} > x_{s}$. Thus we can
build a mapping $\sigma$ from $L_{1}$ into the set of sources
$\{1,\ldots,S \}$ such that $A_{l,\sigma(l)} >0$ and
$x_{\sigma(l)} > x_{s}$ (see Figure~\ref{fig-sigma} for an
illustration). Now we will show that we can increase the rate
$x_{s}$ in a way that contradicts the max-min fairness
assumption. We want to increase $x_{s}$ by some value
$\delta$, at the expense of decreasing $x_{s'}$ by some other
values $\delta_{s'}$, for all $s'$ that are equal to some
$\sigma(l')$. We want the modified allocation to be feasible;
to that end, it is sufficient to have:

\begin{eqnarray}
         A_{l,s} \delta \leq A_{l, \sigma(l)} \delta_{\sigma(l)} \; \mfa l \in
        L_{1}
        \label{eq-mmf-1}  \\
        A_{l,s} \delta  \leq c_{l} - \sum_{i}A_{l,i}x_{i} \; \mfa l \in L_{2}
        \label{eq-mmf-2}  \\
        \delta_{\sigma(l)} \leq x_{\sigma(l)} \; \mfa l \in L_{1}
        \label{eq-mmf-3}
\end{eqnarray}
Equation~(\ref{eq-mmf-1}) expresses that the increase of flow
due to source $s$ on a saturated link $l$ is at least
compensated by the decrease of flow due to source $\sigma(l)$.
Equation~(\ref{eq-mmf-2}) expresses that the increase of flow
due to source $s$ on a non-saturated link $l$ does not exceed
the available capacity. Finally, equation~(\ref{eq-mmf-3})
states that rates must be non-negative.

This leads to the following choice.
\begin{equation}
        \delta = \min_{l \in L_{1}}
        \{  \frac{x_{\sigma(l)} A_{l, \sigma(l)}}{A_{l,s}}\}
        \wedge
        \min_{l \in L_{2}}
        \{  \frac{c_{l}-\sum_{i}A_{l,i}x_{i}}{A_{l,s}}\}
        \label{eq-mmf-4}
\end{equation}
which ensures that Equation~(\ref{eq-mmf-2}) is satisfied and
that $\delta >0$.

In order to satisfy Equations~(\ref{eq-mmf-1}) and
(\ref{eq-mmf-3}) we need to compute the values of
$\delta_{\sigma(l)}$ for all $l$ in $L_{1}$. Here we need to
be careful with the fact that the same source $s'$ may be
equal to $\sigma(l)$ for more than one $l$. We define
$\delta(s')$ by

\begin{eqnarray}
         \delta(s') & = 0  \textrm{ if there is no } l \mst s'=\sigma(l)
        \label{eq-mmf-50}  \\
        \delta(s') & =
                \max_{\{l \mst \sigma(l) = s'\}}
                \{ \frac{\delta A_{l,s}}{A_{l, \sigma(l)}} \}
                \textrm{ otherwise }
        \label{eq-mmf-5}
\end{eqnarray}

This definition ensures that Equation~(\ref{eq-mmf-1}) is
satisfied. We now examine Equation~(\ref{eq-mmf-3}). Consider
some $s'$ for which there exists an $l$ with $\sigma(l)=s$,
and call $l_{0}$ the value which achieves the maximum in
(\ref{eq-mmf-5}), namely:
\begin{equation}
        \delta(s') = \frac{\delta A_{l_{0},s}}{A_{l_{0}, s'}}
        \label{eq-mmf-6}
\end{equation}
From the definition of $\delta$ in (\ref{eq-mmf-4}), we have
$$
\delta \leq \frac{x_{\sigma(l_{0})}A_{l_{0}, \sigma(l_{0})} }{
A_{l_{0},s}}
=
\frac{x_{s'}A_{l_{0}, s'} }{ A_{l_{0},s}}
$$
Combined with (\ref{eq-mmf-6}), this shows that
Equation~(\ref{eq-mmf-3}) holds. In summary, we have shown
that we can increase $x_{s}$ at the expense of decreasing the
rates for only those sources $s'$ such that $s'=\sigma(l)$ for
some $l$. Such sources have a rate higher than $x_{s}$, which
shows that the allocation $\vec{x}$ is not max-min fair and
contradicts our hypothesis.

It remains to examine the case where $L_1$ is empty. The
reasoning is the same, we can increase $x_s$ without
decreasing any other source, and we also have a contradiction.
\qed

\paragraph{The Water-Filling Algorithm}

The previous theorem is particularly useful in deriving a
practical method for obtaining a max-min fair allocation,
called ``progressive filling'' or ``water filling''.  The idea is as follows.  You
start with all rates equal to 0 and grow all rates together at
the same pace, until one or several link capacity limits are
hit.  The rates for the sources that use these links are not
increased any more, and you continue increasing the rates for
other sources.  All the sources that are stopped have a
bottleneck link.  This is because they use a saturated link,
and all other sources using the saturated link are stopped at
the same time, or were stopped before, thus have a smaller or
equal rate.  The algorithm continues until it is not possible
to increase.  The algorithm terminates because $L$ and $S$ are
finite.  Lastly, when the algorithm terminates, all sources
have been stopped at some time and thus have a bottleneck
link.  By application of Theorem~\ref{theo-bn}, the allocation
is max-min fair.

\paragraph{Example}
Let us apply the water-filling algorithm to \fref{D31-f3}.

We first let $x_{0}=x_1=x_2=t$ and increase $t$ until we hit a limit, i.e. we maximize $t$ subject to the constraints imposed by the link capacities. The constraints are
$$
x_0+x_1\leq 10\mand x_1 +9x_2\leq 10$$
which gives
$$
2t\leq 10 \mand 10t\leq 10$$
The maximum is for $t=1$Mb/s and at this value the second constraint is hit, i.e., the second link is a bottleneck link. The sources using this link are the sources of type 0 and 2, therefore we let
$$
x_0=1\mand x_2=1
$$
and these values are final.

In a second round we increase the rates of the other sources, namely we increase $x_1$ until a capacity limit is hit. There is only one constraint left, that of the first link, which is now expressed as
$$1 + x_1\leq 10
$$
The maximal value of $x_1$ is $9$. With this, all sources are stopped and the algorithm terminates. The max-min fair allocation is thus $x=1, x_1=9, x_2=1$.

We see that all sources of type $0$ and $2$ obtain
the same rate. In some sense, max-min fairness ignores the
fact that sources of type $0$ use more network resources than
those of type $2$.

\begin{theorem}For the network defined above,
 with fixed routing parameters $A_{l,s}$, there exists a unique
        max-min fair allocation.  It can be obtained by the water-filling algorithm.
\end{theorem}
\pr We have already proven uniqueness.  Existence follows from the water-filling algorithm. \qed


% Pb avec fairness (exemple de Jim Roberts) avec shortest packet first

The notion of max-min fairness can be generalized by
using weights in the definition \cite{BG, MR98}. 

\subsection{Proportional Fairness}

The previous definition of fairness puts emphasis on
maintaining high values for the smallest rates. As shown in
the previous example, this may be at the expense of some
network inefficiency. An alternative definition of fairness
has been proposed in the context of game theory \cite{MMD91}.

\begin{definition}[Proportional Fairness]
An allocation of rates $\vec{x}$ is ``proportionally fair'' if and only if all rates are positive and, for any other feasible allocation $\vec{y}$, we
have:
$$
\sum_{s=1}^S \frac{y_{s}-x_{s}}{x_{s}} \; \leq 0
$$
        \label{def-pf}
\end{definition}

In other words, any change in the allocation must have a
negative average change.



Let us consider for example the
parking lot scenario in \fref{D31-f3p}. Is the
max-min fair allocation proportionally fair~?


\begin{figure}[htbp]
        \insfig{D31F3p}{0.7}
        \mycaption{A simple network used to illustrate proportional fairness (the``parking
        lot'' scenario)}
        \protect\label{D31-f3p}
\end{figure}



To get the answer, observe that the max-min fair
allocation is obtained easily with water-filling and is $x_{s}=c/2$ for $s=0,1..4$. Consider a new allocation
resulting from a decrease of $x_{0}$ equal to $\delta$:
\begin{eqnarray*}
         y_{0}  &= \frac{c}{2} - \delta \\
        y_{s} &= \frac{c}{2} + \delta \; s=1,\ldots,4
\end{eqnarray*}
For $\delta<\frac{c}{2}$, the new allocation $\vec{y}$ is
feasible. The average rate of change is
$$
\left( \sum_{s=1}^4 \frac{2 \delta }{ c} \right)
 -  \frac{2 \delta }{ c} =
\frac{6\delta }{c}
$$
which is positive.  Thus the max-min fair
allocation for this example is not proportionally fair.  In this example, we see that a decrease in rate for
sources of type $0$ is less important than the corresponding
increase which is made possible for the other sources, because
the increase is multiplied by the number of sources.
Informally, we say that proportional fairness takes into
consideration the usage of network resources.

\begin{theorem}A proportionally fair allocation is Pareto-efficient.
\end{theorem}
\pr Let $\vec{x}$ be a proportionally fair allocation and let $\vy$ be some feasible allocation such that $y_i>x_i$ for some $i$. The rate of change is
$$
\frac{y_i-x_i}{x_i}+\sum_{s=1...S, s\neq i}\frac{y_s-x_s}{x_s}\leq 0
$$ because $\vec{x}$ is proportionally fair. The first term is positive, therefore at least one of the other terms, say $\frac{y_s-x_s}{x_s}$ must be negative, and thus $y_s<x_s$. This shows that $\vec{x}$ is Pareto-efficient.
\qed


Now we derive a practical result which can be used to compute
a proportionally fair allocation.  To that end, we interpret
the average rate of change as $\nabla J_{\vec{x}}\cdot
(\vec{y}-\vec{x})$, with
$$
J(\vec{x})= \sum_{s} \ln(x_{s})
$$
Thus, intuitively, a proportionally fair allocation should
maximize $J$.
\begin{theorem}
There exists one unique proportionally fair allocation. It is
obtained by maximizing $J(\vec{x})= \sum_{s} \ln(x_{s})$ over
the set of feasible allocations.
        \label{theo-pf}
\end{theorem}
\pr 1. We first prove that the maximization problem has a unique
solution.  Function $J$ is concave, as a sum of concave
functions. The feasible set is convex, as intersection of
convex sets, thus any local maximum of $J$ is an absolute
maximum.  Now $J$ is strictly concave, which means that
$$\mif 0< \alpha< 1 \mthen J(\alpha \vec{x}
+ (1-\alpha) \vec{y}) > \alpha J(\vec{x}) + (1-\alpha)
J(\vec{y})
$$
This can be proven by studying the second derivative
of the restriction of $J$ to any linear segment.  Now
a strictly concave function has at most one maximum on
a convex set. %(Chapter~\ref{D4-fluid11}).

Now $J$ is continuous if we allow $\log(0)=-\infty$ and the
set of feasible allocations is compact (because it is a
closed, bounded subset of $\Reals^S$). Thus $J$ has at least
one maximum over the set of feasible allocations.
Combining all the arguments together proves that $J$ has
exactly one maximum over the set of feasible allocations, and
that any local maximum is also exactly the global maximum.

2. For any $\vec{\delta}$ such that $\vec{x}+ \vec{\delta}$
is feasible,

$$
J(\vec{x}+ \vec{\delta})-J(\vec{x})=\nabla J_{\vec{x}}\cdot
\vec{\delta} + \frac{1}{2}   \vec{\delta}^T \nabla^2
J_{\vec{x}} \vec{\delta} +  o(||\vec{\delta}||^2)
$$
Now by the strict concavity, $\nabla^2 J_{\vec{x}}$ is
definite negative thus, for
$||\vec{\delta}||$ small enough:
$$\frac{1}{2}  \vec{\delta}^T \nabla^2
J_{\vec{x}} \vec{\delta} +  o(||\vec{\delta}||^2) \; <0 $$
and therefore
$$
J(\vec{x}+ \vec{\delta})-J(\vec{x})\leq \nabla J_{\vec{x}}\cdot
\vec{\delta}
$$


Now assume that $\vec{x}$ is a proportionally fair allocation.
This means that

$$\nabla(J)_{\vec{x}}\cdot \vec{\delta} \leq 0$$
and thus $J$ has a local maximum at $\vec{x}$, thus also a
global maximum. This also shows the uniqueness of a
proportionally fair allocation.


3. Conversely, assume that $J$ has a global maximum at $\vec{x}$.
Observe that our network model assumes that there exists some positive and feasible allocation, therefore, the maximum of $J$ is not $-\infty$, therefore the maximiser $\vec{x}$ satisfies $x_s>0$ for all $s$. Let $\vec{y}$ be some other feasible allocation and call $D$ the
average rate of change. We also have:
$$ D = \nabla(J)_{\vec{x}}\cdot (\vec{y}-\vec{x}) $$
Since the feasible set is convex, the segment $[\vec{x},
\vec{y}]$ is entirely feasible, and thus $J(\vec{x} + t (\vec{y}-\vec{x})) \leq  J(\vec{x} )$ for $t\in [0;1]$. Observe that
$$D = \lim_{t \rightarrow 0^+}
\frac{J(\vec{x} + t (\vec{y}-\vec{x})) - J(\vec{x} ) }{ t}
$$
thus $D \leq 0$. \qed
\paragraph{Example}
Let us apply Theorem~\ref{theo-pf} to the parking lot
scenario in \fref{D31-f3p}. For any choice of $x_{0}$, we should set $x_{i}$
such that
$$
x_{0} + x_{i} = c,  i=1,\ldots,4
$$
otherwise we could increase $x_{i}$ without affecting other
values, and thus increase function $J$. The value of $x_{0}$
is found by maximizing $f(x_{0})$, defined by
$$
f(x_{0}) =  \ln(x_{0}) + \sum_{i=1}^4  \ln(c
-x_{0})
$$
over the set $0 \leq x_{0} \leq c$. The
derivative of $f$ is
$$
f'(x_{0}) =\frac{1}{x_{0}} - \frac{4}{ c- x_{0}}
$$
After some algebra, we find that the maximum is for
$$
x_{0}= \frac{c}{5}
$$
and
$$
x_{i}= \frac{4c}{5}\mfor i=1...4
$$
%For example, if $n_{i}=1$ for all $i=0,\ldots,4$, we obtain:
%\begin{eqnarray*}
%         x_{0}&= \frac{c }{4 +1} \\
%        x_{i} &= \frac{cI}{4+1}
%\end{eqnarray*}
Compare with max-min fairness, where, in that case, the
allocation is $\frac{c}{2}$ for all rates.  We see that
sources of type $0$ get a smaller rate, since they use more
network resources.

The concept of proportional fairness can easily extended to
\emph{weighted} proportional fairness, where the allocation
maximizes a weighted sum of logarithms \cite{KMT97}.


\subsection{Utility Approach to Fairness}
Proportional fairness is an example of a more general fairness
concept, called the ``utility'' approach, which is defined as
follows. Every source $s$ has a utility function $u_{s}$ where
$u_{s}(x_{s})$ indicates the value to source $s$ of having
rate $x_{s}$. Every link $l$ (or network resource in general)
has a cost function $g_{l}$,  where $g_{l}(f)$ indicates the
cost to the network of supporting an amount of flow $f$ on
link $l$. Then, a ``utility fair'' allocation of rates is an
allocation which maximizes $H(\vec{x})$, defined by
$$H(\vec{x}) = \sum_{s=1}^S u_{s}(x_{s}) - \sum_{l=1}^L g_{l}(f_{l})
$$
with $f_{l}=\sum_{s=1}^S A_{l,s}x_{s}$, over the set of
feasible allocations.

Proportional fairness corresponds to $u_{s}= \ln$ for all $s$,
and $g_{l}(f)=0$ for $f< c_{l}$, $g_{l}(f)=+ \infty$ for
$f\geq c_{l}$. Rate proportional fairness corresponds to
$u_{s}(x_{s})= w_{s}\ln(x_{s})$ and the same choice of
$g_{l}$.

Computing utility fairness requires solving constrained
optimization problems; a reference is \cite{WHI}.

\subsection{Max Min Fairness as a limiting case of Utility Fairness}
We show in this section that max-min fairness is the limiting case
of a utility fairness. Indeed, we can define a set of concave,
increasing utility functions $f_m$, indexed by $m \in \Reals^+$
such that the allocation $\vec{x^m}$ which maximizes
$\sum_{i=1}^{I} f_m(x_i)$ over the set of feasible allocations
converges towards the max-min fair allocation.

The proof is a correct version of the ideas expressed in
\cite{mo-98-a} and later versions of it.

Let $f_m$ be a family of increasing, differentiable and concave
functions defined on $\Reals^+$. Assume that, for any fixed
numbers $x$ and $\delta
>0$,
\begin{equation}\mylabel{eq-deffm}
  \lim_{m \rightarrow + \infty} \frac{f'_m(x+\delta)}{f'_m(x)}=0
\end{equation}
% modified Nov 2002, based on input by BOZ
%\begin{equation}\mylabel{eq-deffm}
%  \lim_{m \rightarrow + \infty} \frac{f_m(x)}{f_m(x+\delta)}=0
%\end{equation}

The assumption on $f_m$ is satisfied if $f_m$ is defined by
 $$f_m(x) = c - g(x)^m
 $$
where $c$ is a constant and $g$ is a differentiable, decreasing,
convex and positive function. For example, consider
 $f_m(x) = 1 - \frac{1}{x^m}$.

Define $\vec{x}^m$ the unique allocation which maximizes
$\sum_{i=1}^{I} f_m(x_i)$ over the set of feasible allocations.
Thus  $\vec{x}^m$ is the fair allocation of rates, in the sense of
utility fairness defined by $f_m$. The existence of this unique
allocation follows from the same reasoning as for
Theorem~\ref{theo-pf}. Our result if the following theorem.

\begin{theorem}
\mylabel{theo-approx} The set of utility-fair allocation
$\vec{x}^m$ converges towards the max-min fair allocation as $m$
tends to $+\infty$.
\end{theorem}
The rest of this section is devoted to the proof of the theorem.
We start with a definition and a lemma.

\begin{definition}
We say that a vector $\vec{z}$ is an accumulation point for a set
of vectors $\vec{x}_m$ indexed by $m\in \Reals^+$ if there exists
a sequence $m_n, n\in \Nats$, with $\lim_{n \rightarrow + \infty}
m_n = + \infty$ and $\lim_{n \rightarrow + \infty} \vec{x}^{m_n} =
\vec{z}$.
\end{definition}

\begin{lemma}
If $\vec{x}^*$ is an accumulation point for the set of vectors
$\vec{x}^{m}$, then $\vec{x}^*$ is the max-min fair allocation.
\end{lemma}
\paragraph{Proof of Lemma: }
We give the proof for the case where $A_{l,i} \in \{0, 1\}$; in
the general case the proof is similar We proceed by contradiction.
Assume that $\vec{x^*}$ is not max-min fair. Then, from
Theorem~\ref{theo-bn}, there is some source $i$ which has no
bottleneck. Call $L_1$ the set (possibly empty) of saturated links
used by $i$, and $L_2$ the set of other links used by $i$. For
every link $l \in L_1$, we can find some source $\sigma(l)$ which
uses $l$ and such that $x^*_{\sigma(l)} > x^*_i$. Define $\delta$
by
 $$ \delta = \frac{1}{5} \min \left\{
 \min_{l \in L1} ( x^*_{\sigma(l)} - x^*_i), \;
 \min_{l \in L2} ( c_l -(A\vec{x}^*)_{l})
 \right\}
 $$
The term $c_l - (A\vec{x}^*)_{l}$ in the last part of the
expression is the unused capacity on link $l$. We also use the
convention that the minimum of an empty set is $+ \infty$.

%Now note that, for $n$ large enough, (say $n \geq n_0$), the
%allocation $\vec{x}^{m_n}$ is such that the set of saturated links
%used by source $i$ is included in $L_1$. Indeed, if this would not
%be true, there would be some link $l$ not in $L_1$ and an infinite
%number of values of $n$ for which link $l$ is saturated for
%allocation $\vec{x}^{m_n}$. Thus, by continuity of the mapping
%$\vec{x}^m \rightarrow (A\vec{x}^*)_l$, link $l$ is also saturated
%for allocation $\vec{x}^*$, which is a contradiction.

From the convergence of $\vec{x}^{m_n}$ to $\vec{x}^*$, we can
find some $n_0$ such that, for all $n \geq n_0$ and for all $j$ we
have
\begin{equation}\mylabel{eq-defdesx}
   x^*_j - \frac{\delta}{I} \leq x^{m_n}_j \leq  x^*_j + \frac{\delta}{I}
\end{equation}
where $I$ is the number of sources. Now we construct, for all $n
\geq n_0$, an allocation $\vec{y}^n$ which will lead to a
contradiction. Define
 $$
  \left\{
  \begin{array}{rcl}
   y^n_i &=& x^{m_n}_i + \delta \\
   y^n_{\sigma(l)} &=&x^{m_n}_{\sigma(l)} - \delta \mbox{ for } l \in L_1\\
   y^n_j&=&x^{m_n}_j  \; \; \mbox{ otherwise}
  \end{array}
  \right.
 $$

We prove first that the allocation $\vec{y}^n$ is feasible.
Firstly, we show that the rates are non-negative. From
Equation~(\ref{eq-defdesx}), we have, for all $l \in L_1$
 $$
 x^{m_n}_{\sigma(l)} \geq x^*_{\sigma(l)} - \frac{\delta}{I}
 \geq x^*_{\sigma(l)} - \delta
 $$
 thus, from the definition of $\delta$:
\begin{equation}\mylabel{eq-3del}
 y^n_{\sigma(l)} \geq x^*_{\sigma(l)} - 2 \delta \geq  x^*_i + 3 \delta
\end{equation}
This shows that $y^n_j \geq 0$ for all $j$.

Secondly, we show that the total flow on every link is bounded by
the capacity of the link. We need to consider only the case of
links $l \in (L_1 \cup L_2)$. If $l \in L_1$ then
$$(A\vec{y}^n)_{l} \leq (A\vec{x}^{m_n})_{l}+ \delta - \delta = (A\vec{x}^{m_n})_{l}$$
thus the condition is satisfied. Assume now that $l \in L_2$. We
have then
 $$
(A\vec{y}^n)_{l} = (A\vec{x}^{m_n})_{l} + \delta $$
 Now, from Equation~(\ref{eq-defdesx})
 $$
 (A\vec{x}^{m_n})_{l} \leq (A\vec{x}^{*})_{l} + I \frac{\delta}{I}
 =(A\vec{x}^{*})_{l} + \delta
 $$
 Thus, from the definition of $\delta$:
 $$
 (A\vec{y}^n)_{l} \leq (A\vec{x}^{*})_{l} + 2 \delta \leq c
 $$
which ends the proof that $\vec{y}^n$ is a feasible allocation.

Now we show that, for $n$ large enough, we have a contradiction
with the optimality of $\vec{x}^{m_n}$. Consider the expression
$A$ defined by
$$A =
 \sum_j \left(f_{m_n}(y^{n}_j) - f_{m_n}(x^{m_n}_j)\right)
 $$
 From the optimality of $\vec{x}^{m_n}$,  we have: $A \leq 0$.

Now
$$A = f_{m_n}(x^{m_n}_i + \delta) - f_{m_n}(x^{m_n}_i ) +
\sum_{l \in L_1} \left(f_{m_n}(x^{m_n}_{\sigma(l)}-\delta) -
f_{m_n}(x^{m_n}_{\sigma(l)})\right)
$$

From the theorem of intermediate values, there exist numbers
$c^{n}_i$ such that
$$
\bracket{
 x^{m_n}_i \leq c^{n}_i \leq x^{m_n}_i + \delta \\
 f_{m_n}(x^{m_n}_i + \delta) - f_{m_n}(x^{m_n}_i ) =
 f'_{m_n}(c^{n}_i)\delta
 }
$$
where $f'_{m_n}$ is the right-derivative of $f_{m_n}$. Combining
with Equation~(\ref{eq-defdesx}) we find
\begin{equation}\mylabel{eq-ci}
c^{n}_i \leq x^*_i +  \frac{\delta}{I} + \delta \leq x^*_i + 2
\delta
\end{equation}

Similarly, there exist some numbers $c^{n}_{\sigma(l)}$ such that
$$
\bracket{
 x^{m_n}_{\sigma(l)}  -  \delta \leq c^{n}_{\sigma(l)}+ \delta \leq x^{m_n}_{\sigma(l)}+
 \delta \\
 f_{m_n}(x^{m_n}_{\sigma(l)} - \delta) - f_{m_n}(x^{m_n}_{\sigma(l)} ) =
 - f'_{m_n}(c^{n}_{\sigma(l)})\delta
 }
$$
and combining with Equation~(\ref{eq-defdesx}) we find also
\begin{equation}\mylabel{eq-cj}
c^{n}_{\sigma(l)} \geq x^*_i +  3 \delta
\end{equation}
Thus
$$
A = \delta \left(
 f'_{m_n}(c^{n}_i)
 -
 \sum_{l \in L_1} f'_{m_n}(c^{n}_{\sigma(l)})
  \right)
$$
Now $f'_{m_n}$ is wide-sense decreasing ($f_{m_n}$ is concave)
thus, combining with Equations~(\ref{eq-ci}) and~(\ref{eq-cj}):
$$A
\geq \delta \left(f'_{m_n}(x^*_i + 2 \delta) - M  f'_{m_n}(x^*_i + 3 \delta)\right) = \delta
f'_{m_n}(x^*_i + 2 \delta) \left( 1- M \frac{f'_{m_n}(x^*_i + 3 \delta)}{f'_{m_n}(x^*_i + 2
\delta)} \right)
$$
where $M$ is the cardinal of set $L_1$. Now from
Equation~(\ref{eq-deffm}), the last term in the above equation
tends to $1$ as $n$ tends to infinity. Now $f'_{m_n} >0$ from our
assumptions thus, for $n$ large enough, we have $A >0$, which is
the required contradiction.

\paragraph{Proof of Theorem: }

The set of vectors $\vec{x}^m$ is in a compact (= closed +
bounded) subset of $\Reals^I$; thus, it has at least one
accumulation point. From the uniqueness of the max-min fair
vector, it follows that the set of vectors $\vec{x}^m$ has a
unique accumulation point, which is equivalent to saying that
$$
\lim_{m \rightarrow + \infty} \vec{x}^m = \vec{x}^*
$$
\qed

\section{Different forms of congestion control}

We can consider that there are three families of solutions for
congestion control.
\begin{description}
        \item[Rate Based: ]  Sources know an explicit rate at which they can
        send. The rate may be given to the source during a negotiation
        phase;
        this is the case with ATM or RSVP. In such cases, we have a network
        with reservation. Alternatively, the rate may be imposed
        dynamically to the source by the network; this is the case
        for the ABR class of ATM. In such cases we have a best effort
        service (since the source cannot be sure of how long a given rate
        will remain valid), with explicit rate. In the example of the
        previous section, source 1 would obtain a rate not exceeding 10 kb/s.

        \item[Hop by Hop: ] A source needs some feedback from the next hop in
        order to send any amount of data. The next hop also must obtain some
        feedback from the following hop and so on. The feedback may be positive
        (credits) or negative (backpressure). In the simplest form, the
        protocol is stop and go. In the example of the
        previous section, node X would be prevented by node Y from sending
        source 2 traffic at a rate higher than 10kb/s; source 2 would then
        be throttled by node X. Hop by hop control is used
        with full duplex Ethernets using 802.3x frames called ``Pause''
        frames.

        \item[End-to-end:] A source continuously obtains feedback from all
        downstream nodes it uses.  The feedback is piggybacked in packets
        returning towards the source, or it may simply be the detection of
        a missing packet.  Sources react to negative feedback by reducing
        their rate, and to positive feedback by increasing it.  The
        difference with hop-by-hop control is that the intermediate nodes
        take no action on the feedback; all reactions to feedback are left
        to the sources.  In the example of the previous section, node Y
        would mark some negative information in the flow of source 2 which would be
        echoed to the source by destination D2; the source would then react
        by reducing the rate, until it reaches 10 kb/s, after which there
        would be no negative feedback. Alternatively, source 2 could detect
        that a large fraction of packets is lost, reduce its rate, until
        there is little loss. In broad terms, this is the method invented for
        Decnet, which is now used after some modification in the Internet.

\end{description}

        In the following section we focus on end-to-end control.

\section{Max-min fairness with fair queuing}
Consider a network implementing fair queuing per flow. This is
equivalent to generalized processor sharing (GPS) with equal
weights for all flows. With fair queuing, all flows that have data
in the node receive an equal amount of service per time slot.

Assume all sources adjust their sending rates such that there is
no loss in the network. This can be implemented by using a sliding
window, with a window size which is large enough, namely, the
window size should be as large as the smallest rate that can be
allocated by the source, multiplied by the round trip time.
Initially, a source starts sending with a large rate; but in order
to sustain the rate, it has to receive acknowledgements. Thus,
finally, the rate of the source is limited to the smallest rate
allocated by the network nodes. At the node that allocates this
minimum rate, the source has a rate which is as high as the rate
of any other sources using this node. Following this line of
thoughts, the alert reader can convince herself that this node is
a bottleneck link for the source, thus the allocation is max-min
fair. The detailed proof is complex and is given in
\cite{Hahne91}.

\begin{proposition}
A large sliding window at the sources plus fair queuing in the
nodes implements max-min fairness.
\end{proposition}


Fixed window plus fair queuing is a possible solution to
congestion control. It is implemented in some (old) proprietary
networks such as IBM SNA.

Assume now that we relax the assumption that sources are using a
window. Assume thus that sources send at their maximum rate,
without feedback from the network, but that the network implements
fair queuing per flow. Can congestion collapse occur as in
Section~\ref{sec-d31-why}?

Let us start first with a simple scenario with only one network
link of capacity $C$. Assume there are $N$ sources, and the only
constraint on source $i$ is a rate limit $r_i$. Thus, sources for
which $r_i \leq \frac{C}{N}$ have a throughput equal to $r_i$ and
thus experience no loss. If some sources have a rate $r_i <
\frac{C}{N}$, then there is some extra capacity which can will be
used to serve the backlog of other sources. This distribution will
follow the algorithm of progressive filling, thus the capacity
will be distributed according to max-min fairness, \emph{in the
case of a single node}.

In the multiple node case, things are not as nice, as we show now on
the following example. The network is as on Figure~\ref{D31-f1}.
There is 1 source at S1, which sends traffic to D1. There are 10
sources at S2, which all send traffic to D2. Nodes X and Y implement
fair queuing per flow. Capacities are :
\begin{itemize}
  \item 11 Mb/s for link X-Y
  \item 10 Mb/s for link Y-D1
  \item 1 Mb/s for link Y-D1
\end{itemize}

Every source receives 1 Mb/s at node X. The S1- source keeps its
share at Y. Every S2 source experiences 90\% loss rate at Y and
has a final rate of 0.1 Mb/s.

Thus finally, the useful rate for every source is
  \begin{itemize}
  \item 1 Mb/s for a source at S1
 \item 0.1 Mb/s for a source at S2
  \end{itemize}

The max-min fair share is
 \begin{itemize}
  \item 10 Mb/s for a source at  S1
 \item 0.1 Mb/s for a source at S2
\end{itemize}

Thus fair queuing alone is not sufficient to obtain max-min
fairness. However, we can say that if all nodes in a network
implement fair queuing per flow, the throughput for any source $s$
is at least $\min_{l \mst l \in s }\frac{C_l}{N_l}$, where $C_l$ is
the capacity of link $l$, and $N_l$ is the number of active sources
at node $l$. This implies that congestion collapse as described
earlier is avoided.


\section{Additive increase,
Multiplicative decrease and Slow-Start}

End-to-end congestion control in packet networks is based on binary
feedback and the adaptation mechanisms of additive increase,
multiplicative decrease and slow start. We describe here a
motivation for this approach. It comes from the following modeling,
from \cite{CJ89}. Unlike the fixed window mentioned earlier, binary
feedback does not require fair queuing, but can be implemented with
FIFO queues. This is why it is the preferred solution today.

\subsection{Additive Increase, Multiplicative Decrease}
Assume $I$ sources, labeled $i=1,\ldots, I$ send data at
a time dependent rate $x_{i}(t)$, into a network constituted of one
buffered link, of rate $c$.  We assume that time is discrete ($t \in
\Nats$), and that the feedback cycle lasts exactly one time unit.
During one time cycle of duration 1 unit of time, the source rates
are constant, and the network generates a binary feedback signal
$y(t) \in \{0, 1 \}$, sent to all sources. Sources react to the
feedback by increasing the rate if $y(t)=0$, and decreasing if
$y(t)=1$. The exact method by which this is done is called the
adaptation algorithm. We further assume that the feedback is defined
by
$$
y(t) = [ \mif (\sum_{i=1}^I x_{i}(t) \leq c ) \mthen  0 \melse 1 ]
$$
The value $c$ is the target rate which we wish the system not to
exceed. At the same time we wish that the total traffic be as
close to $c$ as possible.

We are looking for a linear adaptation algorithm, namely, there
must exist constants $u_{0}, u_{1}$ and $v_{0}, v_{1}$ such that

\begin{equation}
        x_{i}(t+1) = u_{y(t)} x_{i}(t) + v_{y(t)}
        \label{eq-theo-addmul1}
\end{equation}

We want the adaptation algorithm to converge towards a fair
allocation. In this simple case, there is one single bottleneck
and all fairness criteria are equivalent. At equilibrium, we
should have $x_{i}= \frac{c}{I}$. However, a simple adaptation
algorithm as described above cannot converge, but in contrast,
oscillates around the ideal equilibrium.

We now derive a number of necessary conditions. First, we would
like the rates to increase when the feedback is $0$, and to
decrease otherwise. Call $f(t)=\sum_{i=1}^I x_{i}(t)$. We have

\begin{equation}
        f(t+1) = u_{y(t)} f(t) + v_{y(t)}
        \label{eq-jain-48}
\end{equation}
Now our condition implies that, for all $f \geq 0$:
$$
u_{0}f + v_{0} > f
$$
and
$$
u_{1}f + v_{1} < f
$$
This gives the following necessary conditions
 \begin{equation}
 \left \{
        \begin{array}{cc}
        u_{1} < 1 & \mand v_{1} \leq 0  \\
        \mor \\
        u_{1} = 1 & \mand v_{1} < 0
        \end{array}
 \right.
        \label{eq-jain-1}
 \end{equation}
and
\begin{equation}
 \left \{
        \begin{array}{cc}
        u_{0} > 1 & \mand v_{0} \geq 0  \\
        \mor \\
        u_{0} = 1 & \mand v_{0} > 0
        \end{array}
 \right.
        \label{eq-jain-2}
 \end{equation}

The conditions above imply that the total rate $f(t)$ remains
below $c$ until it exceeds it once, then returns below $c$. Thus
the total rate $f(t)$ oscillates around the desired equilibrium.

Now we also wish to ensure fairness.  A first step is to measure
how much a given rate allocation deviates from fairness.  We
follow the spirit of \cite{CJ89} and use as a measure of
unfairness the distance between the rate allocation $\vec{x}$ and
its nearest fair allocation $\Pi(\vec{x})$, where $\Pi$ is the
orthogonal projection on the set of fair allocations, normalized
by the length of the fair allocation (Figure~\ref{fig-jain-1}).
\begin{figure}[h]
        \insfig{D31F4}{0.6}
        \mycaption{The measure of unfairness is $\tan(\alpha)$. The figure
        shows the effect on fairness of an increase or a decrease. The
        vector $\vec{1}$ is defined by $\vec{1}_{i}=1$ for all $i$.}
        \protect\label{fig-jain-1}
\end{figure}
In other words, the measure of unfairness is
$$d(\vec{x}) = \frac{||\vec{x} - \Pi(\vec{x})||}{||\Pi(\vec{x})||}$$
with $\Pi(\vec{x})_{i}= \frac{\sum_{j=1}^I x_{j}  }{ I}$ and the
norm is the standard euclidian norm defined by $||\vec{y}|| =
\sqrt{\sum_{j=1}^I y_{j}^2}$ for all $\vec{y}$.

Now we can study the effect of the linear control on fairness.
Figure~\ref{fig-jain-1}) illustrates that: (1) when we apply a
multiplicative increase or decrease, the unfairness is unchanged;
(2) in contrast, an additive increase decreases the unfairness,
whereas an additive decrease increases the unfairness.

We wish to design an algorithm such that at every step, unfairness
decreases or remains the same, and such that in the long term it
decreases. Thus,  we must have $v_{1}=0$, in other words, the
decrease must be multiplicative, and the increase must have a
non-zero additive component. Moreover, if we want to converge as
quickly as possible towards, fairness, then we must have
$u_{0}=0$. In other words, the increase should be purely additive.
In summary we have shown that:
\begin{fact}
Consider a linear adaptation algorithm of the form in
Equation~\ref{eq-theo-addmul1}. In order to satisfy efficiency and
convergence to fairness, we must have a multiplicative decrease
(namely, $u_{1}<1$ and $v_{1}=0$) and a non-zero additive
component in the increase (namely, $u_{0}\geq 1$ and $v_{0}>0$).
If we want to favour a rapid convergence towards fairness, then
the increase should be additive only (namely, $u_{0}= 1$ and
$v_{0}>0$). \label{fact-addmul}
\end{fact}

The resulting algorithm is called Additive Increase Multiplicative
Decrease (\aref{alg:aimd}).

\begin{algorithm}
\caption{Additive Increase Multiplicative Decrease (\nt{AIMD}) with
increase term $v_0>0$ and decrease factor
$0<u_1<1$.}\label{alg:aimd}
\begin{algorithmic}[0]% 0 means no line numbering; 1=number all lines
 \If{received feedback is negative}
 \State multiply rate by $u_1$
 \Else
 \State add $v_0$ to rate
 \EndIf
\end{algorithmic}
\end{algorithm}



 Let us now consider the dynamics of the
control system, in the case of additive increase, multiplicative
decrease. From Formula~\ref{eq-jain-48} we see that the \emph{total}
amount of traffic oscillates around the optimal value $c$.  In
contrast, we see on the numerical examples below that the measure of
unfairness converges to $0$ (This is always true if the conclusions
of Fact~(\ref{fact-addmul}) are followed; the proof is left as an
exercise to the reader). Figure~\ref{fig-addmulNum} shows some
numerical simulations.

The figure also shows a change in the number of active sources. It
should be noted that the value of $u_{1}$ (the multiplicative
decrease factor) plays an important role. A value close to $1$
ensures smaller oscillations around the target value; in contrast,
a small value of $u_{1}$ can react faster to decreases in the
available capacity.
\begin{figure}[h]
        \insfig{D31F7}{1.0}
        \mycaption{Numerical simulations of additive increase, multiplicative
        decrease. There are three sources, with initial rates of 3 Mb/s, 15
        Mb/s and 0. The total link rate is 10 Mb/s. The third source is
        inactive until time unit 100. Decrease factor = $0.5$; increment for
        additive increase: 1 Mb/s. The figure shows the rates for the three
        sources, as well as the aggregate rate and the measure of
        unfairness. The measure of unfairness is counted for two sources
        until time 100, then for three sources.}
        \protect\label{fig-addmulNum}
\end{figure}

Lastly, we must be aware that the analysis made here ignores the
impact of variable and different round trip times.


\subsection{Slow Start}
Slow start is a mechanism that can be combined with Additive
Increase Multiplicative Decrease; it applies primarily to the
initial state of a flow. It is based on the observation that if we
know that a flow receives much less than its fair share, then we can
deviate from Additive Increase Multiplicative Decrease and give a
larger increase to this flow.

We assume that we use Additive Increase Multiplicative Decrease,
with the same notation as in the previous subsection, i.e. with
additive increase term $u_0$ and decrease factor $v_1$.



\begin{figure}
\subfigure[Without Slow
Start]{\Ifignc{3sourcesnoSS}{0.9}{0.3}\label{fig-ss-noss}}
\subfigure[With Slow
Start]{\Ifignc{3sourcesSS}{0.9}{0.3}\label{fig-ss-ss}}
\mycaption{(a) Three sources applying AIMD, the third starting at
time 0 while the other two have reached a high rate. Left: rate of
source 1 on $x$-axis versus rate of source 3. Right: rates of all
sources versus time (number of iterations) on $x$-axis. AIMD
parameters: $v_0=0.01$; $u_1=0.5$. It takes many iterations for the
third source to reach the same rate. (b) Same but with the third
source applying slow start. It quickly converges to a similar rate.
Slow start parameters: $w_0 = 2$, $r_{\max}=10$}
        \protect\label{fig-ss}
\end{figure}

In \fref{fig-addmulNum} we showed flows that started with
arbitrary rates. In practice, it may not be safe to accept such
a behaviour; in contrast, we would like a flow starts with a
very small rate. We set this small rate to $v_0$. Now, a
starting flow in the situation, mentioned earlier, of a flow
competing with established flows, most probably receives less
than all others (\fref{fig-ss-noss}). Therefore, we can, for
this flow, try to increase its rate more quickly, for example
multiplicatively, until it receives a negative feedback. This
is what is implemented in the algorithm called ``slow start"
(\aref{alg:ss}).

\begin{algorithm}
\caption{Slow Start with the following parameters: AIMD constants
$v_0>0$ ,  $0<u_1<1$; multiplicative increase factor $w_0>1$;
maximum rate $r_{\max}>0$. }\label{alg:ss}
\begin{algorithmic}[1]
 \State $\mbox{rate}\gets v_0$\label{alg:ss:ini}
 \State $\mbox{targetRate}\gets r_{\max}$
 \State \textbf{do forever}
 \State receive feedback
 \If{feedback is positive}
 \State $\mbox{rate}\gets w_0 \cdot \mbox{rate}$\label{alg:ss:pos}
 \If{$\mbox{rate}\geq \mbox{targetRate} $ }
    \State $\mbox{rate}\gets \mbox{targetRate}$
    \State \textbf{exit do loop} \label{alg:ss:exit}
 \EndIf
 \Else
 \State $\mbox{targetRate} \gets \max(u_1 \cdot \mbox{rate}, v_0)$\label{alg:ss:rdw}
 \State $\mbox{rate}\gets v_0 $\label{alg:ss:neg}
 \EndIf
 \State \textbf{end do}
\end{algorithmic}
\end{algorithm}

The algorithm maintains both a rate (called $x_i$ in the
previous section) and a target rate. Initially, the rate is set
to the minimum additive increase $v_0$ and the target rate to
some large, predefined value $r_{\max}$ (lines~\ref{alg:ss:ini} and next).
The rate increases multiplicatively as long as positive
feedback is received (line~\ref{alg:ss:pos}). In contrast, if a
negative feedback is received, the target rate is decreased
multiplicatively (this is applied to the rate achieved so far,
line~\ref{alg:ss:rdw}) as with AIMD, and the rate is returned
to the initial value (lines~\ref{alg:ss:neg} and next).

The algorithm terminates when and if the rate reached the target
rate (line~\ref{alg:ss:exit}). From there on, the source applies
AIMD, starting from the current value of the rate. See \fref{fig-ss}
for a simulation (\fref{fig-ss-1}).

Note that what is slow in \emph{slow} start is the starting point $v_0$,
not the increase.


\begin{figure}\Ifig{ss}{0.5}{0.4}
\caption{Zoom on source 3 of \fref{fig-ss-ss} from times 1 to 20,
i.e. during slow start. Dashed line: target rate; plain line: rate.
Slow start ends at time 16.}\label{fig-ss-1}\end{figure}

\section{The fairness of additive increase, multiplicative
decrease with FIFO queues}
 \mylabel{fpaimd}

A complete modeling is very complex because it contains both a
random feedback (under the form of packet loss) and a random delay
(the round trip time, including time for destinations to give
feedback).  In this section we consider that all round trip times
are constant (but may differ from one source to another). The
fundamental tool is to produce an ordinary differential equation
capturing the dynamics of the network,

\subsection{A simplified model}
Call$x_i(t)$ the sending rate for source $i$. Call $t_{n,i}$ the
$n$th rate update instant for source $i$, and let $E_{n,i}$ be the
binary feedback: $E_{n,i}=1$ is a congestion indication; otherwise
$E_{n,i}=0$. Call $1-\eta_i$ the multiplicative decrease factor
and $r_i$ the additive increase component. The source reacts as
follows.
$$
x(t_{n+1,i}) = E_{n,i} (1-\eta_i) x(t_{n,i}) + (1 - E_{n,i})
(x(t_{n,i}) + r_i)
$$
which we can rewrite as \be x(t_{n+1,i}) - x(t_{n,i})
= r - E_{n,i} \left( \eta_i x(t_{n,i}) + r_i\right)
 \label{eq-sys-ode}
\ee If some cases, we can approximate this dynamic
behaviour by an ordinary differential equation (ODE).
The idea, which was developed by Ljung \cite{lju77}
and Kushner and Clark \cite{KC78}, is that the above
set of equations is a discrete time, stochastic
approximation of a differential equation. The ODE is
obtained by writing \bear
    \frac{dx_i}{dt} & = & \mbox{expected rate of change given the state of all
        sources at time $t$} \\\nonumber
    & = &\mbox{expected change divided by the expected update
    interval} \label{eq-ode}
  \eear
The result of the method is that the stochastic system
in Equation~\ref{eq-sys-ode} converges, in some sense,
towards the global attractor of the ODE in
Equation~\ref{eq-ode}, under the condition that the
ODE indeed has a global attractor \cite{BMP90}. A
global attractor of the ODE is a vector
$\vec{x*}=(x^*_i)$ towards which the solution
converges, under any initial conditions. The
convergence is for $\eta_i$ and $r_i$ tending to $0$.
Thus the results in this section are only
asymptotically true, and must be verified by
simulation.

Back to our model, call $\mu_i(t, \vec{x})$ the expectation of
$E_{n,i}$, given a set of rates $\vec{x}$; also call $u_i(t)$ the
expected update interval for source $i$. The ODE is:
\begin{equation}\mylabel{eq-odedebase}
   \frac{dx_i}{dt}
        =
   \frac{r_i - \mu_i(t, \vec{x}(t) )\left( \eta_i x_i(t) + r_i \right)}{u_i(t)}
\end{equation}

We first consider the original additive increase, multiplicative
decrease algorithm, then we will study the special case of TCP.

\subsection{Additive Increase, Multiplicative Decrease with one Update per RTT}
We assume that the update interval is a constant $\tau_i$ equal to
the round trip time for source $i$. Thus
$$
u_i(t) = \tau_i
$$
The expected feedback $\mu_i$ is given by
\begin{equation}
        \mu_i(t,\vec{x}(t)) = \tau x_{i}(t) \sum_{l=1}^{L} g_{l}(f_{l}(t))
        A_{l,i}
        \label{eq-feedback-2}
\end{equation}

with $f_{l}(t)= \sum_{j=1}^I A_{l,j}x_{j}(t)$.  In the formula,
$f_{l}(t)$ represents the total amount of traffic flow on link
$l$, while $A_{l,i}$ is the fraction of traffic from source $i$
which uses link $l$.  We interprete Equation~(\ref{eq-feedback-2})
by assuming that $g_{l}(f)$ is the probability that a packet is
marked with a feedback equal to 1 (namely, a negative feedback) by
link $l$, given that the traffic load on link $l$ is expressed by
the real number $f$.
%; in the regime of rare negative feedback, we
%assume that we can neglect the occurence of one packet marked with
%a negative feedback on several links within one time cycle.
Then Equation~(\ref{eq-feedback-2}) simply gives the expectation
of the number of marked packets received during one time cycle by
source $i$. This models accurately the case where the feedback
sent by a link is quasi-stationary, as can be achieved by using
active queue management such as RED (see later). This also assumes
that the propagation time is large compared to the transmission
time.

Putting all this together we obtain the ODE:

 \begin{equation}
        \frac{dx_{i}}{dt}
        =
        \frac{r_i}{\tau_i} -  x_{i} (r_{i}+\eta_i x_{i}) \sum_{l=1}^{L}
        g_{l}(f_{l})A_{l,i}
        \label{eq-ode-cas-A}
  \end{equation}

  with
  \begin{equation}
        f_{l} = \sum_{j=1}^{I} A_{l,j} x_{j}
        \label{eq-flux-l}
  \end{equation}

  In order to study the attractors of this ODE, we identify a Lyapunov
  for it \cite{refSurLyapunov}. To that end, we follow \cite{KMT97}
  and \cite{GB98} and note that
  $$
  \sum_{l=1}^{L}
        g_{l}(f_{l})A_{l,i} = \frac{\partial}{\partial x_{i}}\sum_{l=1}^{L}
        G_{l}(f_{l}) = \frac{\partial G (\vec{x})}{\partial x_{i}}
  $$
  where $G_{l}$ is a primitive of $g_{l}$ defined for example by
  $$
  G_{l}(f) = \int_{0}^f g_{l}(u) du
  $$
  and
  $$G(\vec{x}) = \sum_{l=1}^{L} G_{l}(f_{l})
  $$
  We can then rewrite Equation~(\ref{eq-ode-cas-A}) as
  \begin{equation}
                \frac{dx_{i}}{dt}
        =
        x_{i}(r_{i}+ \eta_i x_{i})
        \left \{
            \frac{r_{i}}{\tau_i x_{i}(r_{i}+ \eta_i x_{i})}  -
            \frac{\partial G (\vec{x})}{\partial x_{i}}
        \right \}
        \label{eq-cas-A-modif}
  \end{equation}

  Consider now the function $J_{A}$ defined by
  \begin{equation}
        J_{A}(\vec{x}) = \sum_{i=1}^I \phi(x_{i}) -  G(\vec{x})
        \label{eq-def-de-J}
  \end{equation}

  with
  $$\phi(x_{i}) = \int_{0}^{x_{i}} \frac{r_{i}du}{\tau_i u(r_{i}+ \eta_i u)}=
  \frac{1}{\tau_i}\log \frac{x_{i}}{ r_{i}+ \eta_i x_{i}}
  $$

  then we can rewrite Equation~(\ref{eq-cas-A-modif}) as

  \begin{equation}
                \frac{dx_{i}}{dt}
        =
        x_{i}(r_{i}+ \eta_i x_{i})
        \frac{\partial J_{A} (\vec{x})}{\partial x_{i}}
        \label{eq-cas-A-modif-2}
  \end{equation}

  Now it is easy to see that $J_{A}$ is strictly concave and therefore has a unique
  maximum over any bounded region. It follows from this and from
  Equation~(\ref{eq-cas-A-modif-2}) that $J_{A}$ is a Lyapunov for the ODE
  in~(\ref{eq-ode-cas-A}), and thus, the ODE
  in~(\ref{eq-ode-cas-A}) has a unique attractor, which is the point
  where the maximum of $J_{A}$ is reached. An intuitive explanation of the Lyapunov is as
  follows. Along any
  solution $\vec{x}(t)$ of the ODE, we have
  $$
\frac{d}{dt}J_A(\vec{x}(t)) = \sum_{i=1}^{I} \frac{\partial
J_A}{\partial x_{i}}\frac{dx_i}{dt}
=
\sum_{i=1}^{I} x_i(r_{i}+ \eta_i x_{i})
 \left( \frac{\partial J_A}{\partial x_{i}} \right)^2
  $$
  Thus $J_A$ increases along any solution, thus solutions tend to
  converge towards the unique maximum of $J_A$.

  This shows that
  the rates $x_{i}(t)$ converge at equilibrium towards a
  set of value that maximizes $J_{A}(\vec{x})$, with $J_{A}$ defined by
  $$
  J_{A}(\vec{x}) =
  \sum_{i=1}^I
   \frac{1}{\tau_i}\log \frac{x_{i}}{ r_{i}+ \eta_i x_{i}}
   \; -  \; G(\vec{x})
  $$
\paragraph{Interpretation}
In order to interpret the previous results, we follow \cite{KMT97}
and assume that, calling $c_{l}$ the capacity of link $l$, the
function $g_{l}$ can be assumed to be arbitrarily close to
$\delta_{c_{l}}$, in some sense, where
$$
\delta_{c}(f) = 0 \mif f < c \mand \delta_{c}(f) = + \infty \mif f
\geq c
$$

Thus, at the limit, the method in \cite{KMT97} finds that the
rates are distributed so as to maximize
$$F_{A}(\vec{x})=\sum_{i=1}^I \frac{1}{\tau_i}\log \frac{x_{i}}{ r_{i}+ \eta_i x_{i}}$$
subject to the constraints
$$
\sum_{j=1}^I A_{l,j}x_{j} \; \leq c_{l} \; \mfa l
$$

We can learn two things from this. Firstly, the weight given to
$x_{i}$ tends to $-\log \eta_i$ as $x_{i}$ tends to $+\infty$.
Thus, the distribution of rates will tend to favor small rates,
and should thus be closer to max-min fairness than to proportional
fairness. Secondly, the weight is inversely proportional to the
round trip time, thus flows with large round trip times suffer
from a negative bias, independent of the number of hops they use.
%
%We see that, under the simplifying assumptions above, the additive
%increase, multiplicative decrease principle distributes rates
%according to a utility fairness, which is between max-min fairness
%and proportional fairness.

\subsection{Additive Increase, Multiplicative Decrease with one
Update per Packet}
 Assume now that a source reacts to every feedback received.
 Assume that losses are detected immediately, either because the
 timeout is optimal (equal to the roundtrip time), or because of
 some other clever heuristic. Then we can use the same analysis as
 in the previous section, with the following adaptations.

 The ODE is still given by Equation (\ref{eq-odedebase}). The
 expected feedback is now simply equal to the probability of a
 packet being marked, and is equal to
 $$
 \mu_i(t, \vec{x}) = \sum_{l=1}^{L}
        g_{l}(f_{l})A_{l,i}
 $$

 The average update interval is now equal to $\frac{1}{x_i}$. Thus
 the ODE is
\begin{equation}\mylabel{eq-odeevpack}
\frac{dx_{i}}{dt}
        =
        r_i x_i -  x_{i} (r_{i}+\eta_i x_{i}) \sum_{l=1}^{L}
        g_{l}(f_{l})A_{l,i}
\end{equation}

which can also be written as
\begin{equation}\mylabel{eq-odeevpack-2}
\frac{dx_{i}}{dt}
        =
 x_{i}(r_{i}+ \eta_i x_{i})
        \left \{
            \frac{r_{i}}{r_{i}+ \eta_i x_{i}}  -
            \frac{\partial G (\vec{x})}{\partial x_{i}}
        \right \}
\end{equation}

Thus a Lyapunov for the ODE is
$$
J_B(\vec{x})= \sum_{i=1}^I
   \frac{r_i}{\eta_i}\log ( r_{i}+ \eta_i x_{i})
   \; -  \; G(\vec{x})
$$

Thus, in the limiting case where the feedback expectation is close
to a Dirac function, the rates are distributed so as to maximize
$$F_{B}(\vec{x})=\sum_{i=1}^I \frac{r_i}{\eta_i}\log ( r_{i}+ \eta_i x_{i})$$
subject to the constraints
$$
\sum_{j=1}^I A_{l,j}x_{j} \; \leq c_{l} \; \mfa l
$$
The weight given to $x_{i}$ is close to $\log(\eta_i x_{i})$ ($=
\log x_i + \mbox{a constant} $) for large $x_i$, and is larger
than $\log(\eta_i x_{i})$ in all cases. Thus, the distribution of
rates is derived from proportional fairness, with a positive bias
given to small rates. Contrary to the previous case, there is no
bias against long round trip times.
%

In Section~\ref{sec-tcpalgodecon} on
page~\pageref{sec-tcpalgodecon} we study the case of TCP.
\section{Summary}\begin{enumerate}
    \item In a packet network, sources should limit their sending rate in order
    to account for the state of the network. Failing
    to do so might result in congestion collapse.
    \item Congestion collapse is defined as a severe decrease in total
    network throughput when the offered load increases.
    \item Maximizing network throughput as a primary objective might lead to
large unfairness and is not a viable objective.
  \item The
objective of congestion control is to provide Pareto efficiency and
some form of fairness.


\item Fairness can be defined in various ways: max-min,
proportional and variants of proportional.

\item End-to-end congestion control in packet networks is based on the adaptation mechanism of additive increase,
multiplicative decrease.

\item Slow start is a mechanism for a source to quickly increase its sending rate (instead of starting upfront at a high rate).
\end{enumerate}
%
% calcul de fair share a la golestani sur un exemple (Kuhn Tucker)
%
% objectif perte 0.05
%
% 4.3.2 de Jin Roberts: forme explicite du throughput pour rate prop. fair
% sharing
%
% reprendre mon exmple et calculer le temps moyen en utilisant JimM
% Robert section 4.3.2
%
% exo: calculer le max min fair share pour l'exemple de JEC papier EFOC
%      comparer au throughput total
%      calculer rate proportional fairness pour cet exemple
%
% peut etre: forme produit et methode de Metropolis
%
% peut etre: fixed window de JR
%
% \section{Appendix: Optimization}
%
% Linear programming
%
% Lagrangian Methods
%
% Kuhn Tucker: theoreme de Leunberger page 401 et Gustavo de veciana
%    Lagrangien
%    conditions de KT: page 401 de Luenberger

\setcounter{chapter}{31}
\chapter{Congestion Control for Best Effort:
Internet}
\label{d32}

In this chapter you learn how the theory of congestion
control for best effort is applied in the Internet.

\begin{itemize}

   \item  congestion control algorithms in TCP

%    \item  congestion control with ABR
%
%    \item early packet discard
%
        \item  the concept of TCP friendly sources

        \item  random early detection (RED) routers and active queue management

\end{itemize}
\providecommand{gradCourse}{oui}
%% common to T2 and ED
%% use \gradCourse to control what is included
\section{Congestion control algorithms of TCP}
\mylabel{sec-tcpalgodecon}

We have seen that it is necessary to control the amount of traffic
sent by sources, even in a best effort network.
In the early days of Internet, a congestion collapse did occur. It was
due to a combination of factors, some of them were the absence of
traffic control mechanisms, as explained before. In addition, there
were other aggravating factors which led to ``avalanche'' effects.

\begin{itemize}
\item IP fragmentation: if IP datagrams are fragmented into several
packets, the loss of one single packet causes the destination to
declare the loss of the entire datagram, which will be
retransmsitted. This is addressed in TCP by trying to avoid
fragmentation. With IPv6, fragmentation is possible only at the source
and for UDP only.
\item Go Back $n$ at full window size: if a TCP sender has a large
offered window, then the loss of segment $n$ causes the retransmission
of all segments starting from $n$.  Assume only segment $n$ was lost,
and segments $n+1,\ldots, n+k$ are stored at the receiver; when the
receiver gets those segments, it will send an ack for all segments up
to $n+k$.  However, if the window is large, the ack will reach the
sender too late for preventing the retransmissions.  This has been
addressed in current versions of TCP where all timers are reset when
one expires.
\item In general, congestion translates into larger
delays as well (because of queue buildup). If nothing is done,
retransmission timers may become too short and cause retransmissions
of data that was not yet acked, but was not yet lost.
This has been addressed in TCP by the round trip estimation algorithm.
\end{itemize}

Congestion control has been designed right from the beginning in wide-area,
public networks (most of them are connection oriented using
X.25 or Frame Relay), or in large corporate networks such as IBM's SNA. It
came as an afterthought in the Internet.  In connection oriented
network, congestion control is either hop-by-hop or rate
based: credit or
backpressure per connection (ATM LANs), hop-by-hop sliding window per
connection (X.25, SNA); rate control per connection (ATM).
They can also use end-to-end control,
based on marking packets that have experienced congestion
(Frame Relay, ATM). Connectionless wide area networks all rely
 on end-to-end control.

In the Internet, the
principles are the following.
\begin{itemize}
    \item TCP is used to control traffic

        \item  the rate of a TCP connection is controlled by adjusting the
        window size

        \item  additive increase, multiplicative decrease and slow start, as defined in \cref{d31},
        are
        used

        \item  the feedback from the network to sources is packet loss. It is
        thus assumed that packet loss for reasons other than
        packet dropping in queues is negligible. In particular, all links
        should have a negligible error rate.
\end{itemize}



One implication of these design decisions is that only TCP traffic is
controlled.  The reason is that, originally, UDP was used only for
short transactions.  Applications that do not use TCP have to either
be limited to LANs, where congestion is rare (example: NFS) or have to
implement in the application layer appropriate congestion control
mechanisms (examplee.g., QUIC or some audio and video applications).  We will see later
what is done in such situations.

Only long lasting flows are the object of congestion control.  There
is no congestion control mechanism for short lived flows.

The detailed mechanisms are described below.  Over the years, many variants of TCP congestion control emerged. We describe here only a few representative ones. The concepts are heavily influenced by the historical version, TCP RENO and its variant TCP NEW RENO, which we describe next. Then we also describe TCP CUBIC and Data Center TCP, two widespread variants.

%Note that there are other forms of congestion avoidance in a network.
%One form of congestion avoidance might also performed by
%routing algorithms which support load sharing; this is not very
%much used in the Internet, but is fundamental in telephone networks.
%
%ICMP source quench messages can also be sent by a router to reduce the
%rate of a source. However, this is not used significantly and is not
%recommended in general, as it adds traffic in a period of congestion.
%
%In another context, the synchronization avoidance algorithm (Module
%M3) is also an example of congestion avoidance implemented outside of
%TCP.

\subsection{Congestion Window}

Remember that, with the sliding window protocol concept (used by TCP),
the window size $W$ (in bits or bytes) is equal to the maximum number
of unacknowledged data that a source may send.  Consider a system
where the source has infinite data to send; assume the source uses a
FIFO queue as send buffer, of size $W$.  At the beginning of the connection,
the source immediately fills the buffer which is then dequeued at the
rate permitted by the line.  Then the buffer can receive new data as
old data is acknowledged. Let $T$ be the average time you need
to wait for an acknowledgement to come back, counting from the
instant the data is put into the FIFO queue. This system is an approximate
model of a TCP connection for a source which is infinitely fast and
has an infinite amount of data to send. By Little's formula applied to
the FIFO queue, the throughput $\theta$ of the TCP connection is given by
(prove this as an exercize):
\begin{equation}
        \theta =\frac{ W}{T}
        \label{eq-d32-i8}
\end{equation}
The delay $T$ is equal to the propagation and transmission times of
data and acknowledgement, plus the processing time, plus possible
delays in sending acknowledgement. If $T$ is fixed, then controlling $W$
is equivalent to controlling the connection rate $\theta$. This is
the method used in the Internet. However, in general, $T$ depends
also on the congestion status of the networks, through queueing
delays. Thus, in periods of congestions, there is a first, automatic
congestion control effect: sources reduce their rates whenever the network
delay increases, simply because the time to get acknowledgements
increase. This is however a side effect, which is not essential in
the TCP congestion control mechanism.

TCP defines a variable called congestion window (\texttt{cwnd}); the
window size $W$ is then given by
$$W = \min ( \texttt{cwnd}, \texttt{offeredWindow})$$
Remember that \texttt{offeredWindow} is the window size advertized by
the destination. In contrast, \texttt{cwnd} is computed by the source.

The value of \texttt{cwnd} is decreased when a loss is detected, and
increased otherwise.  In the rest of this section we describe the
details of the operation.

A TCP connection is, from a congestion control point of view, in one
of three phases.

  \begin{itemize}
        \item  slow start: after a loss detected by timeout

        \item  fast recovery: after a loss detected by fast retransmit

        \item  congestion avoidance: in all other cases.
  \end{itemize}

The variable \texttt{cwnd} is updated at phase transitions, and when
useful acknowledgements are received. Useful acknowledgements are
those which increase the lower edge of the sending window, i.e. are
not duplicate.

\subsection{Slow Start and Congestion avoidance}

\begin{figure}[h]
        \insfig{D32F1}{0.6}
        \mycaption{Slow Start and Congestion Avoidance,
        showing the actions taken in the phases and at phase transitions.}
        \protect\label{d32f1}
\end{figure}

In order to simplify the description, we first describe an
incomplete system with only two phases: slow start and congestion
avoidance.  This corresponds to a historical implementation (TCP
Tahoe) where losses are detected by timeout only (and not by the
fast retransmit heuristic).  According to the additive increase,
multiplicative decrease principle, the window size is divided by 2
for every packet loss detected by timeout. In contrast, for every
useful acknowledgement, it is increased according to a method
described below, which produces a roughly additive increase. At the
beginning of a connection, slow start is used (\paref{alg:ss}). Slow
start is also used after every loss detected by timeout. Here the
reason is that we expect most losses to be detected instead by fast
retransmit, and so losses detected by timeout probably indicate a
severe congestion; in this case it is safer to test the water
carefully, as slow start does.

Similar to ``targetRate" in \aref{alg:ss}, a supplementary variable
is used, which we call the target window (\texttt{twnd}) (it is
called \texttt{ssthresh} in RFC 5681).  At the beginning of the
connection, or after a timeout, \texttt{cwnd} is set to 1 segment
and a rapid increase based on acknowledgements follows, until
\texttt{cwnd} reaches \texttt{twnd}.

The algorithm for computing \texttt{cwnd} is shown on
Figure~\ref{d32f1}.  At connection opening, \texttt{twnd} has the
maximum value (64KB by default, more if the window scale option is
used -- this corresponds to the $r_{\max}$ parameter of
\aref{alg:ss}), and \texttt{cwnd} is one segment.  During slow
start, \texttt{cwnd} increases exponentially (with increase factor
$w_0=2$). Slow start ends whenever there is a packet loss detected
by timeout or \texttt{cwnd} reaches \texttt{twnd}.  During
congestion avoidance, \texttt{cwnd} increases according to the
additive increase explained later.  When a packet loss is detected
by timeout, \texttt{twnd} is divided by 2 and slow start is entered
or re-entered, with \texttt{cwnd} set to 1.

Note that \texttt{twnd} and \texttt{cwnd} are equal in the congestion
avoidance phase.

Figure~\ref{d32f2} shows an example built with data from \cite{BP95}.
Initially, the connection congestion state is slow-start.  Then
\texttt{cwnd} increases from one segment size to about 35 KB, (time
0.5) at which point the connection waits for a missing non duplicate
acknowledgement (one packet is lost).  Then, at approximately time 2,
a timeout occurs, causing \texttt{twnd} to be set to half the current
window, and \texttt{cwnd} to be reset to 1 segment size.  Immediately
after, another timeout occurs, causing another reduction of
\texttt{twnd} to 2 $\times$ \texttt{segment size} and of \texttt{cwnd}
to 1 segment size.  Then, the slow start phase ends at point A, as one
acknowledgement received causes \texttt{cwnd} to equal \texttt{twnd}.
Between A and B, the TCP connection is in the congestion avoidance
state.  \texttt{cwnd} and \texttt{twnd} are equal and both increase
slowly until a timeout occurs (point B), causing a return to slow
start until point C. The same pattern repeats later.  Note that some
implementations do one more multiplicative increase when \texttt{cwnd}
has reached the value of \texttt{twnd}.

\begin{figure}[h]
        \insfig{D32F2}{0.7}
        \mycaption{Typical behaviour with slow start and
        congestion avoidance, constructed with data from \cite{BP95}. It
        shows the values of \texttt{twnd} and \texttt{cwnd}.  }
        \protect\label{d32f2}
\end{figure}

The slow start and congestion avoidance phases use three algorithms
for decrease and increase, as shown on Figure~\ref{d32f1}.
\begin{enumerate}
        \item  Multiplicative Decrease for \texttt{twnd}

\begin{verbatim}
                twnd = 0.5 * current window size
                twnd = max (twnd, 2 * segment size)
        \end{verbatim}

        \item  Additive Increase for \texttt{twnd}
\begin{verbatim}
        for every useful acknowledgement received :
                twnd = twnd + (segment size) * (segment size) / twnd
                twnd = min (twnd, maximum window size)
        \end{verbatim}
        \item  Exponential Increase for \texttt{cwnd}
        \begin{verbatim}
        for every useful acknowledgement received :
                cwnd = cwnd + (segment size)
                if (cwnd == twnd) then move to congestion avoidance
        \end{verbatim}

\end{enumerate}
% They algorithms are described now.
%
% \subsubsection{Multiplicative Decrease for \texttt{twnd}}
%       \begin{verbatim}
%               twnd = 0.5 * min (current window size)
%               twnd = max (twnd, 2 * segment size)
%       \end{verbatim}
%
% \subsubsection{Additive Increase for \texttt{twnd}}
%       \begin{verbatim}
%       for every useful acknowledgement received :
%               twnd = twnd + (segment size) * (segment size) / twnd
%               twnd = min (twnd, maximum window size)
%       \end{verbatim}
%
% \subsubsection{Exponential Increase for \texttt{cwnd}}
% \begin{verbatim}
%       for every useful acknowledgement received :
%               cwnd = cwnd + (segment size)
%               if (cwnd == twnd) then move to congestion avoidance
%       \end{verbatim}
%
In order to understand the effect of the Additive Increase algorithm,
remember that TCP windows are counted in bytes, not in packets.
Assume that \texttt{twnd} $= w$ \texttt{segment size}, thus $w$ is the
size counted in packets, assuming all packets have a size equal to
\texttt{segment size}. Thus, for every
acknowledgement received, $\texttt{twnd} / \texttt{segment size}$
is increased by $1/w$, and it takes a full window to increment $w$ by
one. This is equivalent to an additive increase if the time to
receive the acknowledgments for a full window is constant.
Figure~\ref{d32f3} shows an example.

Note that additive increase is applied during congestion avoidance,
during which phase we have \texttt{twnd} $=$ \texttt{cwnd}.
\begin{figure}[h]
        \insfig{D32F3}{0.4}
        \mycaption{The additive increase algorithm.}
        \protect\label{d32f3}
\end{figure}


The Exponential Increase algorithm is applied during the slow start
phase.  The effect is to increase the window size until \texttt{twnd},
as long as acknowledgements are received.  Figure~\ref{d32f4} shows an
example.

\begin{figure}[h]
        \insfig{D32F4}{0.4}
        \mycaption{The exponential increase  algorithm for \texttt{cwnd}.}
        \protect\label{d32f4}
\end{figure}

Finally, Figure~\ref{d32f7} illustrates the additive increase,
multiplicative decrease principle and the role of slow start. Do not
misinterpret the term ``slow start'': it is in reality a phase of
rapid increase; what is slow is the fact that $\texttt{cwd}$ is set
to 1.  The slow increase is during congestion avoidance, not during
slow start.

\begin{figure}[h]
        \insfig{D32F7}{0.6}
        \mycaption{Additive increase, Multiplicative decrease and
slow start.}
        \protect\label{d32f7}
\end{figure}

\subsection{Fast Recovery}

As mentioned earlier, the full specification for TCP involves a third
state, called Fast Recovery, which we describe now.  Remember from the
previous section that when a loss is detected by timeout, the target congestion
window size \texttt{twnd} is divided by 2 (Multiplicative Decrease
for \texttt{twnd}) but we also go into the slow start phase in order
to avoid bursts of retransmissions.

However, this is not very efficient if an isolated loss occurs.
Indeed, the penalty imposed by slow start is large; it will take
about $\log n$ round trips to reach the target window size
\texttt{twnd}, where $n = \texttt{twnd} / \texttt{segment size}$.
This is in particular to severe if the loss is isolated, corresponding
to a mild negative feedback. Now with the current TCP, isolated losses
are assumed to be detected and repaired with the Fast Retransmit
procedure.

Therefore, we add a different mechanism for
every loss detected by  Fast Retransmit. The procedure is as follows.

\begin{itemize}
        \item  when a loss is detected by Fast Retransmit (triplicate ack),
        then run Multiplicative Decrease for \texttt{twnd} as described in
        the previous section.

        \item  Then enter a temporary phase, called  Fast Recovery,
        until the loss is repaired.
        When entering this phase, temporarily
        keep  the congestion
        window high in order to keep sending. Indeed, since an ack is missing, the
        sender is likely to be blocked, which is not the desired effect:

        \begin{verbatim}
  cwnd = twnd + 3 *seg /* exponential increase */
  cwnd = min(cwnd, 65535)
  retransmit missing segment (say n)
    \end{verbatim}

        \item Then continue to interprete every received ack as a positive signal,
        at least until the lost is repaired, running the exponential increase
        mechanism:

        \begin{verbatim}
  duplicate ack received ->
     cwnd = cwnd + seg             /* exponential increase */
     cwnd = min(cwnd, 65535)
     send following segments if window allows

  ack for segment n received ->
     go into to cong. avoidance

        \end{verbatim}

\end{itemize}

Figure~\ref{d32f5} shows an example.
\begin{figure}[h]
        \insfig{D32F5}{0.7}
        \mycaption{A typical example with slow start (C-D) and fast recovery
        (A-B and E-F),
        constructed with data from \cite{BP95}. It shows the values of
        \texttt{twnd} and \texttt{cwnd}.}
        \protect\label{d32f5}
\end{figure}

If we combine the three phases described in this and the previous
section, we obtain the complete diagram, shown on Figure~\ref{d32f6}.
\begin{figure}[h]
        \insfig{D32F6}{0.6}
        \mycaption{Slow Start, Congestion Avoidance and Fast Retransmit
        showing the phase transitions.}
        \protect\label{d32f6}
\end{figure}


\subsection{Summary and Comments}

In summary for that section, we can say that the congestion avoidance
principle for the Internet, used in TCP, is additive increase,
multiplicative decrease. The sending rate of a source is governed by a
target window size \texttt{twnd}. The principle of
additive increase,
multiplicative decrease is summarized as follows. At this point you
should be able to understand this summary; if this is not the case,
take the time to read back the previous sections. See also
Figure~\ref{d32f7} for a synthetic view of the different phases.

\begin{itemize}

        \item when a loss is detected (timeout or fast retransmit), then
        \texttt{twnd} is divided by 2 (``Multiplicative Decrease for
        \texttt{twnd}'')

        \item In general (namely in the congestion avoidance phase), for
        every useful (i.e. non duplicate) ack received, \texttt{twnd} is increased linearly
        (``Additive Increase for \texttt{twnd}'')

        \item  Just after a loss is detected a special transient phase is
        entered. If the loss is detected by timeout, this phase is slow
        start; if the loss is detected by fast retransmit, the phase is fast
        recovery. At the beginning of a
        connection, the slow start phase is also entered.

        During such a transient phase, the congestion window size is
        different from \texttt{twnd}. When the transient phase terminates,
        the connection goes into the congestion avoidance phase. During
        congestion avoidance, the congestion window size is
        equal to \texttt{twnd}.

\end{itemize}


\section{Analysis of TCP Reno}

\subsection{The fairness of TCP RENO}
In this section we determine the fairness of TCP, assuming all
round trip times remain constant over time. We can apply the analysis in
Section~\ref{fpaimd} and use the method of the ODE.

TCP differs slightly from the plain additive increase,
multiplicative decrease algorithm. Firstly, it uses a window
rather than a rate control. We can approximate the rate of a TCP
connection, if we assume that transients due to slow start and
fast recovery can be neglected, by $x= \frac{w}{\tau}$, where
$w$ is equal to \texttt{cwnd} and $\tau$ is the round trip time,
assumed to be constant. Secondly, the increase in rate is not
strictly additive; in contrast, the window is increased by
$\frac{1}{w}$ for every positive acknowledgement received. The
increase in rate at every positive acknowledgement is thus equal
to $\frac{K}{w \tau}=\frac{K}{x \tau^2}=$ where $K$ is a constant.
If the unit of data is the packet, then $K=1$; if the unit of data
is the bit, then $K=L^2$, where $L$ is the packet length in bits.

In the sequel we consider some variation of TCP where the window
increase is still given by $\frac{K}{w \tau}$, but where $K$ is
not necessarily equal to $(1 \mbox{ packet})^2$ and need not be the same for all TCP connections.

We use the same notation as in Section~\ref{fpaimd} and call
$x_i(t)$ the rate of source $i$. The coefficient $K$ may now
depend on the connection $i$ and is denoted with $K_i$. The ODE is obtained by substituting $r_i$
by $\frac{K_i}{x_i \tau_i^2}$ in Equation~(\ref{eq-odeevpack}) on
page~\pageref{eq-odeevpack}:

\begin{equation}\mylabel{eq-odetcp}
\frac{dx_{i}}{dt}
        =
        \frac{K_i}{ \tau_i^2}  -  ( \frac{K_i}{ \tau_i^2}+\eta_i x_{i}^2) \sum_{l=1}^{L}
        g_{l}(f_{l})A_{l,i}
\end{equation}

which can also be written as
\begin{equation}
\mylabel{eq-odetcp-2}
\frac{dx_{i}}{dt}
        =
 ( \frac{K_i}{ \tau_i^2}+\eta_i x_{i}^2)
       \left\{
         \frac
            { \frac{K_i}{ \tau_i^2 \eta_i}}
            {\frac{K_i}{ \tau_i^2 \eta_i} + x_i^2}  -
           \frac{\partial G (\vec{x})}{\partial x_{i}}
        \right\}
\end{equation}

This shows that
  the rates $x_{i}(t)$ converge at equilibrium towards a
  set of value that maximizes $J_{C}(\vec{x})$, with $J_{C}$ defined by
\begin{equation}\mylabel{eq-defJC}
  J_{C}(\vec{x}) =
  \sum_{i=1}^I
   \frac{1}{\tau_i}
   \sqrt{
      \frac{K_i}{\eta_i}
      }
   \arctan
   \frac{x_{i} \tau_i}{\sqrt{\frac{K_i}{\eta_i}}}
   \; -  \; G(\vec{x})
\end{equation}

  Thus, in the limiting case where the feedback expectation is close
to a Dirac function, the rates are distributed so as to maximize
$$F_{C}(\vec{x})=\sum_{i=1}^I \frac{1}{\tau_i}
   \sqrt{
      \frac{K_i}{\eta_i}
      }
   \arctan
   \frac{x_{i} \tau_i}{\sqrt{\frac{K_i}{\eta_i}}}$$
subject to the constraints
$$
\sum_{j=1}^I A_{l,j}x_{j} \; \leq c_{l} \; \mfa l
$$

\paragraph{The bias of TCP Reno against long round trip times}

If we use the previous analysis with the standard parameters used
with TCP-Reno, we have $\eta_i=0.5$ for all sources and (the unit
of data is the packet)
 $$K_i = 1
 $$
With these values, the expected rate of change (right hand side in
Equation~(\ref{eq-odetcp})) is a decreasing function of $\tau_i$.
Thus, the adaptation algorithm of TCP contains a negative bias
against long round trip times. More precisely, the weight given to
source $i$ is
 $$\frac{\sqrt{2}}{\tau_i}
   \arctan
   \frac{x_{i} \tau_i}{\sqrt{2}}
 $$

 If $x_i$ is very small, then this is approximately equal to $2
 x_i$, independent of $\tau_i$. For a very large $x_i$, it is
 approximately equal to $\frac{\sqrt{2} \Pi}{2 \tau_i}$. Thus,
 the negative bias against very large round trip times is
 important
 only in the cases where the rate allocation results into a large
 rate.

 Note that the bias against long round trip times comes beyond and
 above the fact that, like with proportional fairness, the
 adaptation algorithm gives less to sources using many resources.
 Consider for example two sources with the same path, except for an access
 link which has a very large propagation delay for the second source.
 TCP will give less throughput to the second one, though both are
 using the same set of resources.

 We can correct the bias of TCP against long round trip times by
 changing $K_i$. Remember that $K_i$ is such that the
 window $w_i$ is increased for every positive acknowledgement by $\frac{K_i}{w_i}$.
  Equation~(\ref{eq-odetcp}) shows that we should
 let $K_i$ be proportional to $\tau_i^2$. This is the modification
 proposed in \cite{floyd-91-b} and
\cite{henderson-98-a}. Within the limit of our analysis, this
would
 eliminate the non-desired bias against long round trip times.
 Note that, even with this form of fairness, connections using
 many hops are likely to receive less throughput; but this is not
 because of a long round trip time.

If we compare the fairness of TCP to proportional fairness, we see
that the weight given to $x_{i}$ is bounded both as $x_i$ tends to
$0$ or to $+\infty$. Thus, it gives more to smaller rates than
proportional fairness.

 In summary, we have proven that:
\begin{proposition}
TCP tends to distribute rates so as to maximize the utility
function $J_C$ defined in Equation~(\ref{eq-defJC}).
\begin{itemize}
  \item If the window increase parameter is as with TCP Reno ($K_i=1$ for
all sources), then TCP has a non-desired negative bias against
long round trip times.
  \item If in contrast the bias is
corrected (namely, the window $w_i$ is increased for every
positive acknowledgement by $\frac{K \tau_i^2}{w_i}$), then the
fairness of TCP is a variant of proportional fairness which gives
more to smaller rates.
\end{itemize}
\end{proposition}






Thus, TCP Reno distributes rates equally among connections having
the same paths, but has a bias against connections
 \begin{enumerate}
        \item  with a large number of hops

        \item  or with a large round trip time
 \end{enumerate}

The first bias is a result of proportional fairness and can be
considered to be justified; the second bias is less justified and
is indeed a problem for connections using links with large latency, eg., satellite links.

\subsection{The Loss-Throughput Formula}
A coarser, but more enlightening, analysis of the performance of TCP Reno leads to a simple relation between the loss probability experienced by a TCP connection and its throughput, assuming the application has an infinite amount of data to send (i.e., the bottleneck is the network). We expect the throughput of TCP to decrease with the loss probability; this is indeed captured by the following result.
 %
%%%% this part is G1 for grad school only
%This can be achieved by extending the modelling of
%Section~\ref{fpaimd} to cases with different round trip times. A
%simpler alternative can be derived following an approximate analysis.

\begin{theorem}
 [TCP loss - throughput formula \cite{Ott97}]
Consider a TCP Reno connection with constant round trip time $\tau$ and
constant packet size $L$; assume that the network is stationary,
that the transmission time is negligible compared to the round trip
time, that losses are rare and that the time spent in slow start or fast
recovery is negligible; then the average throughput $\theta$ (in
bits/s) and the average packet loss ratio $q$ are linked by the
relation
\begin{equation}
        \theta \approx \frac{L}{\tau} \frac{C}{\sqrt{q}}
        \label{eq-tcplr}
\end{equation}
with $C = 1.22$
\label{theo-lf}
\end{theorem}

\pr We consider that we can neglect the phases where \texttt{twnd} is
different from \texttt{cwnd} and we can thus consider that the
connection follows the additive increase, multiplicative decrease
principle.  We assume that the network is stationary; thus the
connection window size \texttt{cwnd} oscillates as illustrated on
Figure~\ref{d32f8}.
\begin{figure}[h]
        \insfig{D32F8}{0.7}
        \mycaption{The evolution of \texttt{cwnd} under the assumptions in the
        proposition.}
        \protect\label{d32f8}
\end{figure}
The oscillation is made of cycles of duration $T_{0}$. During one
cycle, \texttt{cwnd} grows until a maximum value $W$, then a loss is
encountered, which reduces \texttt{cwnd} to $\frac{W}{2}$. Now from
our assumptions, exactly a full window is sent per round trip time,
thus the window increases by one packet per round trip time, from
where it follows that
$$
T_{0} = \frac{W}{2} \tau
$$
The sending rate is approximately $\frac{W(t)}{\tau}$. It follows also
that the number of packets sent in one cycle, $N$, is given by
$$
N = \int_{0}^{T_{0}} \frac{W(t)}{\tau} dt = \frac{3}{8} W^2
$$
Now one packet is lost per cycle, so the average loss ratio is
$$
q = 1 / N
$$
We can extract $W$ from the above equations and obtain
$$
W = 2 \sqrt{\frac{2}{3}} \frac{1}{\sqrt{q}}
$$
Now the average throughput is given by
$$
\theta =  \frac{(N - 1) }{ T_{0}} L =
\frac{\left ( \frac{1}{q} -1 \right ) L
}{
\sqrt{\frac{2}{3}} \frac{1}{\sqrt{q}} \tau }
$$
If $q$ is very small this can be approximated by
$$
\theta \approx \frac{L}{\tau} \frac{C }{\sqrt{q}}
$$
with $C=\sqrt{\frac{3}{2}}$.
\qed

Formula~\ref{eq-tcplr} is taken as a basis for designing alternatives to TCP Reno. It can be shown that the formula still holds, though
with a slightly different constant, if more realistic modelling
assumptions are taken \cite{Flo91, Lak97}.
%%% end of G1 for undergrads only



%
% Jim Roberts: WFQ + TCP = max min fairness (analyse statique)
%
%
%
% Position du probleme par Golestani
%   minimiser cout de la source plus du reseau
%   appliquer a l'approx TCP ideale (MCFC) ou le feedback est donne par
%   la perte
%
%   si TCP etait rate based (donc si c'etait MCFC) alors on aurait une
%   formule du type
%
%    maximiser $\sum_{s} e_{s}(r_{s}) + \sum_{l}g_{l}(f^l)$
%
%    ou $g_{l}$ est donne par $g_{l}(f)= \xi D_{l}(f) + \lambda_{l}(f)$
%    ou $D_{l}=$ delai au link $l$ et $\lambda_{l}=$ loss ratio au
%    link $l$
%
%    alors sous cette hypothese, le taux de perte vu par flux est
%    $\sum_{l}\Phi_{s}^l \lambda_{l} f^l$ et le delai vu par un flux est
%    $\sum_{l}\Phi_{s}^l D_{l} f^l$
%
%   dans ce cas, le link loss ratio ne permet pas de converger vers un
%   loss ratio faible, sauf si on impose une asymptote au cout du link
%   (hypothese du theorem 2 de Golestani)
%
%   si la fonction de cout du reseau tend vers delta, alors on tend vers
%   le probleme SYSTEM donc vers rate proportional fairness
%
%   open issues: peut on casser le probleme: SYSTEM = USER + NETWORK
%   dans le cas de Golestani ou le cout reseau n'est pas delta
%
%
%

\section{Other Mechanisms for TCP Congestion Control}
\subsection{CUBIC}
\paragraph{Why CUBIC.}
The motivation of this variant is networks where the bandwidth-delay product is very large, i.e., when both the round trip time and the bit-rate available to one connection are large (``Long Fat Networks"). Consider the typical saw-tooth behaviour of TCP (and AIMD) illustrated in \fref{d32f8}. The available bit rate is $b=WL/\tau$ where $\tau$ is the round-trip time and $L$ is the packet length; also recall that the duration $T_0$ of one oscillation is  $T_0=W/2 \tau$. Thus
\ben T_0=  \frac{b\tau^2}{2L}
\een

For example,  if the round-trip time is $\tau=100$~msec, the available bit rate is 10Mb/s and the packet size is $1250$ bytes, then the time $T_0$  taken by one oscillation is equal to 5 sec, which is small. But if the available bit rate becomes 10Gb/s, $T_0$  becomes 1~h~23~mn ! In the latter case, the additive increase is too slow and chances are that the connection is over before completing even a single oscillation.  TCP CUBIC attempts to be a replacement for TCP Reno that alleviates this problem in long fat networks but behaves the same as TCP otherwise \cite{ha2008cubic, cubic2018rfc}.

\paragraph{Mechanisms of CUBIC.} CUBIC keeps multiplicative decrease and slow start as
with TCP Reno, and keeps the same rules for exiting congestion avoidance (on a loss event) as we have seen in \sref{sec-tcpalgodecon}. However, when a loss is detected by duplicate acknowledgements, the multiplicative decrease factor is 0.7 instead of 0.5 (i.e., when a loss is detected by duplicate acknowledgements, the congestion window is set to $0.7 W_{\max}$ where $W_{\max}$ is the value of the congestion window just before the loss event). The smaller window reduction intends to reduce the amplitude of the oscillations.
 \begin{figure}[h]
        \insfig{cubic1}{0.57}
        \mycaption{The window increase with TCP CUBIC.}
        \protect\label{fig-cubic1}
\end{figure}

But the major difference introduced by CUBIC is the replacement of additive increase during congestion avoidance phase.
The linear growth during additive increase is replaced by a cubic function, as illustrated in \fref{fig-cubic1}. Specifically, CUBIC uses a function $W(t)$ to compute the congestion window at time $t$ seconds after a loss event, given by
\be
W(t)=W_{\max}+a (t-K)^3
\ee where $a$ is a constant and $K$ is computed such that $W(0)=0.7 W_{\max}$. As illustrated in the figure, the growth of the congestion window is concave (slower than linear) as long as $W_{\max}$ is not attained, and convex (faster than linear) after that point. Thus, CUBIC increases slowly until the value $W_{\max}$ is reached; at this value, the increase is the slowest. This is indeed a very safe behaviour, since in stable network conditions, $W_{\max}$ is the region where congestion is likely to start, If, in contrast, the network has plenty of capacity when the congestion window reaches $W_{\max}$, then the window increase past this point is convex, i.e. increases fast. This is what allows CUBIC to accelerate the window growth in long fat networks.

%The idea is that if $W_{\max}$ is below the final value of the congestion window at convergence, the increase is fast. In contrast, if $W_{\max}$ is less than, or close to the final value, the increase is slow and thus very safe. The constant $a$ was chosen empirically and taken equal to $a=4~\mbox{sec}^{-3}$.

However, for small values of $W_{\max}$ and round-trip time, which occur in non long fat networks, it can happen that $W(t)$ is smaller than $W_{\mbox{\scriptsize AIMD}}(t)$, the value of the congestion window that would be obtained with additive increase, multiplicative increase (\fref{fig-cubic2}). Since CUBIC aims to obtain as much throughput as TCP Reno, CUBIC sets the value of the congestion window during the congestion avoidance phase to
\be
 W_{\mbox{\scriptsize CUBIC}}(t)=\max\left \{ W(t), W_{\mbox{\scriptsize AIMD}}(t)\right\}
 \label{eq-cubicca}
\ee
There is a small issue here.
The parameters of AIMD should be
adapted in order to account for the multiplicative decrease factor of $0.7$ instead of $0.5$. Indeed, TCP Reno increases the window size by $1$ packet per round trip time in congestion avoidance and multiplies the window by  $0.5$ at a loss event. Therefore, $W_{\mbox{\scriptsize AIMD}}(t)$ is not equal to the value of the congestion window with TCP Reno, but to the value obtained by a hypothetical AIMD that multiplies the window by $0.7$ at a loss event. In order to have a similar behaviour as TCP Reno, this version of AIMD should increases the window size by $r<1$ packet per round-trip time during congestion avoidance, in order to compensate for the smaller decrease. Specifically, we can determine $r$ by requiring that the loss throughput formula  gives the same value for both cases. The loss throughput formula in \eref{eq-tcplr} was derived for the case of TCP Reno; it can easily be modified to the case where the additive increase is $r$ packets per round-trip time and the decrease factor is $\beta$. We find, in this case, that \eref{eq-tcplr} should be modified such that $C$ is replaced by
 \be
 %C_{r,\beta}=\sqrt{ r\frac{1-\frac{\beta}{2}}{\beta}}
 C_{r,\beta}=\sqrt{ \frac{r(1+\beta)}{2(1-\beta)}}
 \label{eq-ltgen}
 \ee
 The case of TCP Reno corresponds to $r=1,\beta=0.5$ and gives $C_{1, 0.5}=\sqrt{\frac{3}{2}}$ as expected. For CUBIC, $\beta=0.7$ and the value of $r$ should be such that $C_{r,0.7}=C_{1,0.5}$ which gives
 $ r=3\frac{1-\beta}{1+\beta}=0.529$.
 In summary, CUBIC uses \eref{eq-cubicca} with $W_{\mbox{\scriptsize AIMD}}(t)=0.529 \frac{t}{RTT}$. When CUBIC uses $W_{\mbox{\scriptsize AIMD}}(t)$, i.e. when $W_{\mbox{\scriptsize AIMD}}(t)> W(t)$, we say that CUBIC is in the ``TCP-friendly" region.


 \begin{figure}[h]
        \insfig{cubic2}{0.99}
        \mycaption{The window increase with TCP CUBIC. Left: RTT is small, the cubic window increase is less than the additive increase and CUBIC uses  $W_{\mbox{AIMD}}(t)$ . Right: the converse holds when RTT is large, and CUBIC uses $W(t)$.}
        \protect\label{fig-cubic2}
\end{figure}

There is an additional mechanism called ``Fast Convergence", which is used to decrease the congestion window size at a loss event more severely when $W_{\max}$ is decreasing from one congestion avoidance phase to the next, see \cite{cubic2018rfc} for details.

CUBIC's increase of the congestion window is independent of the round-trip time, therefore we might expect it to remove the undesired bias of TCP Reno against large round trip time. However, as we see next, CUBIC does remove some of the bias against large round trip times, but not entirely. This is because the sending rate is proportional to the window and the inverse of the round-trip time: the increase in \emph{rate} is slower for large round trip times.

\paragraph{Analysis of CUBIC.}
We can extend the loss throughput formula of TCP Reno in \thref{theo-lf} to the cubic increase function, using the same method.
\begin{theorem}
 [Loss - throughput formula with cubic increase]
Consider a TCP connection with constant round trip time $\tau$ and
constant packet size $L$; assume that the network is stationary,
that the transmission time is negligible compared to the round trip
time, that losses are rare and that the time spent in slow start or fast
recovery is negligible; also assume that when a loss event occurs, the window is multiplied by a factor $\beta W$. Last, assume that the increase function is a cubic function, given by
\be
W(t)=W_{\max}+a (t-K)^3
\ee where $W_{\max}$ is the window size just before the loss event that triggered a new congestion avoidance phase, $a$ is a constant and $K$ is computed such that $W(0)=\beta W_{\max}$.

Then the average throughput $\theta$ (in
bits/s) and the average packet loss ratio $q$ are linked by the
relation
\begin{equation}
        \theta \approx \frac{L}{\tau^{0.25}} \frac{C_{\footnotesize \mbox{cubic}}}{q^{0.75}}
        \label{eq-cubiclt}
\end{equation}
with $C_{\footnotesize \mbox{cubic}} = \left(a\frac{3+\beta}{4(1-\beta)}\right)^{0.25}$.
\label{theo-lfc}
\end{theorem}
\pr The proof is similar to the proof of  \thref{theo-lf}. First observe that, with the conditions in the theorem, the window increase is always in the concave phase and the window size is periodic with period $K$. Let $W$ be the max window size.
The number of packets sent in one period is
\ben
   N=\frac{1}{\tau}\int_0^K \left(W+ a(t-K)^3\right)dt=\frac{1}{\tau}\left(
   WK-a\frac{K^4}{4}\right)
\een
The function $W(t)$ satisfies $W(0)=\beta W$ hence, after some algebra:
\ben
W=\frac{a}{1-\beta}K^3
\een
Combining with the previous equation gives
\ben
N=\frac{1}{\tau}\alpha K^4 \mbox{   with } \alpha=a\frac{3+\beta}{4(1-\beta)}
\een
There is one loss event per period of duration $K$ hence the loss probability is $q=1/N$.
Thus
\ben
K = \frac{\tau^{\frac14}}{(\alpha q)^{\frac14}}
\een
$N-1$ packets are successfully sent every $K$ seconds hence the throughput is
\ben
\theta = L\frac{N-1}{K}\approx L \frac{N}{K}=\frac{L}{q}\times \frac{(\alpha q)^{\frac14}}{\tau^{\frac14}}=\frac{L\alpha^{\frac14}}{\tau^{\frac14}q^{\frac34}}
\een
\qed

This theorem can be used to obtain a very coarse loss-throughput formula for CUBIC. First, using the values $a=0.4$ and $\beta=0.7$ we find $C_{\footnotesize \mbox{cubic}}=1.054$. Second, observe that CUBIC's window increase does not use the cubic function $W(t)$ but is linear when CUBIC operates in the TCP-friendly region; in particular, CUBIC's throughput is at least that of TCP Reno, which, according to \eref{eq-tcplr}, is $\frac{L}{\tau} \frac{1.22}{\sqrt{q}}$. An approximate formula is thus
\ben
\theta_{\footnotesize \mbox{CUBIC}} \approx \max\left\{\frac{L}{\tau} \frac{1.22}{\sqrt{q}}
   ,
      \frac{L}{\tau^{0.25}} \frac{1.054}{q^{0.75}}
      \right\}
\een
 \begin{figure}[h]
        \insfig{cubict}{0.5}
        \mycaption{The throughput of CUBIC, as predicted by \eref{eq-cubiclt}, as a function of the loss probability, for exponentially increasing values of the round-trip time. The red parallel lines (in log-log scale) represent the throughput of TCP Reno. The throughput of CUBIC is given by the blue line down to the point where it crosses the red line.}
        \protect\label{fig-cubict}
\end{figure}


 This formula is a bit coarse in that it ignores mixed cases, where CUBIC spends part of its time in the TCP friendly region, part in the concave region. This is visible in the knee of every curve in \fref{fig-cubict}, which occurs at the point where the max in \eref{eq-cubiclt} goes from one branch to the other. At such points the formula is not accurate.
As with TCP Reno, the throughput of CUBIC decreases with the round trip time; however, for large round trip times, the dependency is less pronounced.

\subsection{Active Queue Management}
\paragraph{The Bufferbloat Syndrom.} This is a non desirable side-effect of using loss as congestion indication. Consider, for example, a scenario where a number of sources share a bottleneck link and use loss based congestion control (such as TCP Reno or CUBIC). Assume sources increase their window size more or less simultaneously. \fref{fig-bloat} shows the evolution of the rate at which every source delivers packets to destination and of the round trip times. At the beginning, when windows are small, the rate is limited by the window and is equal to the window divided by round trip time; in this regime there is hardly any queuing and the round-trip time is constant equal to $RTT_{\min}$. As windows increase, the sum of source rates attains the link capacity (point A on \fref{fig-bloat}) and queuing is still small. At this point, since the buffer is very large, sources do not experience any loss and continue to increase their windows, up to point B. From point A to point B the queuing delay and hence the queuing delay increase, and the rate of delivery of packets to the destination remains the same, as it is determined by the bottleneck link.  At point B the link buffer fills up; beyond point B, the  link buffer starts to overflow, sources experience losses and the window (in average) does not increase any more.

\begin{figure}[h]
        \insfig{bloat}{0.7}
        \mycaption{Operating point of congestion control when there is a single bottleneck link and the buffer is very large. Adapted from \cite{cardwell2016bbr}.}
        \protect\label{fig-bloat}
\end{figure}

Point B is where loss-based congestion control operates in steady state for this scenario: there, the large link buffer is constantly oscillating from almost full to full and the round trip time is large. The buffer is not well utilized since it is constantly very full, which translates into large round trip times. In contrast, it would be better to operate around point A: there, the delivery rate is the same but the round trip time is much less. This discussion illustrates that, with loss-based congestion control, large buffers might be harmful. However, large buffers are needed for bursty traffic and do help avoid losses when there are temporary overloads due to bursts, not due to sustained congestion.

\paragraph{Random Early Detection (RED).}
The root cause of buffer bloat is that the congestion indicator is packet drop, which occurs only when buffers are full (``tail drop"). A mitigation method, called ``active queue management'', is therefore to drop packets in a network buffer well before the buffer is full. It replaces tail drop by an intelligent
admission decision for every incoming packet, based on a local
algorithm.  The algorithm uses a estimator of the long term load,
or queue length; in contrast, tail drop bases the dropping
decision on the instantaneous buffer occupancy only. The most widespread algorithm is
% \paragraph{Random Early Detection (RED)}
%
%We have said so fair that routers drop packets simply when their
%buffers overflow.  We call this the ``tail drop'' policy.  This
%however has three drawbacks:
%\begin{itemize}
%        \item  synchronization: assume there is a queue buildup in the
%        buffer. Then all sources using the buffer reduce their sending rate
%        and will slowly increase again (see Figure~\ref{d32f8}). The figure
%        illustrates that the time it takes for sources to reach their
%        maximum rate is in general much larger than the round trip time. As
%        a consequence, after a queue buildup at a router buffer, there may
%        be a
%        long period of reduced traffic. This is an example of global
%        synchronization. The negative effect is to reduce the long term
%        average utilization. It would be better to spread packet drops more
%        randomly between sources.
%
%        \item bias against bursty sources: during a queue buildup period, a
%        bursty source may suffer several consecutive drops. In that case,
%        the effect of TCP is a dramatic reduction in throughput.
%
%        After all, we might
%        think that it is good to penalize bursty sources since bursty
%        traffic uses up more resources in any queueing system. However,
%        there are many cases where an initially smooth source becomes
%        bursty because of multiplexing effects.
%
%        \item queueing delay: in the presence of bursty
%        sources, efficient utilization of the links requires a
%        large buffer (several thousands of packets). This in turn
%        may induce a large delay jitter (delay jitter is due to
%        the random nature of queuing times). Delay jitter is not
%        very
%        good for TCP applications (it increases the RTT estimate)
%        and is very bad for interactive audio flows. It would be
%        good to have smaller buffers, but then bursty sources
%        would suffer a lot.
%
%        Further, a large average queueing delay implies a smaller
%        throughput for TCP connections, due to TCP's bias against
%        large round trip times.
%
%\end{itemize}
%
%In order to overcome these problems was introduced the concept of
%``active queue management'', which we also call packet admission
%control.  The idea is to replace tail drop by an intelligent
%admission decision for every incoming packet, based on a local
%algorithm.  The algorithm uses a estimator of the long term load,
%or queue length; in contrast, tail drop bases the dropping
%decision on the instantaneous buffer occupancy only.
%
`Random Early Detection'' (RED)
\cite{braden-98-a}, which works as follows.
\begin{figure}
  % Requires \usepackage{graphicx}
  \insfig{d32-red-1}{0.5}
  \caption{Target drop probability as a function of average queue length, as used by RED}\label{d32-red-1}
\end{figure}

\begin{itemize}
        \item  For every incoming packet, an estimator \texttt{avg} of the average queue length is computed. It is
        updated at every packet arrival, using exponential
        smoothing:

 \begin{verbatim}
   avg :=  a * measured queue length + (1 - a) * avg
 \end{verbatim}
where $a$ is the smoothing parameter, between 0 and 1.
 \item The incoming packet is randomly dropped, according to:
 \begin{verbatim}
   if avg <  th-min accept the packet
   else if th-min < avg < th-max drop packet
          with probability p explained below
   else if th-max <= avg drop the packet
 \end{verbatim}
 where \texttt{th-min} and \texttt{th-max} are thresholds on the
 buffer content. By default, data is counted in packets.
\end{itemize}
The drop probability $p$ is computed in two steps. First, the target drop
probability $q$ is computed by using the function of \texttt{avg},
shown on \fref{d32-red-1}, which, for \texttt{avg} between \texttt{th-min} and \texttt{th-max}, is equal to
\begin{verbatim}
   q =  max-p * (avg - th-min) / (th-max - th-min)
 \end{verbatim}
Second, the \emph{uniformization} procedure is applied, as described now. We could simply take $p:=q$. This would
create packet drop events such that the interval between packet drops would tend to be geometrically
distributed (assuming $q$ varies slowly). The designers of RED
decided to have smoother than geometric drops. Specifically, they would like that this interval is uniformly distributed over $\{1, 2,..., \frac{1}{q}\}$, assuming
that $\frac{1}{q}$ is integer.

This can be achieved using the \emph{hazard rate} of the uniform distribution, which we introduce now. Let $T$ be the random variable equal to the packet
drop interval, i.e. $T=1$ if the first packet following the
previous drop is also dropped. The hazard rate $p(k)$ of $T$ is the probability that
a packet is dropped, given that there were $k-1$ packets since the previous packet drop occurred, i.e. $ p(k) = \P(T=k | T \geq k)$. If $T$ has a geometric distribution, $p(k)$ is independent of $k$. In contrast, if $T$ is uniformly distributed between $0$ and $a$, then $p(k)=\frac{1}{a-k+1}$, so that $p(k)$ increases when $k$ increases from $1$ to $a$ and $p(a)=1$. Furthermore, it can easily be shown that if we want drop packets such that the interval between packet drops has a specific distribution, it is sufficient to record the number $k$ elapsed since the last packet drop and to drop a packet with probability $p(k)$.
%
%decide, for every packet, drop packets drop pakcets  wesince the last
%drop where not dropped, i.e. $ p(k) = \P(T=k | T \geq k)$.
%
%
%e design goal is
%to have uniform packet drops, assuming $q$ would be constant.
%
%More precisely, let $T$ be the random variable equal to the packet
%drop interval, i.e. $T=1$ if the first packet following the
%previous drop is also dropped. If we let $p:=q$, then $T$ is a
%geometric random variable, namely $\P(T=k)=q(1-q)^{k-1}$ for
%$k=1,2,...$. Now, instead of this, we would like that $T$ is
%uniformly distributed over $\{1, 2,..., \frac{1}{q}\}$, assuming
%that $\frac{1}{q}$ is integer. Let $p(k)$ be the hazard rate, i.e., the probability that
%a packet is dropped, given that all $k-1$ packets since the last
%drop where not dropped, i.e. $ p(k) = \P(T=k | T \geq k)$.
% Note that the distribution of an integer random variable is
% entirely defined its hazard rate so if we drop a packet according to the hazard rate of the uniform distribution we enforce that the by the values of all $p(k)$ for $ k=0,1,2...$, so if we drop a packet
%%\mq{q-red-1}{Show this.}{$P(T=k)$ can be computed recursively from
%%the equation
%% \ben \bracket{
%%  \P(T=k)= p(k) \P(T \geq k)\\
%%  \P(T \geq k) = (1-p(k-1))\P(T\geq k-1)\\
%%  \P(T\geq 0)=1
%%  }
%% \een}
% For a uniform distribution, it comes easily that $ p(k) = \frac{\P(T=k)}{\P(T\geq k)}=\frac{q}{1- (k-1) q}$.
%
%
This is why RED computes \texttt{p} by letting
 \begin{verbatim}
 p = q/(1 - nb-packets * q)
  \end{verbatim}%
 where \texttt{nb-packets} is the number of packets accepted since the last
 drop.
%\mq{q-red-unif}{Show that this formula indeed provides a uniform
%distribution of times between packet drops when \texttt{target-p}
%is constant.}{
% Let $q=$\tetxtt{Target-p} and denote with $T$ the random variable equal to the packet
% drop interval, i.e. $T=1$ if the first packet following the previous droptbd
% }

%There are many variants to what we present here. For more details
%on RED the RED homepage maintained by Sally Floyd.

When RED is used with proper tuning of its parameters, bufferbloat can be reduced:  large buffers are used to absorb bursts but in average are not very full. Another effect is that long-lived flows experience a drop probability which depends only on the average amount of traffic present in the network, not on short term traffic fluctuations.

Last, a modified version of RED can be used to provide differentiated drop probabilities: a provider may modify the computed value of \texttt{p} in order to give larger drop probabilities to some flows, e.g. in order to give preference to its internal video streaming flows over external flows. This \emph{non neutral} behaviour is against the original ideas of the Internet, but is technically possible and probably in-use today.

%Active queue management can also be used to decouple packet
%admission from scheduling: with active queue management, we decide
%how many packets are accepted in a buffer. Then it is possible to
%schedule those accepted packets differently, depending on the
%nature of the applications. See \cite{Hurley2001mMay} for an
%example of this.

\subsection{Explicit Congestion Notification}

The Internet uses packet drops as the negative feedback signal for
end-to-end congestion control. Losses have a negative impact on the delay performance of TCP as the lost packets need to be retransmitted. A more friendly mechanism is to use an explicit congestion signal, as was originally done with Jain's ``Decbit". In the Internet, this is called
Explicit Congestion Notification (ECN).

\begin{figure}[h]
  % Requires \usepackage{graphicx}
  \insfig{ecn}{0.6}
  \caption{ECN uses a combination of fields in the IP and TCP headers}\label{fig-ecn}
\end{figure}


ECN works with TCP and uses a combination of fields in the IP and TCP headers. With ECN, when a router experiences congestion, it marks a ``Congestion Experienced" bit in the IP header (red bars in \fref{fig-ecn}). A TCP destination that sees such a mark in received packet sets the ECN Echo (ECE) flag in the TCP header of packets sent in the reverse direction (red crosses in \fref{fig-ecn}). When a TCP source receives a packet with the ECE flag from the reverse direction, it performs multiplicative decrease (e.g. reduces the window by 0.5 for TCP Reno, by 0.7 for CUBIC). The source then sets the Congestion Window Reduced (CWR) flag in the TCP headers. The receiver continues to set the ECE flag until it receives a packet with CWR set. The effect is that multiplicative decrease is applied only once per window of data (typically, multiple packets are received with ECE set inside one window of data).

ECN requires an active queue management such as RED to decide when to mark a packet with Congestion Experienced: instead of dropping a packet with the probability computed by RED, it marks it. If all parameters are properly set, a network with only ECN flows can avoid dropping any packet due to congestion in router buffers. This considerably increases the delay performance of every source.

An ECN-capable router needs to known whether a TCP source responds to ECN or not; if not, the router should drop a packet instead of marking it since otherwise the source will not react. This is achieved by using a bit in the IP header to indicate whether a source does support ECN or not; to avoid misconfigurations or frauds, it is combined with a ``Nonce Sum" mechanism (see RFC 3540 for details).


\subsection{Data Center TCP}
Data Center TCP (DCTCP) is a variant of TCP congestion control that is adapted to the TCP traffic seen in data centers. There, typically, there is a coexistence of many short flows with low latencies (user queries, consensus protocols) and \emph{jumbo} flows, i.e.,  with huge volume (backups, data base synchronization). One issue is the throughput performance of jumbo flows. In order to avoid packet losses, ECN is used; a remaining issue is the oscillations of the window due to the multiplicative decrease present in TCP Reno or CUBIC, which occurs for every congestion indication received (Note that for the conditions in a data center, where round-trip time is of the order of 10 microseconds, CUBIC behaves as TCP Reno). As illustrated in \fref{d32f8}, if $W$ is the window size at which congestion occurs, the actual window oscillates between $0.5W$ and $W$ (for TCP Reno) and is thus equal to $0.75W$ in average. The goal of DCTCP is to make the average window size very close to $W$.

To this end, DCTCP modifies TCP Reno as follows:

\begin{itemize}
  \item ECN must be used. A DCTCP source monitors the probability of congestion. In order to do this, the behaviour of ECN is modified such that a TCP receiver marks TCP acknowledgements with ECE flags in proportion to the number of received packets with Congestion Experienced marks. The receiver estimates the congestion probability as the proportion of acknowledgement packets with ECE flags. This differs from standard ECN behaviour, where feedback is a single bit of information per round trip time.

  \item When there is some non zero probability of congestion $q$, the multiplicative decrease factor is
  \ben
    \beta_{\footnotesize \mbox{DCTCP}}=\left( 1-\frac{q}{2}\right)
  \een
\end{itemize}
It follows that if there is little congestion ($q$ is small) then the window reduction is small, much smaller than $0.5$. %Since all other elements are unmodified, DCTCP is more aggressive than TCP Reno with ECN.

%Indeed, we can compute a loss-throughput formula for DCTCP, by using \eref{eq-ltgen} with $r=1$ and $\beta=\beta_{\footnotesize \mbox{DCTCP}}$. We obtain
%
%
%\ben \theta= \frac{L}{\tau}\frac{
%                  \sqrt{ 2-\frac{q}{2}}
%                  }{q}\approx \frac{L}{\tau}\frac{
%                  \sqrt{ 2}
%                  }{q}
%\een
%We see that the throughput is proportional to $1/q$ whereas for TCP Reno it is proportional to $1/\sqrt{q}$, which is much less.
%
It follows that DCTCP competes unfairly with other TCPs; it cannot be deployed outside data centers (or other controlled environments). Inside data centers, care must be given to separate the DCTCP flows (i.e. the internal flows) from other flows. This can be done with class based queuing, as we discuss later.


%\subsection{BBR and TCP Vegas}

%\section{Other Mechanisms for Congestion Control}

\section{TCP Friendly Applications}

It can no longer be assumed that the bulk of long lived flows on the
Internet is controlled by TCP only: some applications such as videoconferencing often use UDP as they have stringent delay requirements. Other applications use
QUIC, an application layer framework that runs on top of UDP and replaces TCP (and the secure socket layer).


The solution which is advocated by the IETF is that \emph{all
TCP/IP applications which produce long lived flows should mimic
the behaviour of a TCP source}. We say that such applications are
``TCP friendly''. In other words, all applications, except short
transactions, should behave, from a traffic point of view, as a
TCP source.
For applications that use QUIC this is simple, as QUIC uses packet numbering and acknowledgements and reproduces the same congestion control features as TCP.


But for
applications that do not use acknowledgements, how can TCP friendliness be defined~? One solution is the TCP-Friendly Rate Control protocol \cite{tfrc2008rfc}, which works as follows.
  \begin{enumerate}
    \item the application determines its sending rate using an
    adaptive algorithm;

        \item  the applications is able to provide feedback in the form of
        amount of lost packets; the loss ratio is the feedback provided to
        the adaptive algorithm;

        \item in average, the sending rate should be the same as for a TCP Reno
        connection experiencing the same average loss ratio. This is typically achieved by using the loss-throughput formula in \eref{eq-tcplr}.
  \end{enumerate}

There are many possible ways to implement an adaptive algorithm
satisfying the above requirements. This can be achieved only very
approximately; for more details on equation based rate control,
see \cite{Vojnovic2002mAugust}.

TCP friendly applications often use the so-called ``Real
Time transport Protocol'' (RTP). RTP is not a transport protocol like
TCP; rather, it defines a number of common data formats used by
multimedia application. RTP comprises a set of control messages,
forming the so called RTP Control Protocol (RTCP). It is in RTCP that
feedback about packet loss ratio is given. Then it is up to the
application to implement a TCP friendly rate adaptation algorithm.

%See also
%http://papers/tcp\_friendly.html for an original source on TCP
%friendly applications.
\section{Class Based Queuing}
In general, all flows compete in the Internet using the congestion control method of TCP (or are TCP-friendly). In controlled environments (e.g. a data center,  a smart grid, a TV distribution network, a cellular network) the competition can be modified by using per-class queuing.

With  per class queuing, routers classify packets (using an access list) and every class is guaranteed a minimum rate -- classes may exceed the guaranteed rate by borrowing from other  classes if there is spare capacity.
This is enforced in routers by implementing one dedicated queues for every class; the arbitration between classes is performed by a scheduler which implements some form of weighted fair queuing. With weighted fair queuing, every class, say $i$, at this router, is allocated a weight $w_i>0$ and the scheduler serves packets on an outgoing link in proportion of the weights. As a result, class $i$ is guaranteed to receive a long-term rate equal to $r_i=c \frac{q_i}{\sum_i q_j}$, where $c$ is the bit rate of the outgoing link and the summation is over all classes that share this link. For a concrete example of scheduler, see Deficit Round Robin \cite{shreedhar1996efficient}. If class $i$ is using less than its guaranteed rate $r_i$, the unused capacity is available to other
classes, also in proportion to their weights.

\begin{figure}[h]
  % Requires \usepackage{graphicx}
  \insfig{cbq}{0.8}
  \caption{A Network with Per-Class Queuing.}\label{fig-cbq}
\end{figure}


\fref{fig-cbq} illustrates a typical use of per-class queuing. Class 1 is for traffic sent by sensors at a constant rate; it is not congestion controlled and thus could cause congestion collapse. This is avoided here by allocating a sufficient rate to class 1 in every router, during a procedure called ``traffic engineering". Class 2 is ordinary TCP/IP traffic, which is congestion controlled. It is also allocated a rate at every router. The rate at which sources of class 2 may send depends on the state of other sources in classes 1 and 2. Assume for example that the sources are exactly as shown on the figure. The two TCP connections share the available capacity, which is 9~Mb/s on the leftmost link and 8~Mb/s on the other links. If their RTTs are identical, each of these TCP connections will achieve a rate of 4~Mb/s. This illustrates how the rate that is allocated to class 1, but is not entirely used, is made available to class~2. Observe that class~2 is guaranteed a rate of 7.5~Mb/s. In normal operation, it obtains more; in contrast, if there is a failure in some of the class 1 devices that causes them to send more traffic than they normally should, class~2 is guaranteed to receive at least 7.5~Mb/s in total: the schedulers achieve isolation of the classes.
\section{Summary}
\begin{enumerate}
\item Congestion control in the Internet is performed primarily by
TCP, using the principles of Additive Increase, Multiplicative
Decrease.
 \item Congestion control in the internet provides a
form of fairness close to proportional fairness, but with a
non-desired bias against long round trip times.
\item The historical version of congestion control was introduced in TCP-Reno. Its loss-throughput formula maps the packet loss probability to the achieved rate.
\item Many variants of TCP congestion control exist. CUBIC is one of them, it replaces the linear window increase by a cubic function in long fat networks.
\item With Explicit Congestion Notification, routers signal losses to TCP sources without dropping packets.
\item RED is a mechanism in routers to drop packets before
 buffers get full. It avoids buffer to fill up unnecessarily when
 the traffic load is high. A variant of it can also be used to compute the congestion signal sent by routers when ECN is used.
 \item Data Center TCP replaces the constant multiplicative decrease factor by a factor that depends on an estimation of the loss or congestion probability. It is more aggressive than regular TCP (is not TCP-friendly) and is typically deployed in closed environments.
 \item Non TCP applications that send large amounts of data should
 be TCP-friendly, i.e. not send more than TCP would. This can be
 achieved by mimicking TCP or by controlling the rate and enforce TCP Reno's loss throughput formula.
 \item Per-class queuing is used in closed environments to support non TCP-friendly traffic.

 \end{enumerate}
% \section{Applications to ATM}
%
% calcul distribue de Max Min (JR)
% ABR
% Erica+
%
% EPD, PPD
%
% exo: quiz to test the three phases
%
% exo: simuler TCP et comparer effet de nbre de hops contre delay
%
% exo: multicast papier de Crowcroft
%

% \chapter{Hop by Hop Congestion Control}
%   updown rule
%   fairness
%   deadlock
%
%   appli: credit based Kung, Cherbonnier, Ilias Iliadis
%   fairness de backpressure: laquelle ?
%   appli: implementation de ABR
%   802.3.x
%
%   reprendre l'exemple de JEC et discuter de l'implication




\appendix
\bibliography{leb}
\bibliographystyle{plain}

\end{document}
